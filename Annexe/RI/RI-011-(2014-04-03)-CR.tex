%-------------------------------------------------------------------------------
%
%     CHARGEMENT DES EXTENSIONS
%
%-------------------------------------------------------------------------------

\documentclass[11pt,fleqn]{report}
\usepackage{GarmirKhatch}

%-------------------------------------------------------------------------------
%     Informations spécifiques au document
%-------------------------------------------------------------------------------

\ZTitle{Système de gestion des transports}
\ZSubject{Compte-rendu du lundi 2014-03-04}
\ZVersion{2.1}
\ZDate{\today}
\ZAuthor{\Balde,\\\Cadon,\\\Gairoard,\\\Julien,\\\Lericolais,\\\Mezelle,\\\Pachy,\\\SuangaWeto,\\\Toure}

%-------------------------------------------------------------------------------
%     Contenu
%-------------------------------------------------------------------------------

\begin{document}

\ZMakeCover

\ZMakeInformations{
	% Historique des modifications
	% Version & Date & Auteur(s) & Modification(s)
	2.0 & 2014-03-03 & \Cadon & Création à partir de l'ancienne version \\
	\midrule
	2.1 & 2014-03-04 & \Pachy & Mise à jour en fonction de la réunion interne du 2014-03-03 \\
}{
	% Historique des approbations
	% Version & Date & Approbateur(s)
	\ZVersion & - & - \\
}{
	% Historique des validations
	% Version & Date & Responsable(s)
	2.1 & - & - \\
}

\ZMakeTableOfContents

\chapter{Points abordés}

\section{Introduction}

Ce document tient compte de la réunion ...
heure début : 14h00
heure fin : 14h55

Groupe de 8 personnes

Présent : MJ, Bien aimé, Ravi, Adrien, Anthony, Lionel
Absents : Ahoua, Rémy, Sory Ibrahima (dcd ?)

\section{Ordre du jour}

PAQ :
  - Méthodes de travail
  - Outils
  - Communication
  - Norme graphique
  - Architecture générale
    
Suite :
 - Date des prochaines réunions
 - Liste des tâches

\subsection{PAQ}

- Intro
  * Présentation des formats de document en LaTeX et Git
  * Amélioration des docs LaTeX (couleur hyperlien, espace titre)

- Communication
  * Mail à l'ensemble du groupe
  * Numéro de téléphone (uniquement en cas de nécessité)
  * Prévenir des tâches non tenues (deadline)

- Outils
  * Git
  * LaTeX (texte + schéma)
  
- Méthodes de Travail
  * 3 réunions / semaine
  * 2 chefs d'équipe de direction en binome
  * 3 sous équipes de 2 pers
  * On mixe les deux groupes dans les binomes

- Réunions à venir 14h 
  * lundi, mercredi, vendredi
  * On envoie un mail ODJ avant la réunion pour se préparer
  * le vendredi aprem on fait la liste des taches

\subsection{Liste des tâches}
  - Suite
    * OJ des réunions
    * Analyse de conception
    * Architecture générale --> Clients --> technologies utilisées
  
  - Devoirs maison
    * Autoformation Latex/Git
    * Complétion du PAQ
    * Reflexions architecturales
    * prochaine réu le 04/03, 14h, salles info du 3e

\end{document}
