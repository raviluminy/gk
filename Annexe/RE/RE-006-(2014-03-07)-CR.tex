%-------------------------------------------------------------------------------
%
%     CHARGEMENT DES EXTENSIONS
%
%-------------------------------------------------------------------------------

\documentclass[11pt,fleqn]{report}
\usepackage{GarmirKhatch}

%-------------------------------------------------------------------------------
%     Informations spécifiques au document
%-------------------------------------------------------------------------------

\ZTitle{Système de gestion des transports}
\ZSubject{Compte-rendu du 2014-02-26}
\ZVersion{2.0}
\ZDate{2014-03-05}
\ZAuthor{\Balde,\\\Cadon,\\\Gairoard,\\\Julien,\\\Lericolais,\\\Mezelle,\\\Pachy,\\\SuangaWeto,\\\Toure}

%-------------------------------------------------------------------------------
%     Contenu
%-------------------------------------------------------------------------------

\begin{document}

\ZMakeCover

\ZMakeInformations{
	% Historique des modifications
	% Version & Date & Auteur(s) & Modification(s)
	1.0 & 2014-03-07 & \Gairoard & Rédaction \\
	1.1 & 2014-03-08 & \Pachy & Mise en forme \\
}{
	% Historique des approbations
	% Version & Date & Approbateur(s)
	1.1 & 2014-03-09 & \Cadon \\
}{
	% Historique des validations
	% Version & Date & Responsable(s)
	1.1 & - & \Agopian \\
}

\chapter{Objectifs \& points abordés}

\section{Objectifs}
Ce document rend compte de la réunion du vendredi matin 2014-02-26, qui s'est déroulée au département informatique du campus universitaire de Saint-Charles à Marseille de 09h05 à 10h45, entre la maîtrise d'ouvrage \mo et son assistance à maîtrise d'ouvrage \amo.

\section{Acteurs}
Étaient présents lors de la réunion~:
\begin{itemize}
	\item \mo~:
	\begin{itemize}
		\item \Agopian
	\end{itemize}
	\item \amo~:
	\begin{itemize}
		\item \Cadon
		\item \Gairoard
		\item \Pachy
		\item \Julien
		\item \Mezelle
		\item \SuangaWeto
		\item \Toure
	\end{itemize}
\end{itemize}
Excusés~:
\begin{itemize}
	\item \Balde
	\item \Lericolais
\end{itemize}

\section{Documents attendus}

M. \Agopian rappelle qu'il n'a toujours pas reçu le CdCF dans les délais contrairement à ce qui était annoncé par \amo, à savoir dimanche 9 mars dans la soirée.

% ==============================================================================
\section{Résumé de la semaine de travail}

Il est proposé par \amo que le récapitulatif de chaque partie développée dans le D.A.T soit présentée par les binômes concernés.

% ------------------------------------------------------------------------------
\subsection{}

% Fin de la sous-section []
% ------------------------------------------------------------------------------

% Fin de la section [Résumé de la semaine de travail]
% ==============================================================================

%- Rappel de l'ordre du jour par Rémi
%  * M. Agopian précise qu'il n'a toujours pas reçu les nouvelles versions du CdC des deux groupes alors qu'ils étaient prévu pour dimanche soir 02/03/2014.

%- Résumé de la semaine de travail (répartition par binôme des sections du DAT)
%  * Réunions :
%    1) PAQ + Màj du CdC
%    2) Discussion autour du schéma d'architecture et création du DAT

%- Explication de chaque partie composant le DAT par le groupe de binonme associé

%- Corrections apportées au DAT :
%  * Suppression de la partie "Plannification" au profit d'une référence au CdC
%  * Partie 2.1 : Schéma + Texte explicite
%    --> glossaire (waybill*)
%  * Partie authentification à paraître dans le DAT
%  * Partie données :
%    version du SGBDR PostgreSQL ?
%    "[...]se fait grâce[..]" au présent !
%  * Partie sauvegarde : "[...]tous les six heures[...]" (inutile, réservé pour l'exploitation)
%  * Partie synchronisation :
%    refaire la conception
%    schéma + texte explicite

%-------------------------------------------------------------------

%- A faire :
%  * Remettre les CdC et le PAQ à M. Agopian
%  * Demander la date et l'heure de la soutenance du projet de fin d'étude à Jean-Luc Massat
%  * Corriger les parties du DAT (suppression, coquilles, ajout...)
%  * GANTT du projet
%  * Faire l'architecture en groupe de binôme/trinôme
%-------------------------------------------------------------------

\end{document}
