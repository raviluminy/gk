%-------------------------------------------------------------------------------
%
%     CHARGEMENT DES EXTENSIONS
%
%-------------------------------------------------------------------------------

\documentclass[11pt,fleqn]{report}
\usepackage{GarmirKhatch}

%-------------------------------------------------------------------------------
%     Informations spécifiques au document
%-------------------------------------------------------------------------------

\ZTitle{Système de gestion des transports}
\ZSubject{Ordre du jour du 2014-02-21}
\ZVersion{2.0}
\ZDate{2014-03-05}
\ZAuthor{\Balde,\\\Cadon,\\\Gairoard,\\\Julien,\\\Lericolais,\\\Mezelle,\\\Pachy,\\\SuangaWeto,\\\Toure}

%-------------------------------------------------------------------------------
%     Contenu
%-------------------------------------------------------------------------------

\begin{document}

\ZMakeCover

\ZMakeInformations{
	% Historique des modifications
	% Version & Date & Auteur(s) & Modification(s)
	1.0 & 2014-02-20 & \Toure & Rédaction \\
	2.0 & 2014-03-05 & \Cadon & Migration \\
}{
	% Historique des approbations
	% Version & Date & Approbateur(s)
	2.0 & 2014-03-05 & \Cadon \\
}{
	% Historique des validations
	% Version & Date & Responsable(s)
	2.0 & - & \Agopian \\
}

\chapter{Objectifs \& points abordés}

\section{Objectifs}
Ce document présente l'Ordre du jour de réunion du 2014-02-21 prévue à 10:30 en salle n°3 au sous sol du 5ième batiment du campus universitaire de Saint-Charles à Marseille, entre la maîtrise d'ouvrage \mo et son assistance à maîtrise d'ouvrage \amo.
\\
L'objectif de cette seconde réunion sera de présenter une deuxième version du cahier des charges, afin de pallier à tout élément incorrect ou manquant.

\section{Points abordés}
Au cours de cette rencontre, seront au minimum abordés les grands points suivants~:
\begin{enumerate}
	\item présentation de la version 1.2.7 du cahier des charges conforme à la norme AFNOR NF X50-15 afin d'être corrigé~;
	\item présentation du Plan d'Assurance Qualité (PAQ) modifié~;
	\item questions~:
	\begin{enumerate}
		\item contraintes supplémentaires~?
		\item date limite de dépôt de dossier de réponse~?
	\end{enumerate}
\end{enumerate}
En outre, le droit est laissé, pendant la réunion, d'aborder d'autres points non définis dans ce document, en vue d'améliorer la compréhension du projet.

\end{document}
