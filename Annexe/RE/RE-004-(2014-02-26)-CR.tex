%-------------------------------------------------------------------------------
%
%     CHARGEMENT DES EXTENSIONS
%
%-------------------------------------------------------------------------------

\documentclass[11pt,fleqn]{report}
\usepackage{GarmirKhatch}

%-------------------------------------------------------------------------------
%     Informations spécifiques au document
%-------------------------------------------------------------------------------

\ZTitle{Système de gestion des transports}
\ZSubject{Compte-rendu du 2014-02-26}
\ZVersion{2.0}
\ZDate{2014-03-05}
\ZAuthor{\Balde,\\\Cadon,\\\Gairoard,\\\Julien,\\\Lericolais,\\\Mezelle,\\\Pachy,\\\SuangaWeto,\\\Toure}

%-------------------------------------------------------------------------------
%     Contenu
%-------------------------------------------------------------------------------

\begin{document}

\ZMakeCover

\ZMakeInformations{
	% Historique des modifications
	% Version & Date & Auteur(s) & Modification(s)
	1.0 & 2014-02-27 & \Gairoard & Rédaction \\
	2.0 & 2014-03-05 & \Cadon & Migration \\
}{
	% Historique des approbations
	% Version & Date & Approbateur(s)
	1.0 & 2014-02-24 & \Balde, \Pachy \\
	2.0 & 2014-03-05 & \Cadon \\
}{
	% Historique des validations
	% Version & Date & Responsable(s)
	1.0 & 2014-02-14 & \Agopian \\
}

\chapter{Objectifs \& points abordés}

\section{Objectifs}
Ce document rend compte de la réunion du vendredi matin 2014-02-26, qui s'est déroulée au département informatique du campus universitaire de Saint-Charles à Marseille de 13:38 à 16:27, entre la maîtrise d'ouvrage \mo et son assistance à maîtrise d'ouvrage \amo.
\\
À cette fin, rappelons que cette réunion se déroule dans le cadre de la demande de prestation d'assistance à maîtrise d'ouvrage (AMO) dont l'objectif est de fournir un cahier des charges répondant aux besoins de la maîtrise d'ouvrage.

\section{Acteurs}
Étaient présents lors de la réunion (et par ordre alphabétique)~:
\begin{enumerate}
	\item représentant \mo~:
	\begin{enumerate}
		\item M. \Agopian (maîtrise d'ouvrage),
	\end{enumerate}
	\item représentant \amo~:
	\begin{enumerate}
		\item M. \Cadon (chef de projet),
		\item M. \Gairoard (assistant, conseiller et secrétaire),
		\item M. \Pachy (assistant, conseiller et secrétaire).
	\end{enumerate}
\end{enumerate}
Étaient absents lors de la réunion (et par ordre alphabétique)~:
\begin{enumerate}
	\item représentant \amo~:
	\begin{enumerate}
		\item M. \Balde (assistant)~: motif n.c.,
		\item Mlle. \Toure (assistant, conseiller et secrétaire)~: en déplacement à l'étranger.
	\end{enumerate}
\end{enumerate}

\section{Points abordés}

\subsection{Ordre du jour de la réunion du 2014-02-26}
L'ordre du jour n'ayant pas été précisé à l'avance, il a été défini en début de réunion.
\\
M. \Agopian précise qu'il était préférable de convenir de cette réunion plutôt que d'échanger par mail, au vu des modifications qui devront être apportées au cahiers des charges communiqué vendredi dernier.
\\
Les représentants de \amo ont également présenté la nouvelle version du CdCF (version 1.3.1) afin d'avoir un premier retour de la maîtrise d'ouvrage.

\subsection{Documents}
Avant d'évoquer en détail les modifications à effectuées au sein du CdCF, M.  \Agopian a tiré deux exemplaire des manuels d'utilisation des appareils de communication  utilisés en mission par \mo. Le GPS GARMIN eTrex® 20 fut présenté à l'assistante à maîtrise d'ouvrage avec une brève démonstration des principales fonctionnalités du terminal.

\subsection{Cahier des charges v1.2.7}
Suite à la présentation de la version 1.2.7 du cahier des charges, plusieurs corrections ont été apportées.

\subsubsection{Corrections apportées au Cahier des charges v1.2.7}

\paragraph{Contexte métier}
Au vu du CdCF de vendredi dernier, M. \Agopian a fait part de tout d'abord bien définir pourquoi le CdCF est important, puis les objectifs, la stratégie mise en place et les enjeux. Cela permet dès le début de la lecture du document, de se \og plonger \fg{} dans le contexte global de \mo tout en restant général.

\paragraph{Forme du Cahier des charges}
Une réorganisation concernant l'agencement des parties du CdFC a été évoquée afin de permettre une meilleure lisibilité de l'information et non se conforter au plan décrit dans la norme AFNOR NF X50-151 suivie.

\paragraph{Contenu du CdCF}
Concernant l'illustration, M. \Agopian regrette qu'il n'y ai pas plus de schéma afin d'expliciter les dires par écrit et rendre ainsi la lecture plus facile.

\paragraph{Installation, maintenance et support}
La maîtrise d'ouvrage nous a également demandé de préciser si la solution proposée par les soumissionnaires comprend l'installation de matériels, services.
Les différents types de maintenances (correctives/évolutive) et les critères de supports ont été évoqués.

\paragraph{Gestion des droits d'utilisateurs}
La gestion des droits d'utilisateurs ont été redéfinit pour permettre l'attribution d'un droit à plusieurs groupes ou un droit à un utilisateur particulier. Les types de droits sont classés en trois catégories~:
\begin{enumerate}
	\item autorisé~;
	\item indéfini~;
	\item interdit~;
\end{enumerate}

\paragraph{Schéma du contexte métier}
Un schéma du contexte métier fut réalisé sur le white-board de la salle de réunion représentant~:
\begin{enumerate}
	\item les utilisateurs~;
	\item les outils de communication (GDM, portable, BGAN...)~;
	\item les serveurs (central et locaux avec les architectures assurant la haute disponibilité)~;
	\item les interactions avec les serveurs (mode de synchronisation, privilèges des droits d'accès)~;
	\item la solution de sauvegarde externalisée~;
	\item les bases de données principales (utilisateurs, missions, prestataires et celle relative à la documentation).
\end{enumerate}

\paragraph{Gestion des transports}
Si la gestion de transports était bien compris et défini, la maîtrise d'ouvrage explique qu'il n'en reste pas moins important de gérer la révisions des véhicules ainsi que de numériser les documents relatifs aux transporteurs.

\paragraph{Tableaux de bord}
Le format d'exportation des tableaux de bord devra respecter un format préalablement défini.

\paragraph{Contraintes applicatives}
La solution proposée par les soumissionnaires devra utiliser un langage de programmation non exotique reposant sur une technologie pérenne. Une architecture n-tiers fut évoquée en opposition à l'utilisation d'une architecture monolithique.

\section{Objectifs en vue de la prochaine réunion}
Pour la prochaine réunion, il a été demandé de finir la réalisation du CdCF. Une version finale est donc attendue, et l'ordre du jour de la réunion à venir sera donc la validation de ce dernier.

\end{document}
