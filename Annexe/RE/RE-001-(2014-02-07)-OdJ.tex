%-------------------------------------------------------------------------------
%
%     CHARGEMENT DES EXTENSIONS
%
%-------------------------------------------------------------------------------

\documentclass[11pt,fleqn]{report}
\usepackage{GarmirKhatch}

%-------------------------------------------------------------------------------
%     Informations spécifiques au document
%-------------------------------------------------------------------------------

\ZTitle{Système de gestion des transports}
\ZSubject{Ordre du jour du 2014-02-07}
\ZVersion{2.0}
\ZDate{2014-03-05}
\ZAuthor{\Balde,\\\Cadon,\\\Gairoard,\\\Julien,\\\Lericolais,\\\Mezelle,\\\Pachy,\\\SuangaWeto,\\\Toure}

%-------------------------------------------------------------------------------
%     Contenu
%-------------------------------------------------------------------------------

\begin{document}

\ZMakeCover

\ZMakeInformations{
	% Historique des modifications
	% Version & Date & Auteur(s) & Modification(s)
	2.0 & 2014-03-05 & \Cadon & Rédaction \\
}{
	% Historique des approbations
	% Version & Date & Approbateur(s)
	2.0 & 2014-03-05 & \Cadon \\
}{
	% Historique des validations
	% Version & Date & Responsable(s)
	2.0 & - & \Agopian \\
}

\chapter{Objectifs \& points abordés}

\section{Objectifs}
Premier contact entre la maîtrise d'ouvrage \mo et son assistance à maîtrise d'ouvrage \amo en vue d'établir les bases du cahier des charges du projet en cours.
\\
Cette première réunion a donc pour objectif de comprendre le métier du demandeur, ses besoins relatifs au projet, ses objectifs, ses enjeux et ses contraintes.

\section{Points abordés}
Tout d'abord, et à cette fin, les points suivants seront abordés~:
\begin{enumerate}
	\item découpe du projet en différentes sous-parties les plus distinctes possibles et hiérarchisation de ces dernières par priorité~;
	\item un processus existe-t-il déjà pour gérer les éléments suivants et si oui, quelle(s) donnée(s) et/ou documentation(s) nécessaire(s) fournit-il (tout exemple sera le bienvenu)~:
	\begin{enumerate}
		\item livraisons,
		\item prestataires de services,
		\item chauffeurs,
		\item incidents,
		\item cartographie~;
	\end{enumerate}
	\item profils utilisateurs, besoins et droits associés~;
	\item documents sous forme numérique devant être conservé par le secrétariat central (format, durée etc.)~;
	\item capacité, continuité et disponibilité de la solution logicielle~;
	\item besoins et contraintes techniques~:
	\begin{enumerate}
		\item le cas échéant, dans quelle infrastructure déjà existante doit-elle s'intégrer~?
		\item sur quel(s) système(s) d'exploitation doit-elle fonctionner~?
		\item depuis quel type de matériel doit-elle être accessible (ordinateur de bureau, ordinateur portable, téléphone etc.)~?
	\end{enumerate}
\end{enumerate}
En outre, pendant cette première réunion seront abordées des notions plus techniques visant l'établissement d'un plan d'assurance qualité. Dans cette optique, les éléments annexes suivants seront abordés~:
\begin{enumerate}
	\item date(s), horaire(s) et lieu(x) de rendez-vous entre \mo et \amo. Proposé de façon hebdomadaire, les vendredis 07, 14 et 21 février 2014, à 10:30 du matin (UTC+01:00), en salle CH301 du campus de la faculté de Saint-Charles, à Marseille~;
	\item modèle de cahier des charges, d'ordre du jour et de compte rendu~;
	\item définition des outils utilisés~:
	\begin{enumerate}
		\item planification Gantt,
		\item formats informatiques des documents (PDF par exemple).
	\end{enumerate}
\end{enumerate}
Enfin, le droit est laissé, pendant la réunion, d'aborder d'autres points non définis dans  ce document, en vue d'améliorer la compréhension du projet.

\end{document}
