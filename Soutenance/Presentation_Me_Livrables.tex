\subsection{Livrables}
\begin{frame}
	\frametitle{Livrables}
	\begin{center}
	 \includegraphics[scale=0.5]{Images/livrables}
	\end{center}
\end{frame}

% Conclusion~: tout ceci constitue la réponse technique aux besoins spécifiés dans le CdCF, de même que le DAT qui spécifie dans le détail l'architecture de ces solutions.
%% Cahier de recette (spécifié comme livrable par le CdCF).
%%% Problématique~: Tester la solution, ses fonctionnalité, ses performances, son intégration, etc.
%%%% Procédure d'installation technique (spécifié comme livrable par le CdCF).

\subsubsection[Nature des documents]{Nature des documents}

\begin{frame}
\frametitle{Nature des documents}
\begin{columns}[c]
\begin{column}{12cm}
\begin{block}{\textbf{Quelle sont les documents attendu par le CdCF~?}}
\begin{itemize}
\item \textcolor{green}{$\checkmark$} \textbf{DAT (Dossier d'Architecture Technique)}
\item \textcolor{red}{$\times$} \textbf{PTI (Procédure d'Installation Technique)}
\item \textcolor{green}{$\checkmark$} \textbf{PdR (Procédure de Recette)}
\begin{itemize}
\item \textcolor{green}{$\checkmark$} Tests d'interface graphique
\item \textcolor{green}{$\checkmark$} Tests unitaires sur l'annuaire
\item \textcolor{green}{$\checkmark$} Tests unitaires sur la base de données
\item \textcolor{red}{$\times$} Tests fonctionnels validé par le client
\end{itemize}
\item \textcolor{red}{$\times$} \textbf{Dossier d'exploitation}
\item \textcolor{red}{$\times$} \textbf{Plan de continuité de services}
\end{itemize}
\end{block}
\end{column}
\end{columns}
\end{frame}

