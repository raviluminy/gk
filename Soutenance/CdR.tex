\begin{frame}
\frametitle{Les tests}
\begin{block}{\textbf{Pourquoi les tests sont importants dans ce projet}}
\begin{itemize}
\item l'outil doit être fiable
\item les fonctionnalités du cahier des charges doivent être opérationnelles
\end{itemize}
\end{block}
\end{frame}

\begin{frame}
\frametitle{}
\begin{block}{\textbf{Différents types de tests ont été réalisé : }}
\begin{itemize}
\item les tests sur l'interface graphique,
\item les tests unitaires,
\item les tests sur le système de gestion de données.
\end{itemize}
\end{block}
\begin{block}{\textbf{Configuration des tests : }}
\begin{itemize}
\item le framework Qt intègre une librairie de test : QTestlib
\item les tests sont lancés à partir de l'éxecutable
\item différentes options permettent de lancer les différents tests ( \emph{-testldap} )
\end{itemize}
\end{block}
\end{frame}

\begin{block}{\textbf{Test d'un champ sur l'onglet Réquisition }}
\begin{center}
\newcolumntype{R}[1]{>{\raggedleft\arraybackslash }b{#1}}
\newcolumntype{L}[1]{>{\raggedright\arraybackslash }b{#1}}
\newcolumntype{C}[1]{>{\centering\arraybackslash }b{#1}}
\begin{tabular}{|L{2.7cm}|L{7cm}|}
\hline \emph{Cas de test :} & Test-Graphique-08  \\
\hline \emph{Titre :} & testCountryCode   \\
\hline \emph{Objectif :} & Vérifier le bon fonctionnement du champ du code pays dans l'onglet réquisition   \\
\hline \emph{Procédure :} & Cliquer dans le champ du code pays, le remplir avec une chaîne et vérifier que la chaîne correspond   \\
\hline \emph{Données de test :} & On rempli avec la chaîne ``Fr''   \\
\hline \emph{Config :} & Using QtTest library 5.2.1, Qt 5.2.1   \\
\hline \emph{Résultat :} & PASS   \\
\hline 
\end{tabular} 
\end{center}
\end{block} 

\begin{frame}
\begin{block}{\textbf{Les tests graphiques permettent de }}
\begin{itemize}
\item tester les zones de saisies
\item tester les validateurs
\end{itemize}
\end{block}
\end{frame}

\begin{block}{\textbf{Test d'un champ sur l'onglet Providers }}
\begin{center}
\newcolumntype{R}[1]{>{\raggedleft\arraybackslash }b{#1}}
\newcolumntype{L}[1]{>{\raggedright\arraybackslash }b{#1}}
\newcolumntype{C}[1]{>{\centering\arraybackslash }b{#1}}
\begin{tabular}{|L{2.7cm}|L{7cm}|}
\hline \emph{Cas de test :} & Test-Graphique-31  \\
\hline \emph{Titre :} & testWaybillCountryCode   \\
\hline \emph{Objectif :} & Vérifier le bon fonctionnement du champ du code pays du prestataire.   \\
\hline \emph{Procédure :} & Cliquer dans le champ, le remplir avec une chaîne et vérifier que la chaîne correspond   \\
\hline \emph{Données de test :} & On rempli avec la chaîne ``FRA''   \\
\hline \emph{Config :} & Using QtTest library 5.2.1, Qt 5.2.1   \\
\hline \emph{Résultat :} & PASS   \\
\hline 
\end{tabular} 
\end{center}
\end{block}

\begin{frame}
\begin{block}{\textbf{Les tests unitaires de l'annuaire LDAP permettent de}}
\begin{itemize}
\item tester les connexions au serveur
\item tester l'authentification
\item vérifier l'implémentation des groupes
\item tester les droits en\begin{itemize} \item lecture
					  \item écriture
					  \item ajout
					  \item suppression
			  \end{itemize}		
\end{itemize}
\end{block}
\end{frame}

\begin{block}{\textbf{Test d'une connexion à l'annuaire }}
\begin{center}
\newcolumntype{R}[1]{>{\raggedleft\arraybackslash }b{#1}}
\newcolumntype{L}[1]{>{\raggedright\arraybackslash }b{#1}}
\newcolumntype{C}[1]{>{\centering\arraybackslash }b{#1}}
\begin{tabular}{|L{2.7cm}|L{7cm}|}
\hline \emph{Cas de test :} & Test-Unitaire-01  \\
\hline \emph{Titre :} & testConnection   \\
\hline \emph{Objectif :} & Vérifier une connexion à l'annuaire LDAP   \\
\hline \emph{Procédure :} & Initialisation d'une connexion avec un nom d'hôte et un port existant   \\
\hline \emph{Données de test :} & hostname : localhost, port : 389   \\
\hline \emph{Résultat :} & PASS   \\
\hline 
\hline \emph{Cas de test :} & Test-Unitaire-02  \\
\hline \emph{Titre :} & testConnectionBadHostname   \\
\hline \emph{Objectif :} & Vérifier une connexion à l'annuaire LDAP   \\
\hline \emph{Procédure :} & Initialisation d'une connexion avec un faux nom d'hôte et un port existant   \\
\hline \emph{Données de test :} & hostname : toto, port : 389   \\
\hline \emph{Résultat :} & FAIL comme prévu   \\
\hline 
\end{tabular} 
\end{center}
\end{block}

\begin{frame}
\begin{block}{\textbf{Tous les tests sont visualisables dans le cahier des recettes}}
\end{block}
\end{frame}

