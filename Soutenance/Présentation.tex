% **************************** %
%          Slides GK           %
% **************************** %

\documentclass[10pt,fleqn]{beamer}
 
% **************************** %
%            Package           %
% **************************** %

% \usepackage{GarmirKhatch}
\usepackage{etex} % pour éviter erreurs de compilation avec tikz
\reserveinserts{20}

\usepackage[utf8]{inputenc}
\usepackage[T1]{fontenc}
\usepackage[french,english]{babel}
\usepackage[french]{layout}
\usepackage{lmodern}
\usepackage{ragged2e}
\usepackage{fancyhdr}
\usepackage{verbatim}
\usepackage{graphicx}
\usepackage{wrapfig}
\usepackage{url}
\usepackage{hyperref}
\usepackage{amsmath}
\usepackage{multirow}
\usepackage{multicol}
\usepackage{array}
\usepackage{colortbl}
\usepackage{comment}
\usepackage{tikz}
\usepackage{tikz-uml}
\usetikzlibrary{positioning, shadows}
\newcommand{\mo}{\textsc{Garmir~Khatch}}

% **************************** %
%          Préambule           %
% **************************** %
% Option pdf
\hypersetup{
      %pdfpagemode = FullScreen,
      pdfauthor   = {AMU FSI 2014},
      pdftitle    = {Présentation du projet TMS pour \mo},
      pdfsubject  = {Système de gestion des transports},
      pdfkeywords = {AMO, GK, TMS, CdC, DAT, PTI}
}
% Thème du pdf 
\usetheme{Warsaw}
% Logo de l'université d'Aix-Marseille
%\logo{\includegraphics[height=6mm]{logo}}
% Affiche les notes
%\setbeameroption{show notes}
% Blocks arrondies et ombrés
\setbeamertemplate{blocks}[rounded][shadow=true] 
% Balle pour la liste d'items
\setbeamertemplate{itemize item}[ball]
% Triangle pour la liste de sous items
\setbeamertemplate{itemize subitem}[triangle]
% Affiche l'ensemble du frame en gris clair
\beamertemplatetransparentcovered
% Faire apparaître un sommaire avant chaque section
\AtBeginSection[]{
	\begin{frame}
		\frametitle{Sommaire}
		\tableofcontents[currentsection, hideallsubsections]
	\end{frame}
}

% **************************** %
%        Page de garde         %
% **************************** %
\title[]{{\Large \textsc{\mo \\ Système de gestion des transports}}}
\author[\textsc{\mo - Système de gestion des transports}]{M2 FSIL - FSI}
\institute{Encadrant : M. Roland \textsc{Agopian}\\
Faculté des Sciences d'Aix-Marseille Université\\
Campus de Luminy}
\date{\scriptsize{ 27 mars 2014}}

% **************************** %
%       Corps du document      %
% **************************** %
\begin{document}
 
% **************************** %
%            Entête            %
% **************************** %
\begin{frame}
\begin{figure}
\centering
\includegraphics[scale=0.52]{Images/EnTeteSciences}
\end{figure}
\titlepage
\end{frame}

\begin{frame}
\frametitle{Sommaire}
\tableofcontents[hideallsubsections]
\end{frame}

\section[Mission d'AMO~: recueil des besoins (CdCF)]{Mission d'AMO~: recueil des besoins (CdCF)}

% ------------------------------------------------------------------------------
\subsection{Norme NF X50-151}

% ▿▿▿▿▿▿▿▿▿▿▿▿▿▿▿▿▿▿▿▿▿▿▿▿▿▿▿▿▿▿▿▿▿▿▿▿▿▿▿▿▿▿▿▿▿▿▿▿▿▿▿▿▿▿▿▿▿▿▿▿▿▿▿▿▿▿▿▿▿▿▿▿▿▿▿▿▿▿
\begin{frame}
\tableofcontents[subsectionstyle=show/shaded/hide, subsubsectionstyle=hide, sectionstyle=show/hide]
\end{frame}
% ▵▵▵▵▵▵▵▵▵▵▵▵▵▵▵▵▵▵▵▵▵▵▵▵▵▵▵▵▵▵▵▵▵▵▵▵▵▵▵▵▵▵▵▵▵▵▵▵▵▵▵▵▵▵▵▵▵▵▵▵▵▵▵▵▵▵▵▵▵▵▵▵▵▵▵▵▵▵

% ▿▿▿▿▿▿▿▿▿▿▿▿▿▿▿▿▿▿▿▿▿▿▿▿▿▿▿▿▿▿▿▿▿▿▿▿▿▿▿▿▿▿▿▿▿▿▿▿▿▿▿▿▿▿▿▿▿▿▿▿▿▿▿▿▿▿▿▿▿▿▿▿▿▿▿▿▿▿
\begin{frame}
\frametitle{Contenu (non exhaustif) de la norme NF X50-151}

\begin{block}{Pourquoi une norme ?}
\begin{itemize}
    \item Nous aider
    \item Être \og{}standard\fg{}
\end{itemize}

\end{block}

\begin{exampleblock}{1/3 - Présentation du projet}
\begin{itemize}
    \item Le projet en lui même \small(finalités, retours sur investissements)
    \item Contexte \small(études réalisées, suites, nature des prestations, confidentialité)
    \item Énoncé du besoin \small(du point de vue de l'utillisateur)
    \item Environnement \small(utilisateurs, caractéristiques et contraintes matérielles)
\end{itemize}
\end{exampleblock}

\begin{exampleblock}{2/3 - Expression fonctionnelle du besoin}
\begin{itemize}
    \item Fonctions principales (métier)
    \item Fonction complémentaires
    \item Critères d'appréciation de la solution fournie
\end{itemize}
\end{exampleblock}

\end{frame}
% ▵▵▵▵▵▵▵▵▵▵▵▵▵▵▵▵▵▵▵▵▵▵▵▵▵▵▵▵▵▵▵▵▵▵▵▵▵▵▵▵▵▵▵▵▵▵▵▵▵▵▵▵▵▵▵▵▵▵▵▵▵▵▵▵▵▵▵▵▵▵▵▵▵▵▵▵▵▵

% ▿▿▿▿▿▿▿▿▿▿▿▿▿▿▿▿▿▿▿▿▿▿▿▿▿▿▿▿▿▿▿▿▿▿▿▿▿▿▿▿▿▿▿▿▿▿▿▿▿▿▿▿▿▿▿▿▿▿▿▿▿▿▿▿▿▿▿▿▿▿▿▿▿▿▿▿▿▿
\begin{frame}
\frametitle{Contenu (suite)}

\begin{exampleblock}{3/3 - Cadre de réponse}
Pour chaque fonction :
\begin{itemize}
    \item Description de la solution proposée
    \item Mise en évidence des critères d'appréciation
    \item Coût
\end{itemize}
Pour l'ensemble :
\begin{itemize}
    \item Écarts par rapport au CdCF
    \item Mesures prises pour respecter les contraintes
    \item Documentation technique \& utilisateur
    \item Modularité
    \item Fiabilité
    \item Évolutions technologiques
\end{itemize}
\end{exampleblock}

\end{frame} % Fin de la frame [Contenu (suite)]
% ▵▵▵▵▵▵▵▵▵▵▵▵▵▵▵▵▵▵▵▵▵▵▵▵▵▵▵▵▵▵▵▵▵▵▵▵▵▵▵▵▵▵▵▵▵▵▵▵▵▵▵▵▵▵▵▵▵▵▵▵▵▵▵▵▵▵▵▵▵▵▵▵▵▵▵▵▵▵

% Fin de la sous-section [Norme]
% ------------------------------------------------------------------------------


\subsection{Compréhension du contexte}

\begin{frame}
\begin{block}{\vspace{1cm}\begin{center}\textbf{Compréhension du contexte} \end{center}\vspace{1cm}}
\end{block}
\end{frame}

\subsubsection{Contexte métier}
\begin{frame}
  \frametitle{Contexte métier}
  \begin{figure}[htbp]
	\centering
	\begin{tikzpicture}
		% définition des styles
		\tikzstyle{metier}=[rectangle,draw,fill=yellow!50,text=black]
		\tikzstyle{support}=[rectangle,draw,fill=blue!50,text=black]
		\tikzstyle{supporte}=[->,>=latex,thick,rounded corners=4pt]
		% les nœuds
		\node[metier] (e) at (-3,-3) {Eau et sanitaire};
		\node[metier] (m) at (0,0.5) {Médecine};
		\node[metier] (d) at (3,-3) {Distribution};
		\node[support] (l) at (-2,-1.5) {Logistique};
		\node[support] (t) at (2,-1.5) {Télécoms};
		% les flèches
		\draw[supporte] (l) to[bend left] (t);
		\draw[supporte] (l) to[bend right] (e);
		\draw[supporte] (l) to[bend left] (m);
		\draw[supporte] (l) to[bend right] (d);
		\draw[supporte] (t) to[bend left] (l);
		\draw[supporte] (t) to[bend left] (e);
		\draw[supporte] (t) to[bend right] (m);
		\draw[supporte] (t) to[bend left] (d);
		% la légende
		\draw[supporte] (4,0) -- (7,0) node[midway,above]{Supporte};
	\end{tikzpicture}
	\caption{Dépendances entre les métiers}
	\label{dep}
\end{figure}
\end{frame}

\begin{frame}
\frametitle{Contexte métier}
\begin{block}
Les processus et les fonctions clefs de la logistique
\end{block}
\begin{block}
Les logisticiens de terrain
\end{block}
\begin{block}
Les transports
\end{block}
\end{frame} 





\subsection{Expression fonctionnelle des besoins}
\begin{frame}
\begin{block}{\vspace{1cm}\begin{center}\textbf{Expression fonctionnelle des besoins} \end{center}\vspace{1cm}}
\end{block}
\end{frame}
\subsubsection{Fonction de service et de contrainte}
\begin{frame}
  \frametitle{Fonction de service et de contrainte}
  \begin{block}{}
  \begin{itemize}
  \item La localisation \pause
  \item L'interface utilisateur \pause
  \item Les cas d'utilisations
  \end{itemize}
  \end{block}
\end{frame}

\begin{frame}
\frametitle{Les cas d'utilisations} \pause
\begin{block}{Liés à l'administrateur}
Gestion des niveaux de sécurité, des droits d'accès et des sauvegardes
\end{block}
\pause
\begin{block}{Liés à l'utilisateur}
\begin{enumerate}
\item Gestion des préférences,de la synchronisation \pause
\item Gestion des prestataires, des moyens de transport et des chauffeurs \pause
\item Gestion des réquisitions, des waybills/delivery notes et des tableaux de bord
\end{enumerate}
\end{block}
\end{frame}

\subsubsection{Les critères d'appréciation}
\begin{frame}
  \frametitle{Les critères d'appréciation}
  \begin{block}{Les plus demandés}
  \begin{itemize}
  \item Disponibilité \pause
  \item Capacité \pause
  \item Sécurité \pause
  \item Continuité 
  \end{itemize}
  \end{block}
  \end{frame}

\subsection{Cadre de réponse}
\begin{frame}
\begin{block}{\vspace{1cm}\begin{center}\textbf{Cadre de réponse} \end{center}\vspace{1cm}}
\end{block}
\end{frame}

\section[Mission de maîtrise d'œuvre~: réalisation du projet]{Mission de maîtrise d'œuvre~: réalisation du projet}

% ------------------------------------------------------------------------------
%-------------------------------------------------------------------------------
%
%     CHARGEMENT DES EXTENSIONS
%
%-------------------------------------------------------------------------------

\documentclass[11pt,fleqn]{report}

\usepackage{garmirkhatch}



%-------------------------------------------------------------------------------
%
%     GLOBAL VALUES
%
%-------------------------------------------------------------------------------


%-------------------------------------------------------------------------------
%     Informations spécifiques au document
%-------------------------------------------------------------------------------

\ZTitle{Système de gestion des transports}
\ZSubject{Cahier des charges}
\ZVersion{2.0}
\ZDate{2014-03-03}
\ZAuthor{\Balde,\\\Cadon,\\\Gairoard,\\\Julien,\\\Lericolais,\\\Mezelle,\\\Pachy,\\\SuangaWeto,\\\Toure}


%-------------------------------------------------------------------------------
%     Contenu
%-------------------------------------------------------------------------------

\begin{document}

\ZMakeCover

\ZMakeInformations{
	% Historique des modifications
	% Version & Date & Auteur(s) & Modification(s)
	0.0 & 2014-03-03 & \Cadon & Création \\
}{
	% Historique des approbations
	% Version & Date & Approbateur(s)
	0.0 & 2014-03-03 & \Cadon \\
}{
	% Historique des validations
	% Version & Date & Responsable(s)
	0.0 & 2014-03-03 & \Cadon \\
}

\ZMakeTableOfContents

\chapter{Introduction}
Ce document présente le Plan d'Assurance Qualité (PAQ) relatif au projet de Transport Management System (TMS) par l'assistance à maîtrise d'ouvrage \amo au bénéfice de \mo.
\\
Il définit donc les méthodes, l'organisation et les éléments permettant d'assurer et de contrôler la qualité du projet.

\chapter{Structure du projet}
L'ensemble des documents produits durant le projet sera composé~:
\begin{enumerate}
	\item du PAQ~;
	\item du cahier des charges~;
	\item des comptes rendus de réunions~;
	\item des listes des tâches avec les assignations.
\end{enumerate}

\chapter{Organisation et suivi du projet}

\section{Acteurs}
\begin{table}[htbp]
	\begin{tabularx}{\linewidth}{X X}
		\toprule
		\textbf{Prénom(s) \& Nom} & \textbf{Adresse e-mail} \\
		\midrule
		\Agopian & \AgopianEmail \\
		\hline
		\Balde & \BaldeEmail \\
		\hline
		\Cadon & \CadonEmail \\
		\hline
		\Gairoard & \GairoardEmail \\
		\hline
		\Julien & \JulienEmail \\
		\hline
		\Lericolais & \LericolaisEmail \\
		\hline
		\Mezelle & \MezelleEmail \\
		\hline
		\Pachy & \PachyEmail \\
		\hline
		\SuangaWeto & \SuangaWetoEmail \\
		\hline
		\Toure & \ToureEmail \\
		\bottomrule
	\end{tabularx}
	\caption{Liste des acteurs}
	\label{Acteurs}
\end{table}

\section{Méthodologie de gestion du projet}
La méthodologie de gestion du projet de réalisation du TMS suit un cycle de développement en V, guidé par les tests et dont les principes sont~:
\begin{enumerate}
	\item tous les acteurs sont joignables par e-mail~;
	\item des réunions hebdomadaires externes sont planifiées jusqu'à la livraison finale tous les vendredis matins sur le campus universitaire de Saint-Charles à Marseille~;
	\item les présentations de l'avancement de l'application sont fréquentes et se déroulent pendant les réunions avec le représentant de la société.
\end{enumerate}
Afin de favoriser le partage des connaissances entre chaque membre du groupe, les différentes tâches seront effectuées principalement en binôme. La composition de chaque binôme variera en fonction des tâches à réaliser et des savoir-faire de chacun.
\\
Une fois qu'une tâche est terminée, si celle-ci a une suite logique, il incombe aux prédécesseurs de prévenir les successeurs.

\section{Réunions}

\subsection{Réunions internes}
Des réunions internes bihebdomadaires sont organisées afin de garantir une bonne coordination de l'équipe. Durant ces réunions tous les membres sont présents.

\subsection{Réunions externes}
Des réunions externes hebdomadaires sont organisées avec le représentant de la société afin de garantir une réactivité optimale concernant ses attentes et ses besoins.
\\
Les réunions externes donnent lieu à la rédaction d'un compte rendu contenant la liste exhaustive des tâches à réaliser, et remis au client au plus tard 48h après la fin de la réunion (dans les jours ouvrables).

\section{Outils de travail}
Afin d'éviter des problèmes d'incompatibilité, les outils suivants seront utilisés~:
\begin{table}[htbp]
	\centering
	\begin{tabularx}{\linewidth}{m{30mm} m{30mm} X}
		\toprule
		\textbf{Nom} & \textbf{Version} & \textbf{URL} \\
		\midrule
		GNU/Linux & Ubuntu 12.04 LTS & www.ubuntu.fr \\
		GNU/Linux & Debian & www.debian.org \\
		Microsoft~Windows & 7 & www.microsoft.com \\
		\bottomrule
	\end{tabularx}
	\caption{Outils de travail~: systèmes d'exploitation}
	\label{OutilsOS}
\end{table}
\begin{table}[htbp]
	\centering
	\begin{tabularx}{\linewidth}{m{30mm} m{30mm} X}
		\toprule
		\textbf{Nom} & \textbf{Version} & \textbf{URL} \\
		\midrule
		LaTeX & - & - \\
		\bottomrule
	\end{tabularx}
	\caption{Outils de travail~: documentation}
	\label{OutilsDocumentation}
\end{table}
\begin{table}[htbp]
	\centering
	\begin{tabularx}{\linewidth}{m{30mm} m{30mm} X}
		\toprule
		\textbf{Nom} & \textbf{Version} & \textbf{URL} \\
		\midrule
		Github & 1.7.9.5 & https://github.com/raviluminy/gk \\
		\bottomrule
	\end{tabularx}
	\caption{Outils de travail~: partage des fichiers}
	\label{OutilsDepot}
\end{table}

\chapter{Abréviations}
\begin{table}[htbp]
	\centering
	\begin{tabularx}{\linewidth}{m{30mm} X}
		\toprule
		\textbf{Abréviation} & \textbf{Signification} \\
		\midrule
		AMO & Assistance à Maîtrise d'Ouvrage \\
		CdC & Cahier des Charges \\
		CdCF & Cahier des Charges Fonctionnel \\
		CdCT & Cahier des Charges Technique \\
		CR & Compte-rendu \\
		Go & Giga-octet \\
		LdT & Liste des Tâches \\
		OdJ & Ordre du Jour \\
		PAQ & Plan d'Assurance Qualité \\
		SGBD & Système de Gestion de Bases de Données \\
		SGBDR & Système de Gestion de Bases de Données Relationnelles \\
		DAT & Dossier d'Architecture Technique \\
		PTI & Protocole Technique d'Installation \\
		PdR & Procédure de Recette \\
		DE & Dossier d'Exploitation \\
		PCS & Plan de Continuité des Services \\
		\bottomrule
	\end{tabularx}
	\caption{Abréviations communes}
	\label{AbreviationsCommunes}
\end{table}

\chapter{Gestion de la documentation}
Cette section décrit la manière dont sont gérés les documents relatifs au projet.

\section{Structure des documents}
Tout document produit doit se plier au modèle de document de la société.

\section{Nomenclature}
La liste exhaustive des documents concernés est décrite ci-dessous (hors sous-documents, images et librairies)~:
\begin{enumerate}
	\item \textbf{/Annexe/Docs/}
	\\
	Contient les documents relatifs à \mo, dont~:
	\begin{enumerate}
		\item GARMIR KHATCH - EXEMPLE DE CONTRAT DE TRANSPORT.doc,
		\item GARMIR KHATCH - SUIVI DES TRANSPORTS (EXEMPLE).xlsx,
		\item LOGISTICS REQUISITION.pdf,
		\item WAYBILL DELIVERY NOTE.pdf~;
	\end{enumerate}
	\item \textbf{/Annexe/RE/}
	\\
	Contient les documents relatifs aux réunions externes~:
	\begin{enumerate}
		\item RE-XXX-(YYYY-MM-DD)-CR-\textit{version}.tex,
		\item RE-XXX-(YYYY-MM-DD)-LdT-\textit{version}.tex,
		\item RE-XXX-(YYYY-MM-DD)-OdJ-\textit{version}.tex~;
	\end{enumerate}
	\item \textbf{/Annexe/RI/}
	\\
	Contient les documents relatifs aux réunions internes~:
	\begin{enumerate}
		\item RI-XXX-(YYYY-MM-DD)-CR-\textit{version}.tex,
		\item RI-XXX-(YYYY-MM-DD)-LdT-\textit{version}.tex,
		\item RI-XXX-(YYYY-MM-DD)-OdJ-\textit{version}.tex~;
	\end{enumerate}
	\item \textbf{/Conception/}
	\\
	Contient les documents relatifs à la conception~:
	\begin{enumerate}
		\item CdC-\textit{version}.tex,
		\item CdCF-\textit{version}.tex,
		\item CdCT-\textit{version}.tex,
		\item PAQ-\textit{version}.tex~;
	\end{enumerate}
	\item \textbf{/Realisation/}
	\\
	Contient les documents relatifs à la réalisation~:
	\begin{enumerate}
		\item DAT-\textit{version}.tex,
		\item PTI-\textit{version}.tex,
		\item PdR-\textit{version}.tex,
		\item DE-\textit{version}.tex,
		\item PCS-\textit{version}.tex.
	\end{enumerate}
\end{enumerate}
Dans chaque document énoncé, l'élément \textit{version} représente la version du fichier dans un format pointé. Ci-dessous quelques exemples~:
\begin{enumerate}
	\item /Annexe/RE/RE-001-(2014-02-07)-CR-2.0.tex~;
	\item /Annexe/RI/RI-003-(2014-02-21)-OdJ-1.3.tex~;
	\item /Conception/CdCF-2.1.tex~;
	\item /Realisation/DAT-0.1.tex.
\end{enumerate}

\chapter{Planification}
\begin{figure}
	\centering
	\begin{ganttchart}[
			y unit title=0.4cm,
			y unit chart=0.5cm,
			vgrid,
			hgrid,
			title label anchor/.style={below=-1.6ex},
			title left shift=.05,
			title right shift=-.05,
			title height=1,
			bar/.style={draw=black,fill=gray!50},
			incomplete/.style={fill=white},
			progress label text={},
			bar height=0.7,
			group right shift=0,
			group top shift=.6,
			group height=.3,
			group peaks height =.2
		]{1}{28}
		% Titres
		\gantttitle{Itération X}{28} \\
		\gantttitle{Vendredi}{4}
		\gantttitle{Samedi}{4}
		\gantttitle{Dimanche}{4}
		\gantttitle{Lundi}{4}
		\gantttitle{Mardi}{4}
		\gantttitle{Mercredi}{4}
		\gantttitle{Jeudi}{4} \\
		% Tâches
		\ganttgroup{Groupe 1}{1}{8} \\
		\ganttbar{Tâche 1}{1}{2} \\
		\ganttbar{Tâche 2}{3}{8} \\
		\ganttbar{Tâche 3}{9}{10} \\
		\ganttgroup{Groupe 2}{1}{12} \\
		\ganttbar{Tâche 4}{11}{15} \\
		\ganttbar{Tâche 5}{20}{22} \\%[progress=33]
		\ganttgroup{Groupe 3}{1}{12} \\
		\ganttgroup{Groupe 4}{6}{12} \\
		\ganttbar{Tâche 6}{18}{19} \\
		\ganttbar{Tâche 7}{16}{18} \\
		\ganttbar{Tâche 8}{21}{24}%[progress=0]
		% Relations
		\ganttlink{elem0}{elem1}
		\ganttlink{elem0}{elem3}
		\ganttlink{elem1}{elem2}
		\ganttlink{elem3}{elem4}
		\ganttlink{elem1}{elem5}
		\ganttlink{elem3}{elem5}
		\ganttlink{elem2}{elem6}
		\ganttlink{elem3}{elem6}
		\ganttlink{elem5}{elem7}
		\node[fill=white,draw] at ([yshift=-12pt]current bounding box.south){
			\Balde, \Cadon...
		};
%		\node[fill=white,draw,anchor=west] at (current bounding box.north west){Box North West};
%		\node[fill=white,draw,anchor=west] at (current bounding box.north east){Box North East};
	\end{ganttchart}
	\caption{Planification par Gantt}
	\label{PlanificationGantt}
\end{figure}

\end{document}

% ------------------------------------------------------------------------------
\subsection[Réponse technique aux besoins fonctionnels (DAT)]{Réponse technique aux besoins fonctionnels (DAT)}

\subsection[Authentification et contrôle d'accès (LDAP)]{Authentification et contrôle d'accès (LDAP)}
\begin{frame}
\tableofcontents[subsectionstyle=show/shaded/hide, subsubsectionstyle=hide, sectionstyle=show/hide]
\end{frame}
\begin{frame}
  \frametitle{Authentification et contrôle d'accès}
  \begin{block}{\textbf{ Présentation }}
  \begin{itemize}
  \item 
  \item 
  \end{itemize}
  \end{block}
\end{frame}

\begin{frame}
  \frametitle{Authentification et contrôle d'accès}
  \begin{block}{\textbf{Pourquoi un annuaire ? }}
  \begin{itemize}
  \item Plus consulté que mis à jour
  \item Structure hierachique 
  \end{itemize} 
  \end{block}
\end{frame}

\begin{frame}
  \frametitle{Authentification et contrôle d'accès}
  \begin{block}{\textbf{Pourquoi LDAP ? }}
  \begin{itemize}
  \item Protocole réseau léger
  \item Recherche simple et critérisée
  \item Organisation des résultats
  \item Mécanisme de référencement
  \item Authentification et contrôle d'accès
  \item Haute disponibilité via réplication
  \end{itemize}
  \end{block}
\end{frame}

  \frametitle{Authentification et contrôle d'accès}
  \begin{figure}[htbp]
	  \centering
	  \includegraphics[scale=0.6]{Images/SchemaLDAP.png}
	  \caption{Structure de l'annuaire}
	  \label{SchemaLDAP}
  \end{figure}
 
  \begin{figure}[htbp]
	\centering
	\includegraphics[scale=0.4]{Images/SchemaGlobal.png}
	\caption{Architecture globale}
	\label{SchemaGlobal}
\end{figure}

\begin{figure}[htbp]
	\centering
	\includegraphics[scale=0.6]{Images/SchemaAuthentification.png}
	\caption{Authentification pour la synchronisation}
	\label{SchemaAuthentification}
\end{figure}

\begin{frame}
  \frametitle{Authentification et contrôle d'accès}
  \begin{block}{\textbf{Droits utilisateurs}}
  \begin{itemize}
  \item Droits pour chaque utilisateur
  \item Droits pour chaque groupe
  \item Autorisations et/ou refus
  \end{itemize}
  \end{block}
\end{frame}

\begin{frame}
  \frametitle{Authentification et contrôle d'accès}
  \begin{block}{\textbf{Contrôle d'accès}}
  \begin{itemize}
  \item Gestion des conflits 
  \item 
  \end{itemize}
  \end{block}
\end{frame}

% ------------------------------------------------------------------------------
\subsection{Synchronisation}

% Fin de la sous-section [Synchronisation]
% ------------------------------------------------------------------------------


\subsection[Stockage des données]{Stockage des données}
\begin{frame}
\tableofcontents[subsectionstyle=show/shaded/hide, subsubsectionstyle=hide, sectionstyle=show/hide]
\end{frame}

\begin{frame}
\begin{block}{\vspace{1cm}\begin{center}\textbf{Stockage des données}\end{center}\vspace{1cm}}
\end{block}
\end{frame}

\subsection[Sauvegarde]{Sauvegarde}
\begin{frame}
\tableofcontents[subsectionstyle=show/shaded/hide, subsubsectionstyle=hide, sectionstyle=show/hide]
\end{frame}
\include{Sauvegarde}

\subsection[Tests sur l'interface graphique et unitaires (CdR)]{Tests sur l'interface graphique et unitaires (CdR)}
\begin{frame}
\tableofcontents[subsectionstyle=show/shaded/hide, subsubsectionstyle=hide, sectionstyle=show/hide]
\end{frame}
\begin{frame}
\frametitle{}
\begin{block}{\textbf{Pourquoi les tests sont importants dans ce projet}}
\begin{itemize}
\item Organisation humanitaires (187 Sociétés Nationales)
\item Vient en aide aux personnes plus vulnérables
\item Fort réseau de volontaires
\item Rend les communautés plus résistantes
\end{itemize}
\end{block}


\end{frame}


\subsection[Procédure de déploiement et d'installation (PIT)]{Procédure de déploiement et d'installation (PIT)}
\begin{frame}
\tableofcontents[subsectionstyle=show/shaded/hide, subsubsectionstyle=hide, sectionstyle=show/hide]
\end{frame}
\begin{frame}
\end{frame}

\section*[Conclusion]{Conclusion}

\subsection[Conclusion]{Conclusion}
\begin{frame}
	\transdissolve[duration=0.2]<1->
	\frametitle{Conclusion}
\end{frame}

\subsection[Questions~?]{Questions~?}
\begin{frame}
	\transdissolve[duration=0.2]<1->
	\frametitle{Merci de votre attention}
\end{frame}

\end{document}
