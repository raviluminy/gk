% ------------------------------------------------------------------------------
\subsection{Norme NF X50-151}

% ▿▿▿▿▿▿▿▿▿▿▿▿▿▿▿▿▿▿▿▿▿▿▿▿▿▿▿▿▿▿▿▿▿▿▿▿▿▿▿▿▿▿▿▿▿▿▿▿▿▿▿▿▿▿▿▿▿▿▿▿▿▿▿▿▿▿▿▿▿▿▿▿▿▿▿▿▿▿
\begin{frame}
\tableofcontents[subsectionstyle=show/shaded/hide, subsubsectionstyle=hide, sectionstyle=show/hide]
\end{frame}
% ▵▵▵▵▵▵▵▵▵▵▵▵▵▵▵▵▵▵▵▵▵▵▵▵▵▵▵▵▵▵▵▵▵▵▵▵▵▵▵▵▵▵▵▵▵▵▵▵▵▵▵▵▵▵▵▵▵▵▵▵▵▵▵▵▵▵▵▵▵▵▵▵▵▵▵▵▵▵

% ▿▿▿▿▿▿▿▿▿▿▿▿▿▿▿▿▿▿▿▿▿▿▿▿▿▿▿▿▿▿▿▿▿▿▿▿▿▿▿▿▿▿▿▿▿▿▿▿▿▿▿▿▿▿▿▿▿▿▿▿▿▿▿▿▿▿▿▿▿▿▿▿▿▿▿▿▿▿
\begin{frame}
\frametitle{Contenu (non exhaustif) de la norme NF X50-151}

\begin{block}{Pourquoi une norme ?}
\begin{itemize}
    \item Nous aider % première fois, difficile de savoir ce qu'on met dans un cdcf, on a eu des redondance à cause différence fonctionnel/technique
    \item Être \og{}standard\fg{} % Pour permetre une réponse adaptée, produire un document sans manque d'infos
\end{itemize}
\end{block}
\end{frame}

\begin{frame}
\frametitle{Contenu (non exhaustif) de la norme NF X50-151}
\begin{exampleblock}{1/3 - Présentation du projet}
\begin{itemize}
    \item Le projet en lui même \small(finalités, retours sur investissements)
    \item Contexte \small(études réalisées, suites, nature des prestations, confidentialité)
    \item Énoncé du besoin \small(du point de vue de l'utillisateur)
    \item Environnement \small(utilisateurs, caractéristiques et contraintes matérielles)
\end{itemize}
\end{exampleblock}
\end{frame}

\begin{frame}
\frametitle{Contenu (non exhaustif) de la norme NF X50-151}
\begin{exampleblock}{2/3 - Expression fonctionnelle du besoin}
\begin{itemize}
    \item Fonctions principales (métier)
    \item Fonction complémentaires
    \item Critères d'appréciation de la solution fournie
\end{itemize}
\end{exampleblock}

\end{frame}
% ▵▵▵▵▵▵▵▵▵▵▵▵▵▵▵▵▵▵▵▵▵▵▵▵▵▵▵▵▵▵▵▵▵▵▵▵▵▵▵▵▵▵▵▵▵▵▵▵▵▵▵▵▵▵▵▵▵▵▵▵▵▵▵▵▵▵▵▵▵▵▵▵▵▵▵▵▵▵

% ▿▿▿▿▿▿▿▿▿▿▿▿▿▿▿▿▿▿▿▿▿▿▿▿▿▿▿▿▿▿▿▿▿▿▿▿▿▿▿▿▿▿▿▿▿▿▿▿▿▿▿▿▿▿▿▿▿▿▿▿▿▿▿▿▿▿▿▿▿▿▿▿▿▿▿▿▿▿
\begin{frame}
\frametitle{Contenu (non exhaustif) de la norme NF X50-151}
%\frametitle{Contenu (suite)}

\begin{exampleblock}{3/3 - Cadre de réponse}
Pour chaque fonction :
\begin{itemize}
    \item Description de la solution proposée
    \item Mise en évidence des critères d'appréciation
    \item Coût
\end{itemize}
Pour l'ensemble :
\begin{itemize}
    \item Écarts par rapport au CdCF
    \item Mesures prises pour respecter les contraintes
    \item Documentation technique \& utilisateur
    \item Modularité
    \item Fiabilité
    \item Évolutions technologiques
\end{itemize}
\end{exampleblock}

\end{frame} % Fin de la frame [Contenu (suite)]
% ▵▵▵▵▵▵▵▵▵▵▵▵▵▵▵▵▵▵▵▵▵▵▵▵▵▵▵▵▵▵▵▵▵▵▵▵▵▵▵▵▵▵▵▵▵▵▵▵▵▵▵▵▵▵▵▵▵▵▵▵▵▵▵▵▵▵▵▵▵▵▵▵▵▵▵▵▵▵

% Fin de la sous-section [Norme]
% ------------------------------------------------------------------------------
