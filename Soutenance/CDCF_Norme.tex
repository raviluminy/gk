% ------------------------------------------------------------------------------
\subsubsection{Norme NF X50-151}

%% ▿▿▿▿▿▿▿▿▿▿▿▿▿▿▿▿▿▿▿▿▿▿▿▿▿▿▿▿▿▿▿▿▿▿▿▿▿▿▿▿▿▿▿▿▿▿▿▿▿▿▿▿▿▿▿▿▿▿▿▿▿▿▿▿▿▿▿▿▿▿▿▿▿▿▿▿▿▿
%\begin{frame}
%\tableofcontents[subsectionstyle=show/shaded/hide, subsubsectionstyle=hide, sectionstyle=show/hide]
%\end{frame}
%% ▵▵▵▵▵▵▵▵▵▵▵▵▵▵▵▵▵▵▵▵▵▵▵▵▵▵▵▵▵▵▵▵▵▵▵▵▵▵▵▵▵▵▵▵▵▵▵▵▵▵▵▵▵▵▵▵▵▵▵▵▵▵▵▵▵▵▵▵▵▵▵▵▵▵▵▵▵▵

% ▿▿▿▿▿▿▿▿▿▿▿▿▿▿▿▿▿▿▿▿▿▿▿▿▿▿▿▿▿▿▿▿▿▿▿▿▿▿▿▿▿▿▿▿▿▿▿▿▿▿▿▿▿▿▿▿▿▿▿▿▿▿▿▿▿▿▿▿▿▿▿▿▿▿▿▿▿▿
\begin{frame}
\frametitle{Idée générale}

\begin{block}{Pourquoi une norme ?}
\begin{itemize}
    \item Nous aider % première fois, difficile de savoir ce qu'on met dans un cdcf, on a eu des redondance à cause différence fonctionnel/technique
    \item Être \og{}standard\fg{} % Pour permetre une réponse adaptée, produire un document sans manque d'infos
\end{itemize}
\end{block}
\end{frame}
% ▵▵▵▵▵▵▵▵▵▵▵▵▵▵▵▵▵▵▵▵▵▵▵▵▵▵▵▵▵▵▵▵▵▵▵▵▵▵▵▵▵▵▵▵▵▵▵▵▵▵▵▵▵▵▵▵▵▵▵▵▵▵▵▵▵▵▵▵▵▵▵▵▵▵▵▵▵▵

% ▿▿▿▿▿▿▿▿▿▿▿▿▿▿▿▿▿▿▿▿▿▿▿▿▿▿▿▿▿▿▿▿▿▿▿▿▿▿▿▿▿▿▿▿▿▿▿▿▿▿▿▿▿▿▿▿▿▿▿▿▿▿▿▿▿▿▿▿▿▿▿▿▿▿▿▿▿▿
\begin{frame}
\frametitle{Contenu (non exhaustif) de la norme NF X50-151}
\begin{block}{1/3 - Présentation du projet}
\begin{itemize}
    \item Le projet en lui même % (finalités, retours sur investissements)
    \item Contexte % (études réalisées, suites, nature des prestations, confidentialité)
    \item Énoncé du besoin % (du point de vue de l'utillisateur)
    \item Environnement % (utilisateurs, caractéristiques et contraintes matérielles)
\end{itemize}
\end{block}
\end{frame}
% ▵▵▵▵▵▵▵▵▵▵▵▵▵▵▵▵▵▵▵▵▵▵▵▵▵▵▵▵▵▵▵▵▵▵▵▵▵▵▵▵▵▵▵▵▵▵▵▵▵▵▵▵▵▵▵▵▵▵▵▵▵▵▵▵▵▵▵▵▵▵▵▵▵▵▵▵▵▵

% ▿▿▿▿▿▿▿▿▿▿▿▿▿▿▿▿▿▿▿▿▿▿▿▿▿▿▿▿▿▿▿▿▿▿▿▿▿▿▿▿▿▿▿▿▿▿▿▿▿▿▿▿▿▿▿▿▿▿▿▿▿▿▿▿▿▿▿▿▿▿▿▿▿▿▿▿▿▿
\begin{frame}
\frametitle{Contenu (non exhaustif) de la norme NF X50-151}
\begin{block}{2/3 - Expression fonctionnelle du besoin}
\begin{itemize}
    \item Fonctions principales % (métier)
    \item Fonction complémentaires % Ce qui serait sympa mais pas obligatiore
    \item Critères d'appréciation de la solution fournie % Comment sera jugé chaque point
\end{itemize}
\end{block}

\end{frame}
% ▵▵▵▵▵▵▵▵▵▵▵▵▵▵▵▵▵▵▵▵▵▵▵▵▵▵▵▵▵▵▵▵▵▵▵▵▵▵▵▵▵▵▵▵▵▵▵▵▵▵▵▵▵▵▵▵▵▵▵▵▵▵▵▵▵▵▵▵▵▵▵▵▵▵▵▵▵▵

% ▿▿▿▿▿▿▿▿▿▿▿▿▿▿▿▿▿▿▿▿▿▿▿▿▿▿▿▿▿▿▿▿▿▿▿▿▿▿▿▿▿▿▿▿▿▿▿▿▿▿▿▿▿▿▿▿▿▿▿▿▿▿▿▿▿▿▿▿▿▿▿▿▿▿▿▿▿▿
\begin{frame}
\frametitle{Contenu (non exhaustif) de la norme NF X50-151}
%\frametitle{Contenu (suite)}

\begin{block}{3/3 - Cadre de réponse}
Pour chaque fonction :
\begin{itemize}
    \item Description de la solution proposée % Expliquer comment ça marche
    \item Mise en évidence des critères d'appréciation % Par rapport à ce qui était demandé dans le CDCF
    \item Coût % Parce que le client paie
\end{itemize}
Pour l'ensemble :
\begin{itemize}
    \item Écarts par rapport au CdCF % Voir ce qui a pas été fait, ou fait en plus
    \item Mesures prises pour respecter les contraintes % Comme le PAQ
    \item Documentation technique \& utilisateur % 
    \item Modularité, fiabilité, évolutions technologiques % Une façon de demander "est ce que la solution a de l'avenir ?"
\end{itemize}
\end{block}

\end{frame} % Fin de la frame [Contenu (suite)]
% ▵▵▵▵▵▵▵▵▵▵▵▵▵▵▵▵▵▵▵▵▵▵▵▵▵▵▵▵▵▵▵▵▵▵▵▵▵▵▵▵▵▵▵▵▵▵▵▵▵▵▵▵▵▵▵▵▵▵▵▵▵▵▵▵▵▵▵▵▵▵▵▵▵▵▵▵▵▵

% Fin de la sous-section [Norme]
% ------------------------------------------------------------------------------
