% ------------------------------------------------------------------------------
\subsection{Plan d'Assurance Qualité}
% ▿▿▿▿▿▿▿▿▿▿▿▿▿▿▿▿▿▿▿▿▿▿▿▿▿▿▿▿▿▿▿▿▿▿▿▿▿▿▿▿▿▿▿▿▿▿▿▿▿▿▿▿▿▿▿▿▿▿▿▿▿▿▿▿▿▿▿▿▿▿▿▿▿▿▿▿▿▿
\begin{frame}
\tableofcontents[subsectionstyle=show/shaded/hide, sectionstyle=show/hide]
\end{frame}
% ▵▵▵▵▵▵▵▵▵▵▵▵▵▵▵▵▵▵▵▵▵▵▵▵▵▵▵▵▵▵▵▵▵▵▵▵▵▵▵▵▵▵▵▵▵▵▵▵▵▵▵▵▵▵▵▵▵▵▵▵▵▵▵▵▵▵▵▵▵▵▵▵▵▵▵▵▵▵

% ▿▿▿▿▿▿▿▿▿▿▿▿▿▿▿▿▿▿▿▿▿▿▿▿▿▿▿▿▿▿▿▿▿▿▿▿▿▿▿▿▿▿▿▿▿▿▿▿▿▿▿▿▿▿▿▿▿▿▿▿▿▿▿▿▿▿▿▿▿▿▿▿▿▿▿▿▿▿
\begin{frame}
\frametitle{Objectifs}
\begin{block}{Documenter}
\begin{itemize}
    \item Méthodologie % Comprendre la méthodologgie => comprendre la conception du projet
    \item \emph{Mise en bouche} avant de rentrer dans le projet % Abréviations, 
    \item Garanties pour le client, support de communication, ...
\end{itemize}
\end{block}
\begin{block}{Une référence}
\begin{itemize}
    \item Pour le groupe % Pour savoir comment s'organiser
    \item Reprise du projet % Outils notemment
\end{itemize}
\end{block}
\end{frame} % Fin de la frame [Objectifs]
% ▵▵▵▵▵▵▵▵▵▵▵▵▵▵▵▵▵▵▵▵▵▵▵▵▵▵▵▵▵▵▵▵▵▵▵▵▵▵▵▵▵▵▵▵▵▵▵▵▵▵▵▵▵▵▵▵▵▵▵▵▵▵▵▵▵▵▵▵▵▵▵▵▵▵▵▵▵▵

% ▿▿▿▿▿▿▿▿▿▿▿▿▿▿▿▿▿▿▿▿▿▿▿▿▿▿▿▿▿▿▿▿▿▿▿▿▿▿▿▿▿▿▿▿▿▿▿▿▿▿▿▿▿▿▿▿▿▿▿▿▿▿▿▿▿▿▿▿▿▿▿▿▿▿▿▿▿▿
\begin{frame}
\frametitle{Contenu}
\begin{itemize}
    \item Documents produits
    \item Membres de l'équipe
    \item \emph{Organisation interne}
    \item Réunions
    \item \emph{Outils de travail}
    \item Abréviations
    \item Arborescence et nomenclature des fichiers
    \item Modèle pour la plannification
    \item Chaine de responsabilité
\end{itemize}
\end{frame} % Fin de la frame [Contenu]
% ▵▵▵▵▵▵▵▵▵▵▵▵▵▵▵▵▵▵▵▵▵▵▵▵▵▵▵▵▵▵▵▵▵▵▵▵▵▵▵▵▵▵▵▵▵▵▵▵▵▵▵▵▵▵▵▵▵▵▵▵▵▵▵▵▵▵▵▵▵▵▵▵▵▵▵▵▵▵

% ▿▿▿▿▿▿▿▿▿▿▿▿▿▿▿▿▿▿▿▿▿▿▿▿▿▿▿▿▿▿▿▿▿▿▿▿▿▿▿▿▿▿▿▿▿▿▿▿▿▿▿▿▿▿▿▿▿▿▿▿▿▿▿▿▿▿▿▿▿▿▿▿▿▿▿▿▿▿
\begin{frame}
\frametitle{Organisation interne}

\begin{block}{Binômes}
\begin{itemize}
    \item Partage des connaissances
    \item Fusion des deux groupes
\end{itemize}
\end{block}

\begin{block}{Communication \& répartition des tâches}
\begin{itemize}
    \item Réunions internes % 3 fois par semaine
    \begin{itemize}
        \item Bilan % du travail réalisé et des difficultés
        \item Organisation % des jours qui viennent (tâches, répartition)
    \end{itemize}
    \item Chaînage des tâches % On prévient quand on a fini son taf pour ceux qui suivent
\end{itemize}
\end{block}

\end{frame} % Fin de la frame [Organisation interne]
% ▵▵▵▵▵▵▵▵▵▵▵▵▵▵▵▵▵▵▵▵▵▵▵▵▵▵▵▵▵▵▵▵▵▵▵▵▵▵▵▵▵▵▵▵▵▵▵▵▵▵▵▵▵▵▵▵▵▵▵▵▵▵▵▵▵▵▵▵▵▵▵▵▵▵▵▵▵▵

% ▿▿▿▿▿▿▿▿▿▿▿▿▿▿▿▿▿▿▿▿▿▿▿▿▿▿▿▿▿▿▿▿▿▿▿▿▿▿▿▿▿▿▿▿▿▿▿▿▿▿▿▿▿▿▿▿▿▿▿▿▿▿▿▿▿▿▿▿▿▿▿▿▿▿▿▿▿▿
\begin{frame}
\frametitle{Outils utilisés}

\begin{block}{Les outils libres ...}
\begin{itemize}
    \item GNU/Linux
    \item Github
    \item \LaTeX
    \item Dia
    \item Inkscape
    \item Qt
\end{itemize}
\end{block}

\begin{block}{... et les moins libres}
\begin{itemize}
    \item Windows % mais on l'a pas beaucoup utilisé
\end{itemize}
\end{block}

\end{frame} % Fin de la frame [Outils utilisés]
% ▵▵▵▵▵▵▵▵▵▵▵▵▵▵▵▵▵▵▵▵▵▵▵▵▵▵▵▵▵▵▵▵▵▵▵▵▵▵▵▵▵▵▵▵▵▵▵▵▵▵▵▵▵▵▵▵▵▵▵▵▵▵▵▵▵▵▵▵▵▵▵▵▵▵▵▵▵▵

% Fin de la sous-section [PLan d'Assurance Qualité]
% ------------------------------------------------------------------------------

