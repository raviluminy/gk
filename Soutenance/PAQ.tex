% ------------------------------------------------------------------------------
\subsection{Plan d'Assurance Qualité}

% ▿▿▿▿▿▿▿▿▿▿▿▿▿▿▿▿▿▿▿▿▿▿▿▿▿▿▿▿▿▿▿▿▿▿▿▿▿▿▿▿▿▿▿▿▿▿▿▿▿▿▿▿▿▿▿▿▿▿▿▿▿▿▿▿▿▿▿▿▿▿▿▿▿▿▿▿▿▿
\begin{frame}
\tableofcontents[subsectionstyle=show/shaded/hide, subsubsectionstyle=hide, sectionstyle=show/hide]
\end{frame}
% ▵▵▵▵▵▵▵▵▵▵▵▵▵▵▵▵▵▵▵▵▵▵▵▵▵▵▵▵▵▵▵▵▵▵▵▵▵▵▵▵▵▵▵▵▵▵▵▵▵▵▵▵▵▵▵▵▵▵▵▵▵▵▵▵▵▵▵▵▵▵▵▵▵▵▵▵▵▵

% ▿▿▿▿▿▿▿▿▿▿▿▿▿▿▿▿▿▿▿▿▿▿▿▿▿▿▿▿▿▿▿▿▿▿▿▿▿▿▿▿▿▿▿▿▿▿▿▿▿▿▿▿▿▿▿▿▿▿▿▿▿▿▿▿▿▿▿▿▿▿▿▿▿▿▿▿▿▿
\begin{frame}
\frametitle{Objectifs}

\begin{block}{Uniformiser}
\begin{itemize}
	\item Outils
	\item Termes utilisés
	\item Gestion des fichiers
\end{itemize}
\end{block} % Fin du block [Uniformiser]

\pause 

\begin{block}{Organiser}
\begin{itemize}
	\item Réunions internes
	\item Partage des tâches
\end{itemize}
\end{block} % Fin du block [Organiser]

\end{frame} % Fin de la frame [Objectifs]
% ▵▵▵▵▵▵▵▵▵▵▵▵▵▵▵▵▵▵▵▵▵▵▵▵▵▵▵▵▵▵▵▵▵▵▵▵▵▵▵▵▵▵▵▵▵▵▵▵▵▵▵▵▵▵▵▵▵▵▵▵▵▵▵▵▵▵▵▵▵▵▵▵▵▵▵▵▵▵

% ▿▿▿▿▿▿▿▿▿▿▿▿▿▿▿▿▿▿▿▿▿▿▿▿▿▿▿▿▿▿▿▿▿▿▿▿▿▿▿▿▿▿▿▿▿▿▿▿▿▿▿▿▿▿▿▿▿▿▿▿▿▿▿▿▿▿▿▿▿▿▿▿▿▿▿▿▿▿
\begin{frame}
\frametitle{Détails}

\begin{block}{Projet}
\begin{itemize}
	\item Binômes % Direct, binômes (partage connaissances, motivation) !
	\item Cycle en spirale % Réu > modifs > Réu
	\item Réunions % Organisation (sommaire) des réunions
\end{itemize}
\end{block} % Fin du block [Projet]

\begin{block}{Documents}
\begin{itemize}
	\item Arborescence % Hiérarchie des dossiers
	\item (Début de) nomenclature % Pour standardiser et s'y retrouver
	\item Entêtes % Ce que doit figurer dans les débuts de documents pour maintenir une cohérence
\end{itemize}
\end{block} % Fin du block [Documents]

\end{frame} % Fin de la frame [Contenu]
% ▵▵▵▵▵▵▵▵▵▵▵▵▵▵▵▵▵▵▵▵▵▵▵▵▵▵▵▵▵▵▵▵▵▵▵▵▵▵▵▵▵▵▵▵▵▵▵▵▵▵▵▵▵▵▵▵▵▵▵▵▵▵▵▵▵▵▵▵▵▵▵▵▵▵▵▵▵▵

% ▿▿▿▿▿▿▿▿▿▿▿▿▿▿▿▿▿▿▿▿▿▿▿▿▿▿▿▿▿▿▿▿▿▿▿▿▿▿▿▿▿▿▿▿▿▿▿▿▿▿▿▿▿▿▿▿▿▿▿▿▿▿▿▿▿▿▿▿▿▿▿▿▿▿▿▿▿▿
\begin{frame}
\frametitle{Détails (suite)}

\begin{block}{Outils} % Ils sont libres, important par l'activité même de Garmir Khatch
\begin{itemize}
    \item LibreOffice % Pour rédac de docs
    \item Github % Pour partage de fichiers
    \item Gantt Project % Pour les diagrammes de gantt
	\item Ubuntu/Debian % Comme support aux outils
\end{itemize}
\end{block} % Fin du block [outils libres]

\end{frame} % Fin de la frame [Contenu]
% ▵▵▵▵▵▵▵▵▵▵▵▵▵▵▵▵▵▵▵▵▵▵▵▵▵▵▵▵▵▵▵▵▵▵▵▵▵▵▵▵▵▵▵▵▵▵▵▵▵▵▵▵▵▵▵▵▵▵▵▵▵▵▵▵▵▵▵▵▵▵▵▵▵▵▵▵▵▵

% ▿▿▿▿▿▿▿▿▿▿▿▿▿▿▿▿▿▿▿▿▿▿▿▿▿▿▿▿▿▿▿▿▿▿▿▿▿▿▿▿▿▿▿▿▿▿▿▿▿▿▿▿▿▿▿▿▿▿▿▿▿▿▿▿▿▿▿▿▿▿▿▿▿▿▿▿▿▿
\begin{frame}
\frametitle{Bilan du PAQ}

\begin{block}{Ce que le PAQ nous a apporté}
\begin{itemize}
	\item Difficultés % À le respecter, problèmes engendrés (git+odt)
	\item Manque de précisions % sur la communication notemment
	\item Une base pour la suite % On a compris son importance
\end{itemize}
\end{block} % Fin du block [Ce que le PAQ nous a apporté]

\end{frame} % Fin de la frame [Bilan du PAQ]
% ▵▵▵▵▵▵▵▵▵▵▵▵▵▵▵▵▵▵▵▵▵▵▵▵▵▵▵▵▵▵▵▵▵▵▵▵▵▵▵▵▵▵▵▵▵▵▵▵▵▵▵▵▵▵▵▵▵▵▵▵▵▵▵▵▵▵▵▵▵▵▵▵▵▵▵▵▵▵

% Fin de la sous-section [PLan d'Assurance Qualité]
% ------------------------------------------------------------------------------

