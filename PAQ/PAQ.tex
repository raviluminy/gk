%-------------------------------------------------------------------------------
%
%     CHARGEMENT DES EXTENSIONS
%
%-------------------------------------------------------------------------------

\documentclass[11pt,fleqn]{report}
\usepackage{GarmirKhatch}

%-------------------------------------------------------------------------------
%     Informations spécifiques au document
%-------------------------------------------------------------------------------

\ZTitle{Système de gestion des transports}
\ZSubject{Plan d'assurance qualité}
\ZVersion{2.2}
\ZDate{2014-03-09}

%-------------------------------------------------------------------------------
%     Contenu
%-------------------------------------------------------------------------------

\begin{document}

\ZMakeCover

\ZMakeInformations{
	% Historique des modifications
	% Version & Date & Auteur(s) & Modification(s)
	2.0 & 2014-03-03 & \Cadon & Création à partir de l'ancienne version \\
	\midrule
	2.1 & 2014-03-04 & \Pachy & Mise à jour en fonction de la réunion interne du 2014-03-03 \\
	\midrule
	2.2 & 2014-03-09 & \Cadon & Mise à jour chaine de responsabilité et planification \\
}{
	% Historique des approbations
	% Version & Date & Approbateur(s)
	2.2 & 2014-03-09 & \Cadon \\
}{
	% Historique des validations
	% Version & Date & Responsable(s)
	2.2 & - & - \\
}

\ZMakeTableOfContents

\chapter{Introduction}
Ce document présente le Plan d'Assurance Qualité (PAQ) relatif au projet de Transport~Management~System (TMS) par l'assistance à maîtrise d'ouvrage \amo au bénéfice de \mo.
\\
Il définit donc les méthodes, l'organisation et les éléments permettant d'assurer et de contrôler la qualité du projet.

\chapter{Organisation et suivi du projet}

\section{Acteurs}
\begin{table}[htbp]
	\begin{tabularx}{\linewidth}{X X X}
		\toprule
		\textbf{Prénom(s) \& Nom} & \textbf{Adresse e-mail} & \textbf{Rôle(s)} \\
		\midrule
		\Agopian & \AgopianEmail & MO \\
		\hline
		\Balde & \BaldeEmail & - \\
		\hline
		\Cadon & \CadonEmail & Co-chef de projet, conseiller, intégrateur, concepteur, développeur, designer \\
		\hline
		\Gairoard & \GairoardEmail & Concepteur, conseiller, secrétaire \\
		\hline
		\Julien & \JulienEmail & Intégrateur, Concepteur, conseiller, développeur \\
		\hline
		\Lericolais & \LericolaisEmail & Concepteur, conseiller, développeur \\
		\hline
		\Mezelle & \MezelleEmail & Concepteur, conseiller, développeur \\
		\hline
		\Pachy & \PachyEmail & Co-chef de projet, concepteur, conseiller, designer, secrétaire \\
		\hline
		\SuangaWeto & \SuangaWetoEmail & Concepteur, conseiller, développeur \\
		\hline
		\Toure & \ToureEmail & Concepteur, conseiller, secrétaire \\
		\bottomrule
	\end{tabularx}
	\caption{Liste des acteurs}
	\label{Acteurs}
\end{table}

\section{Méthodologie de gestion du projet}
La méthodologie de gestion du projet de réalisation du TMS suit un cycle de développement en V, guidé par les tests et dont les principes sont~:
\begin{enumerate}
	\item tous les acteurs sont joignables par e-mail~;
	\item des réunions hebdomadaires externes sont planifiées jusqu'à la livraison finale tous les vendredis matins sur le campus universitaire de Saint-Charles à Marseille~;
	\item les présentations de l'avancement de l'application sont fréquentes et se déroulent pendant les réunions avec le représentant de la société.
\end{enumerate}
Afin de favoriser le partage des connaissances entre chaque membre du groupe, les différentes tâches seront effectuées principalement en binôme. La composition de chaque binôme variera en fonction des tâches à réaliser et des savoir-faire de chacun.
\\
Une fois qu'une tâche est terminée, si celle-ci a une suite logique, il incombe aux prédécesseurs de prévenir les successeurs.

\section{Travail d'équipe}

\subsection{Communication}
Lorsque des membres du groupe ont besoin de communiquer\footnote{Les moyens de communication sont décrits dans le tableau \ref{OutilsCommunication}}.

\subsection{Binômes}
Toutes les tâches sont réalisées en binôme afin de faciliter le partage des connaissances.
Toujours dans cette optique, les binômes seront constitués d'un membre de chaque groupe pour faciliter la prise en main du projet existant par l'équipe arrivante.

\subsection{Direction}
Un binôme de direction est constitué à chaque itération afin d'avoir une organisation optimale.

\subsection{Réalisation des tâches}
Lorsqu'une tâche est terminée et qu'une autre lui succède directement, il revient au binôme qui précède d'avertir le suivant que la tâche est terminée et qu'il peut prendre la relève.

\section{Réunions}

\subsection{Réunions internes}
Des réunions interne sont organisées afin de garantir une bonne coordination de l'équipe.
Durant ces réunions tous les membres sont présents, et un bilan concis des tâches à réaliser est fait par les responsables de celles-ci.
Avant une réunion interne, un ordre du jour succint est envoyé par mail par la direction du projet pour rappeler l'objectif de la réunion et éventuellement rajouter des points spécifiques à aborder.
\\
Il y a trois réunions internes par semaine, à savoir~:
\begin{itemize}
	\item Lundi~: Bilan du travail du week end~;
	\item Mercredi~: Bilan du travail de début de semaine et préparation de la réunion externe du vendredi~;
	\item Vendredi~: Compte-rendu de la réunion externe et assignation des tâches pour la semaine suivante.
\end{itemize}
Toutes les réunions internes commencent à 14h, se déroulent dans les salles informatiques du 3ième étage, et donnent lieu à une prise de notes sommaires servant de récapitulatif.
Si des retards sont inévitables ou constatés, c'est lors de ces réunions que la planification est ajustée pour respecter au mieux les dates limites.

\subsection{Réunions externes}
Des réunions externes hebdomadaires sont organisées avec le représentant de la société afin de garantir une réactivité optimale concernant ses attentes et ses besoins.
\\
Les réunions externes donnent lieu à la rédaction d'un compte rendu contenant la liste exhaustive des tâches à réaliser, et remis au client au plus tard 48h après la fin de la réunion (dans les jours ouvrables).

\section{Outils de travail}
Afin d'éviter des problèmes d'incompatibilité, les outils suivants seront utilisés~:
\begin{table}[htbp]
	\centering
	\begin{tabularx}{\linewidth}{m{30mm} m{30mm} X}
		\toprule
		\textbf{Nom} & \textbf{Version} & \textbf{URL} \\
		\midrule
		GNU/Linux & Ubuntu 12.04 LTS & www.ubuntu.fr \\
		GNU/Linux & Debian & www.debian.org \\
		Microsoft~Windows & 7 & www.microsoft.com \\
		\bottomrule
	\end{tabularx}
	\caption{Outils de travail~: systèmes d'exploitation}
	\label{OutilsOS}
\end{table}
\begin{table}[htbp]
	\centering
	\begin{tabularx}{\linewidth}{m{30mm} m{30mm} X}
		\toprule
		\textbf{Nom} & \textbf{Version} & \textbf{URL} \\
		\midrule
		Qt (MinGW pour Windows) (32 bits) & 5.2 & - \\
		\bottomrule
	\end{tabularx}
	\caption{Outils de travail~: frameworks (compilateurs)}
	\label{OutilsOS}
\end{table}
\begin{table}[htbp]
	\centering
	\begin{tabularx}{\linewidth}{m{30mm} X}
		\toprule
		\textbf{Type} & \textbf{Commentaires} \\
		\midrule
		E-mail & Client au choix de l'utilisateur; \textbf{tous les membres} du groupe sont mis en copie à chaque envoi de mail.\\
		Téléphone & Dans le cas ou la situation l'exige; e-mail sinon.\\
		\bottomrule
	\end{tabularx}
	\caption{Outils de travail~: communication}
	\label{OutilsCommunication}
\end{table}
\begin{table}[htbp]
	\centering
	\begin{tabularx}{\linewidth}{m{30mm} m{30mm} X}
		\toprule
		\textbf{Nom} & \textbf{Version} & \textbf{URL} \\
		\midrule
		LaTeX & LaTeX2e & - \\
		\bottomrule
	\end{tabularx}
	\caption{Outils de travail~: documentation}
	\label{OutilsDocumentation}
\end{table}
\begin{table}[htbp]
	\centering
	\begin{tabularx}{\linewidth}{m{30mm} m{30mm} X}
		\toprule
		\textbf{Nom} & \textbf{Version} & \textbf{URL} \\
		\midrule
		Github & 1.7.9.5 & https://github.com/raviluminy/gk \\
		\bottomrule
	\end{tabularx}
	\caption{Outils de travail~: partage des fichiers}
	\label{OutilsDepot}
\end{table}

\chapter{Abréviations}
\begin{table}[htbp]
	\centering
	\begin{tabularx}{\linewidth}{m{30mm} X}
		\toprule
		\textbf{Abréviation} & \textbf{Signification} \\
		\midrule
		AMO & Assistance à Maîtrise d'Ouvrage \\
		CdR & Cahier des Recette \\
		CdC & Cahier des Charges \\
		CdCF & Cahier des Charges Fonctionnel \\
		CdCT & Cahier des Charges Technique \\
		CR & Compte-rendu \\
		Go & Giga-octet \\
		LdT & Liste des Tâches \\
		OdJ & Ordre du Jour \\
		PAQ & Plan d'Assurance Qualité \\
		SGBD & Système de Gestion de Bases de Données \\
		SGBDR & Système de Gestion de Bases de Données Relationnelles \\
		DAT & Dossier d'Architecture Technique \\
		PTI & Protocole Technique d'Installation \\
		PdR & Procédure de Recette \\
		DE & Dossier d'Exploitation \\
		PCS & Plan de Continuité des Services \\
		\bottomrule
	\end{tabularx}
	\caption{Abréviations communes}
	\label{AbreviationsCommunes}
\end{table}

\chapter{Gestion de la documentation}
Cette section décrit la manière dont sont gérés les documents relatifs au projet.

\section{Structure des documents}
Tout document produit doit se plier au modèle de document de la société.

\section{Nomenclature}
La liste exhaustive des documents concernés est décrite ci-dessous (hors sous-documents, images et librairies)~:
\begin{enumerate}
	\item \textbf{/Annexe/Docs/}
	\\
	Contient les documents relatifs à \mo, dont~:
	\begin{enumerate}
		\item GARMIR KHATCH - EXEMPLE DE CONTRAT DE TRANSPORT.doc,
		\item GARMIR KHATCH - SUIVI DES TRANSPORTS (EXEMPLE).xlsx,
		\item LOGISTICS REQUISITION.pdf,
		\item WAYBILL DELIVERY NOTE.pdf~;
	\end{enumerate}
	\item \textbf{/Annexe/RE/}
	\\
	Contient les documents relatifs aux réunions externes~:
	\begin{enumerate}
		\item RE-XXX-(YYYY-MM-DD)-CR.tex,
		\item RE-XXX-(YYYY-MM-DD)-LdT.tex,
		\item RE-XXX-(YYYY-MM-DD)-OdJ.tex~;
	\end{enumerate}
	\item \textbf{/Annexe/RI/}
	\\
	Contient les documents relatifs aux réunions internes~:
	\begin{enumerate}
		\item RI-XXX-(YYYY-MM-DD)-CR.tex,
		\item RI-XXX-(YYYY-MM-DD)-LdT.tex,
		\item RI-XXX-(YYYY-MM-DD)-OdJ.tex~;
	\end{enumerate}
	\item \textbf{/AMO/}
	\\
	Contient les documents relatifs à la mission d'AMO~:
	\begin{enumerate}
		\item CdC.tex,
		\item CdCF.tex,
		\item CdCT.tex,
		\item PAQ.tex~;
	\end{enumerate}
	\item \textbf{/ME/}
	\\
	Contient les documents relatifs à la mission de ME~:
	\begin{enumerate}
		\item DAT.tex,
		\item PTI.tex,
		\item PdR.tex,
		\item DE.tex,
		\item PAQ.tex,
		\item PCS.tex.
	\end{enumerate}
\end{enumerate}
Ci-dessous quelques exemples~:
\begin{enumerate}
	\item /Annexe/RE/RE-001-(2014-02-07)-CR.tex~;
	\item /Annexe/RI/RI-003-(2014-02-21)-OdJ.tex~;
	\item /AMO/CdCF.tex~;
	\item /ME/DAT.tex.
\end{enumerate}

\chapter{Planification}
\begin{figure}
	\centering
	\begin{ganttchart}[
			y unit title=0.4cm,
			y unit chart=0.5cm,
			vgrid,
			hgrid,
			title label anchor/.style={below=-1.6ex},
			title left shift=.05,
			title right shift=-.05,
			title height=1,
			bar/.style={draw=black,fill=gray!50},
			incomplete/.style={fill=white},
			progress label text={},
			bar height=0.7,
			group right shift=0,
			group top shift=.6,
			group height=.3,
			group peaks height =.2
		]{1}{28}
		% Titres
		\gantttitle{Itération X}{28} \\
		\gantttitle{Vendredi}{4}
		\gantttitle{Samedi}{4}
		\gantttitle{Dimanche}{4}
		\gantttitle{Lundi}{4}
		\gantttitle{Mardi}{4}
		\gantttitle{Mercredi}{4}
		\gantttitle{Jeudi}{4} \\
		% Tâches
		\ganttgroup{Groupe 1}{1}{8} \\
		\ganttbar{Tâche 1}{1}{2} \\
		\ganttbar{Tâche 2}{3}{8} \\
		\ganttbar{Tâche 3}{9}{10} \\
		\ganttgroup{Groupe 2}{1}{12} \\
		\ganttbar{Tâche 4}{11}{15} \\
		\ganttbar{Tâche 5}{20}{22} \\%[progress=33]
		\ganttgroup{Groupe 3}{1}{12} \\
		\ganttgroup{Groupe 4}{6}{12} \\
		\ganttbar{Tâche 6}{18}{19} \\
		\ganttbar{Tâche 7}{16}{18} \\
		\ganttbar{Tâche 8}{21}{24}%[progress=0]
		% Relations
		\ganttlink{elem0}{elem1}
		\ganttlink{elem0}{elem3}
		\ganttlink{elem1}{elem2}
		\ganttlink{elem3}{elem4}
		\ganttlink{elem1}{elem5}
		\ganttlink{elem3}{elem5}
		\ganttlink{elem2}{elem6}
		\ganttlink{elem3}{elem6}
		\ganttlink{elem5}{elem7}
		\node[fill=white,draw] at ([yshift=-12pt]current bounding box.south){
			\Balde, \Cadon...
		};
%		\node[fill=white,draw,anchor=west] at (current bounding box.north west){Box North West};
%		\node[fill=white,draw,anchor=west] at (current bounding box.north east){Box North East};
	\end{ganttchart}
	\caption{Planification par Gantt}
	\label{PlanificationGantt}
\end{figure}

\section{Chaine de responsabilité}
De manière générale, la gestion des incidents s'articule autour de la chaine de responsabilité suivante.
\\
Dans le cadre d'un incident interne, ce dernier est communiqué à l'un des 2 chefs de projets désignés, par défaut, \Cadon et \Pachy.
\\
Si cet incident est susceptible de présenter des conséquences externes, ce dernier est communiqué au contact client désigné, par défaut, \Cadon, chargé de transférer l'information au client.

\end{document}
