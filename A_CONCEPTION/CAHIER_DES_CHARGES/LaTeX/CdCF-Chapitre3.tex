\chapter{Cadre de réponse}
Le cadre de réponse devra tenir compte des critères de choix du prestataire du projet, du planning de l'appel d'offres ainsi que des procédures et modalités de validation sans oublier le coût de la prestation.
La partie suivante décrit en détail pour chaque fonctions les facteurs énoncés ci-dessus.

\section{Pour chaque fonction}
Pour chaque fonction plusieurs livrables sont attendus :
Mémoire technique
Mémoire financier
Détails du support du service :
Horaires
Garantie
Maintenance
Catalogue de services associés, précisant le cas échéant et pour chacun d'eux, un bordereau de prix unitaire (BPU).
Pour toute question relative au projet, contacter M. Cadon Mehdi-Jonathan ou M. Agopian Roland par mail.

\subsection{Solution proposée}
L'application doit permettre la gestion d'une flotte de véhicule dédiée au transport de marchandises (TMS).
Pour ce faire, il convient que la solution proposée puisse :
Planifier les transports, estimer les coûts
Faire un suivi des transporteurs, livraisons, prestataires de services mais aussi des incidents
Utiliser des tableaux de bord afin de fournir une évaluation des performances de service (taux  de réussite pour la livraison de marchandises)
Éditer des cartes en vue de proposer un itinéraire
Assurer la sauvegarde des données par une solution externe compatible avec les plus utilisées
Rechercher des informations par critères de sélection
Permettre la gestion de documents administratifs (permis, assurances, contrats...)

\subsection{Niveau atteint pour chaque critère d'appréciation de cette fonction et modalités de contrôle}
Pour chaque fonction un critère de droit d'accès est requis. Il conviendra donc de permettre une gestion aisée des droits utilisateurs pour ajouter, modifier supprimer l'accès aux données.
Jalonnement attendus :
Au minimum, les jalons suivants, sont attendus dans l'ordre exprimés. Par le terme \og{}gestion\fg{}, nous entendrons qu'il s'agit de permettre toutes les opérations d'ajout, modification et suppression.

\subsection{Part du prix attribué à chaque fonction}
La part du prix sera ici évaluée au regard du temps passé faut de pouvoir indiquer une part du prix qui n'a pas été chiffré. L'indication n'est pas exactement la même mais peut s'avérer intéressante pour l'évaluation du projet. 
La planification des transports peut être estimée à 30 \% de la part du travail
Le suivi des transport peut être estimé à 20 \% de la part du travail
L'édition des tableaux de bords et des cartes peuvent être estimés à 15 \% de la part du travail
La sauvegarde externalisée peut être estimé à 10 \% de la part du travail
La recherche d'informations par critère peut être estimée à 10 \% de la part du travail
La gestion des documents administratifs peut être estimée à 15 \% de la part du travail
Pour un choix donné de regroupement des fonctions, le soumissionnaire donnera un schéma montrant les échanges entre modules incluant une estimation des flux. 
Ces estimations devront être élaborées sur la base des spécifications fonctionnelles. Il indiquera clairement les hypothèses prises pour faire cette estimation. 

\section{Pour l'ensemble du produit}

\subsection{Prix de la réalisation de la version de base}
Le soumissionnaire devra établir le prix de l'ensemble de sa solution applicative ainsi que des modules la composant.
\\
Il conviendra de fournir également le détail des ressources mobilisés et les coûts associés à la réalisation de chaque module. 

\subsection{Options et variantes proposées non retenues au cahier des charges}
Toutes variantes (modifications, ajouts, suppression) dans la solution proposée devra faire l'objet d'une justification par l'intermédiaire de livrables affirmant que la solution applicative une fois les modifications encourues, reste homogène dans contexte existant et ne dénature pas les objectifs et les contraintes fixés par le projet.
\\
Les solutions proposées non retenues seront néanmoins répertoriées au sein d'un livrable en but d'une consultation ultérieure si nécessaire lors d'une évolution de l'applicatif en cours.

\subsection{Mesures prises pour respecter les contraintes et leurs conséquences économiques}
Nous précisons que le soumissionnaire est laissé libre dans le choix des solutions. En revanche les règles et les choix d'architecture, en particulier pour la plate-forme logistique devront être respectés.

\subsection{Outils d'installation, de maintenance ... à prévoir}
En vue de permettre la facilité d'installation, d'exploitation et de maintenance, plusieurs documents sont préconisés.
Sont attendus les livrables suivants :
Code source
Droits de propriété attendus
Documentation d'exploitation et d'architecture comme décrit ci-dessous :

\subsubsection{Dossier d'Architecture Technique (DAT)}
Le dossier d'architecture technique définit et justifie les hypothèses techniques structurantes du projet. Ce document a vocation à être technique, mais contient des hypothèses qui doivent être validées par les commanditaires du projet. En effet, c'est à partir de ces hypothèses qu'est conçu le projet (et ses services) sur un plan technique, c'est pourquoi elles sont fondamentales.
Ce document s'articule autour de 4 grands volets :
La présentation du projet : son calendrier, ses enjeux, ses exigences et le planning ;
L'architecture fonctionnelle : les utilisateurs, les traitements, les données ;
L'architecture technique : les différents serveurs, les postes de travail, le réseau ;
L'exploitation et sa préparation : la mise en production, le déploiement, la formation, le support et l'exploitation proprement dite.
Il est donc le document fondateur en terme de description de la future solution informatique. Il servira de source à la constitution du Protocole Technique d'Installation (PTI) et du dossier d'exploitation.

\subsubsection{Protocole technique d'installation (PTI)}
Il présente les opérations à réaliser sur les machines de l'exploitant afin d'effectuer sa mise en place.
\\
Ce document permet en principe de rendre l'exploitation totalement autonome du projet dans la phase d'installation du projet et des services.

\subsubsection{Procédure de recette}
La recette informatique désigne l'étape pendant laquelle les futurs utilisateurs testent le projet et des services associés pour vérifier qu'il est conforme au cahier des charges.
La Procédure de recette informatique vise à expliciter les objectifs de la recette, la structure mise en place pour celle-ci, les acteurs concernés, sa préparation, son déroulement ainsi que la gestion des anomalies. Elle identifie en particulier les détails de correction par la maîtrise d'œuvre et de re-livraison.

\subsubsection{Dossier d'exploitation}
Le Dossier d'exploitation informatique vise à donner toutes les informations nécessaires au responsable de l'exploitation informatique quotidienne du projet. Il décrit l'ensemble des procédures à lancer, leur planning, les codes erreur et leur signification, les procédures à suivre pour les traiter, les principes de sauvegardes et les commandes pour les déclencher, la supervision des ressources machines, etc... Muni d'un tel dossier, l'exploitant est autonome et l'application et ses services respectent la qualité de service demandée dans le dossier d'architecture technique et le plan de continuité de service.
\\
Il est construit de façon à ce qu'un exploitant qui n'a absolument pas participé au déploiement de l'outil soit en mesure d'assurer ce travail.

\subsubsection{Plan de continuité des services}
L'objectif du Plan de Continuité de Service est :
D'atteindre le niveau de qualité attendu de l'application et ses services, c'est à dire horaires d'ouverture, reprise de l'activité en cas d'erreur, d'incidents, etc...
De décrire les modalités techniques et opérationnelles dans lesquelles seront assuré le niveau de qualité de service par les moyens informatiques.
Ce document permet de formaliser un engagement conditionné aux ressources mises effectivement en œuvre et décrites dans le document entre l'exploitant informatique et la direction de l'établissement.

\subsection{Décomposition en modules, sous-ensembles}
Quelle que soit la méthode qu'il aura employé pour définir l'architecture applicative, le 
soumissionnaire devra présenter les résultats sous la forme suivante : 
Synthèse de toutes les fonctions ou groupes de fonctions à traiter au sein du projet. Il les regroupe dans des modules. Ces modules correspondent à des ensembles dont les interfaces sont suffisamment bien définies pour pouvoir être développées séparément
Description des modules
Définition des dépendances et interfaces
Définition des flux entre chacun des modules 
Décomposition des modules en sous-modules
Le soumissionnaire pourra apporter tout complément d'information concernant l'architecture proposée au-delà des éléments demandés ci-dessus. 

\subsection{Prévisions de fiabilité}
La solution proposée par le soumissionnaire bénéficiera d'une haute fiabilité en raison de son utilisation dans des environnements hétérogènes.
Un livrable décrivant le taux de fiabilité de chaque module composant la solution applicative est attendu.

\subsection{Perspectives d'évolution technologique}
Chaque évolution de la solution proposée devra tout d'abord faire l'objet d'une analyse de risque afin de s'assurer que la nouvelle version proposée soit efficiente.
Une fois celle-ci validé, il convient au soumissionnaire de fournir des livrables approuvant la non-régression de nouvelle solution et garantissant un taux de fiabilité égale ou supérieure à la solution précédente.
Enfin, toutes évolutions même mineures doivent être listées au sein d'un document afin d'avoir un historique des modifications appliquées à la suite logicielle au même titre que celles figurant dans le mémoire des solutions non retenues.

%----------------------------------------------------------------------------------------
%	CHAPTER 4
%----------------------------------------------------------------------------------------

\chapterimage{chapter_head_2.pdf} % Chapter heading image

\chapter{Text Chapter}

\section{Paragraphs of Text}\index{Paragraphs of Text}

\lipsum[1-7] % Dummy text

%------------------------------------------------

\section{Citation}\index{Citation}

This statement requires citation \cite{book_key}; this one is more specific \cite[122]{article_key}.

%------------------------------------------------

\section{Lists}\index{Lists}

Lists are useful to present information in a concise and/or ordered way\footnote{Footnote example...}.

\subsection{Numbered List}\index{Lists!Numbered List}

\begin{enumerate}
\item The first item
\item The second item
\item The third item
\end{enumerate}

\subsection{Bullet Points}\index{Lists!Bullet Points}

\begin{itemize}
\item The first item
\item The second item
\item The third item
\end{itemize}

\subsection{Descriptions and Definitions}\index{Lists!Descriptions and Definitions}

\begin{description}
\item[Name] Description
\item[Word] Definition
\item[Comment] Elaboration
\end{description}

%----------------------------------------------------------------------------------------
%	CHAPTER 2
%----------------------------------------------------------------------------------------

\chapter{In-text Elements}

\section{Theorems}\index{Theorems}

This is an example of theorems.

\subsection{Several equations}\index{Theorems!Several Equations}
This is a theorem consisting of several equations.

\begin{theorem}[Name of the theorem]
In $E=\mathbb{R}^n$ all norms are equivalent. It has the properties:
\begin{align}
& \big| ||\mathbf{x}|| - ||\mathbf{y}|| \big|\leq || \mathbf{x}- \mathbf{y}||\\
&  ||\sum_{i=1}^n\mathbf{x}_i||\leq \sum_{i=1}^n||\mathbf{x}_i||\quad\text{where $n$ is a finite integer}
\end{align}
\end{theorem}

\subsection{Single Line}\index{Theorems!Single Line}
This is a theorem consisting of just one line.

\begin{theorem}
A set $\mathcal{D}(G)$ in dense in $L^2(G)$, $|\cdot|_0$. 
\end{theorem}

%------------------------------------------------

\section{Definitions}\index{Definitions}

This is an example of a definition. A definition could be mathematical or it could define a concept.

\begin{definition}[Definition name]
Given a vector space $E$, a norm on $E$ is an application, denoted $||\cdot||$, $E$ in $\mathbb{R}^+=[0,+\infty[$ such that:
\begin{align}
& ||\mathbf{x}||=0\ \Rightarrow\ \mathbf{x}=\mathbf{0}\\
& ||\lambda \mathbf{x}||=|\lambda|\cdot ||\mathbf{x}||\\
& ||\mathbf{x}+\mathbf{y}||\leq ||\mathbf{x}||+||\mathbf{y}||
\end{align}
\end{definition}

%------------------------------------------------

\section{Notations}\index{Notations}

\begin{notation}
Given an open subset $G$ of $\mathbb{R}^n$, the set of functions $\varphi$ are:
\begin{enumerate}
\item Bounded support $G$;
\item Infinitely differentiable;
\end{enumerate}
a vector space is denoted by $\mathcal{D}(G)$. 
\end{notation}

%------------------------------------------------

\section{Remarks}\index{Remarks}

This is an example of a remark.

\begin{remark}
The concepts presented here are now in conventional employment in mathematics. Vector spaces are taken over the field $\mathbb{K}=\mathbb{R}$, however, established properties are easily extended to $\mathbb{K}=\mathbb{C}$.
\end{remark}

%------------------------------------------------

\section{Corollaries}\index{Corollaries}

This is an example of a corollary.

\begin{corollary}[Corollary name]
The concepts presented here are now in conventional employment in mathematics. Vector spaces are taken over the field $\mathbb{K}=\mathbb{R}$, however, established properties are easily extended to $\mathbb{K}=\mathbb{C}$.
\end{corollary}

%------------------------------------------------

\section{Propositions}\index{Propositions}

This is an example of propositions.

\subsection{Several equations}\index{Propositions!Several Equations}

\begin{proposition}[Proposition name]
It has the properties:
\begin{align}
& \big| ||\mathbf{x}|| - ||\mathbf{y}|| \big|\leq || \mathbf{x}- \mathbf{y}||\\
&  ||\sum_{i=1}^n\mathbf{x}_i||\leq \sum_{i=1}^n||\mathbf{x}_i||\quad\text{where $n$ is a finite integer}
\end{align}
\end{proposition}

\subsection{Single Line}\index{Propositions!Single Line}

\begin{proposition} 
Let $f,g\in L^2(G)$; if $\forall \varphi\in\mathcal{D}(G)$, $(f,\varphi)_0=(g,\varphi)_0$ then $f = g$. 
\end{proposition}

%------------------------------------------------

\section{Examples}\index{Examples}

This is an example of examples.

\subsection{Equation and Text}\index{Examples!Equation and Text}

\begin{example}
Let $G=\{x\in\mathbb{R}^2:|x|<3\}$ and denoted by: $x^0=(1,1)$; consider the function:
\begin{equation}
f(x)=\left\{\begin{aligned} & \mathrm{e}^{|x|} & & \text{si $|x-x^0|\leq 1/2$}\\
& 0 & & \text{si $|x-x^0|> 1/2$}\end{aligned}\right.
\end{equation}
The function $f$ has bounded support, we can take $A=\{x\in\mathbb{R}^2:|x-x^0|\leq 1/2+\epsilon\}$ for all $\epsilon\in\intoo{0}{5/2-\sqrt{2}}$.
\end{example}

\subsection{Paragraph of Text}\index{Examples!Paragraph of Text}

\begin{example}[Example name]
\lipsum[2]
\end{example}

%------------------------------------------------

\section{Exercises}\index{Exercises}

This is an example of an exercise.

\begin{exercise}
This is a good place to ask a question to test learning progress or further cement ideas into students' minds.
\end{exercise}

%------------------------------------------------

\section{Problems}\index{Problems}

\begin{problem}
What is the average airspeed velocity of an unladen swallow?
\end{problem}

%------------------------------------------------

\section{Vocabulary}\index{Vocabulary}

Define a word to improve a students' vocabulary.

\begin{vocabulary}[Word]
Definition of word.
\end{vocabulary}

%----------------------------------------------------------------------------------------
%	CHAPTER 3
%----------------------------------------------------------------------------------------

\chapterimage{chapter_head_1.pdf} % Chapter heading image

\chapter{Presenting Information}

\section{Table}\index{Table}

\begin{table}[h]
\centering
\begin{tabular}{l l l}
\toprule
\textbf{Treatments} & \textbf{Response 1} & \textbf{Response 2}\\
\midrule
Treatment 1 & 0.0003262 & 0.562 \\
Treatment 2 & 0.0015681 & 0.910 \\
Treatment 3 & 0.0009271 & 0.296 \\
\bottomrule
\end{tabular}
\caption{Table caption}
\end{table}

%------------------------------------------------

\section{Figure}\index{Figure}

\begin{figure}[h]
\centering\includegraphics[scale=0.5]{placeholder}
\caption{Figure caption}
\end{figure}
