\chapter{Expression fonctionnelle du besoin}
Cette section tient un compte détaillé des spécifications fonctionnelles du projet. Sont donc abordés ci-dessous~:
\begin{itemize}
	\item l'ensemble des fonctionnalités de service attendues et de contraintes à respecter~;
	\item les critères d'acceptation~;
	\item les niveaux de critères d'appréciation et ce qui les caractérise.
\end{itemize}

\section{Fonctions de service et de contrainte}
Cette section précise les services attendus et contraintes à respecter.
Il est attendu de la suite logicielle qu'elle réponde à l'architecture de la figure \ref{Architecture générale}.

\subsection{Fonctions de service principales}
Cette section définit l'ensemble des services principaux attendus. Ces derniers sont la raison d'être du produit. Ils sont incontournables.

\subsubsection{La localisation}
La solution proposera des versions traduites en plusieurs langues, y compris avec des alphabets et des sens de lecture différents. Au minimum, les traductions dans les langages suivants devront être fournis~:
\begin{itemize}
	\item Anglais~;
	\item Arabe~;
	\item Espagnol~;
	\item Français.
\end{itemize}
En outre, la solution devra permettre de passer \emph{aisément} d'une langue à l'autre sans nécessiter de redémarrage.

\subsubsection{L'interface utilisateur}
La solution doit fournir au minimum et pour chacun de ces groupes utilisateurs une interface propre permettant de réaliser les actions suivantes~:
\begin{itemize}
	\item gestionnaire de transport~:
	\begin{itemize}
		\item moyen de contrôle des opérations (à venir, en cours...),
		\item outil d'édition de tableaux de bord~;
	\end{itemize}
	\item gestionnaire de la planification~:
	\begin{itemize}
		\item fournir aux métiers des données relatives à l'état des transports en cours~;
	\end{itemize}
	\item gestionnaire des prestataires~:
	\begin{itemize}
		\item édition de bons pour accord de paiement~;
	\end{itemize}
	\item métiers~:
	\begin{itemize}
		\item consultation des transports en cours en relation avec le métier~;
	\end{itemize}
	\item secrétariat central~:
	\begin{itemize}
		\item accès aux documents sous forme numérique,
		\item statistiques.
	\end{itemize}
\end{itemize}

\subsubsection{Les cas d'utilisations}
Cette section présente le diagramme des cas d'utilisation et précise pour chacun d'eux l'ensemble des actions attendues.
\begin{figure}[htbp]
	\begin{tikzpicture}
		\begin{umlsystem}[x=8.00,y=10.00,fill=red!10]{Administration}
			\umlusecase[x=0.00,y=0.00,name=GSauvegarde]{Gestion des sauvegardes}
			\umlusecase[x=0.00,y=1.00,name=GDroits]{Gestion des droits d'accès}
			\umlusecase[x=0.00,y=2.00,name=GSécurité]{Gestion des niveaux de sécurité (cryptographie)}
		\end{umlsystem}
		\begin{umlsystem}[x=8.00,y=0.00,fill=green!10]{Gestion des ressources}
			\umlusecase[x=0.00,y=0.00,name=GPrestataires]{Gestion des prestataires}
			\umlusecase[x=0.00,y=1.00,name=GTransports]{Gestion des moyens de transport}
			\umlusecase[x=0.00,y=2.00,name=GChauffeurs]{Gestion des chauffeurs}
			\umlusecase[x=0.00,y=3.00,name=GRéquisitions]{Gestion des réquisitions}
			\umlusecase[x=0.00,y=4.00,name=GWaybills]{Gestion des waybills/delivery~notes}
			\umlusecase[x=0.00,y=5.00,name=GTableauxBord]{Gestion des tableaux de bord}
			\umlusecase[x=0.00,y=6.00,name=GSynchronisation]{Gestion de la synchronisation}
			\umlusecase[x=0.00,y=7.00,name=GPréférences]{Gestion des préférences}
		\end{umlsystem}
		\umlactor[x=0.00,y=10.00]{Administrateur}
		\umlactor[x=0.00,y=3.00]{Utilisateur}
		\umlinherit{Administrateur}{Utilisateur}
		\umlassoc{Utilisateur}{GPrestataires}
		\umlassoc{Utilisateur}{GTransports}
		\umlassoc{Utilisateur}{GChauffeurs}
		\umlassoc{Utilisateur}{GRéquisitions}
		\umlassoc{Utilisateur}{GWaybills}
		\umlassoc{Utilisateur}{GTableauxBord}
		\umlassoc{Utilisateur}{GSynchronisation}
		\umlassoc{Utilisateur}{GPréférences}
		\umlassoc{Administrateur}{GSauvegarde}
		\umlassoc{Administrateur}{GDroits}
		\umlassoc{Administrateur}{GSécurité}
		%\umlinherit{usecase-2}{usecase-1}
		%\umlVHextend{usecase-5}{usecase-4}
		%\umlinclude[name=incl]{usecase-3}{usecase-4}
		%\umlnote[x=7, y=-7]{incl-1}{note on include dependency}
	\end{tikzpicture}
	\caption{Diagramme des cas d'utilisation}
	\label{ucd}
\end{figure}

\paragraph{Gestion des niveaux de sécurité (cryptographie)}
Ce cas d'utilisation comprend la possibilité de définir les règles de sécurités appliquées aux communications sur un lieu d'intervention. Par règles de sécurité sont entendus les éléments suivants~:
\begin{itemize}
	\item les algorithmes de chiffrement utilisés lors des échanges de données sur le réseau.
\end{itemize}
Initialement, sont attendus à minima, l'ensemble des solutions suivantes~:
\begin{itemize}
	\item primitives d'échange de clé~:
	\begin{itemize}
		\item NULL~: on se réserve la possibilité de ne pas crypter l'échange de clé,
		\item RSA~: conformément à la \href{http://tools.ietf.org/html/rfc3447}{RFC 3447},
		%\item DH\_DSS,
		%\item DH\_RSA,
		%\item DHE\_DSS,
		%\item DHE\_RSA,
		%\item DH\_anon~;
	\end{itemize}
	\item primitives de chiffrement~:
	\begin{itemize}
		\item NULL~: on se réserve la possibilité de ne pas chiffrer l'échange,
		\item AES~: conformément à la \href{http://tools.ietf.org/html/rfc3394}{RFC 3394},
		%\item RC4\_128,
		%\item 3DES\_EDE\_CBC,
		%\item AES\_128\_CBC,
		%\item AES\_256\_CBC~;
	\end{itemize}
	\item primitives de hachage~:
	\begin{itemize}
		\item NULL~: on se réserve la possibilité de ne pas signer l'échange,
		\item MD5~: conformément à la \href{http://tools.ietf.org/html/rfc1321}{RFC 1321},
		\item SHA~: conformément à la \href{http://tools.ietf.org/html/rfc3174}{RFC 3174},
		%\item SHA256.
	\end{itemize}
\end{itemize}
La cryptographie étant un domaine en constante évolution, il est attendu de la suite logicielle qu'elle permette d'ajouter et supprimer facilement des solutions cryptographiques. 
\begin{constraint}[Contraintes sur la sécurité des données échangées]
Les situations d'urgence ne dispensent pas les acteurs humanitaires de se conformer à la législation des pays dans lesquels ils interviennent, notamment pour ce qui est des moyens de sécurisation des échanges de données. La transparence est si possible de mise, afin de ne pas instiller de doutes et mettre en particulier en danger la crédibilité et l'action de l'Organisation. En outre, des interventions dans des territoires potentiellement conflictuels ou sous contrôle militaire strict - Cachemire par exemple - quand bien même celles-ci restent très exceptionnelles eu égard au mandat de l'Organisation sont possibles. Dans ce cadre, l'utilisation de moyens de sécurité des données transmises (cryptage) peut être proscrite ou tout du moins adaptée justifiable et adaptée à une cryptanalyse \og{}rapide\fg{}. La suite logicielle doit donc permettre d'effectuer ces changements \emph{facilement}.
\end{constraint}

\paragraph{Gestion des droits d'accès}
Chaque profil utilisateur peut recevoir des droits d'accès spécifiques et/ou être lié à \emph{un ou plusieurs} groupe d'utilisateur.
Chaque groupe d'utilisateurs doit être associé à un ensemble de droits d'accès.
Au minimum, la suite logicielle doit fournir un groupe d'utilisateurs particulier, appelé \emph{administrateurs} et détenteurs de tous les droits d'accès disponibles (cf. figure \ref{ucd}). L'opportunité souhaitant être laissée de pouvoir gérer (ajouter, modifier et/ou supprimer) utilisateurs et groupes d'utilisateurs et leurs droits associés, ce dernier doit permettre d'opérer facilement ces manipulations.

\begin{figure}[htbp]
	\centering
	\begin{tikzpicture}
		\begin{scope}[xscale=1,yscale=1]
			\tikzstyle{block} = [rectangle,draw=ocre,fill=black!5,text=black,text centered]
			\tikzstyle{arrow} = [->,>=latex]
			% description et nommage des noeuds
			\node[block] (U)  at (-4.00,0.00) {\begin{tabular}{c}Utilisateur\end{tabular}};
			\node[block] (GU) at (0.00,0.00) {\begin{tabular}{c}Groupe d'utilisateurs\end{tabular}};
			\node[block] (DA) at (4.30,0.00) {\begin{tabular}{c}Droits d'accès\end{tabular}};
			% description des arêtes
			% -- arête rectiligne entre les noeuds nommés
			\draw[arrow] (U)  -- (GU);
			\draw[arrow] (U) to[bend right] (DA);
			\draw[arrow] (GU) -- (DA);
			% |- départ vertical arrivée horizontale
			% -| départ horizontal (du noeud de coordonnée (0,1)) arrivée verticale
		\end{scope}
	\end{tikzpicture}
	\caption{Utilisateurs, groupes d'utilisateurs et droits d'accès}
	\label{ar}
\end{figure}

\paragraph{Gestion des sauvegardes}
La suite logicielle doit être pourvu d'un système de sauvegarde externalisé. Il est attendu de ce système qu'il permette les manipulations suivantes~:
\begin{itemize}
	\item lancer une sauvegarde (aussi bien distante que locale)~;
	\item charger une sauvegarde existante (qu'elle soit distante ou locale)~;
	\item ajouter, modifier, supprimer, importer et exporter une configuration de sauvegarde. Une configuration étant chargée de rassembler les paramètres suivants~:
	\begin{itemize}
		\item la liste exhaustive des données sauvegardées,
		\item la planification (date, heure, minute, ...) et ses répétitions (1 fois, tous les jours - ou seulement certains, toutes les semaines, tous les mois, ...),
		\item le support de réception qu'il soit distant ou local (serveur, client, baie de disque, disque dur, disque optique, clé USB, ...).
	\end{itemize}
\end{itemize}
\begin{constraint}[Contrainte sur le système de sauvegarde]
Il est demandé que le système de sauvegarde soit compatible avec le top 10 des solutions de sauvegarde les plus communes, aux cas où l'une d'entre elle soit mise en place à postériori.
\end{constraint}

\paragraph{Gestion des préférences}
La suite logicielle devrait pouvoir permettre aux utilisateurs de garder en mémoire leurs préférences, comme la langue utilisée. L'objectif étant d'épargner aux utilisateurs la charge de devoir constamment régler ces paramètres à chaque utilisation.
\begin{notation}
\emph{Fonctions de service complémentaires.}
\\
Cette fonction participe à améliorer, faciliter ou compléter le service rendu. Elle est souhaitée mais non prioritaires.
\end{notation}

\paragraph{Gestion de la synchronisation}
La suite logicielle doit gérer explicitement la synchronisation des données, à la fois entre postes clients et serveur local, mais également entre serveur local et serveur central. Il est important de comprendre que le travail sur la suite logicielle ne doit pas s'arrêter ni subir de dégradation malgré l'absence de connexion ou de réseau. Les services doivent donc continuer de tourner en local, jusqu'à la reprise d'activité du réseau, moment à partir duquel les données modifiées doivent être synchronisées. Entre autres choses, les conflits doivent être gérés efficacement sans perturber les opérations métier déjà en cours. Enfin, outre l'aspect automatisé demandé, il est également souhaité de pouvoir gérer manuellement la synchronisation.
\begin{constraint}[Contraintes sur les moyens de communication]
En situation d'urgence, les moyens de communication disponibles notamment dans les régions affectées peuvent être \og{}réduits\fg{} (réseau GSM indisponible, par exemple). Les procédures de l'Organisation mettent, certes, à disposition des équipes terrain, pour leur propre sécurité, des moyens de communication de type satellitaires, mais les coûts de communication restent élevés qu'ils soient basés sur la quantité de données échangés ou sur les temps de communication. De fait, préférence est donnée aux moyens \og{}standards\fg{} de communication (ADSL, GSM...). Afin de minimiser les coûts, il est donc \emph{impératif} de pouvoir faire fonctionner la suite logicielle en mode \og{}hors ligne\fg{}, pour synchroniser les informations une fois les moyens de communication retrouvés.
\end{constraint}

\paragraph{Gestion des tableaux de bord}
La suite logicielle doit permettre de générer des \emph{tableaux de bord}, qui sont des documents de synthèse servant de supports qualité et contenant les informations préalablement choisies en fonction du destinataire du document.
\\
Par exemple, pour une livraison donnée, on peut distinguer deux types de documents synthèse (liste non exhaustive) en fonction du destinataire~:
\begin{itemize}
\item le métier recevrait un document mettant en valeur les lieux et temps de chargement des marchandises avec les dates de chaque opération en rapport avec le contenu de la livraison~;
\item le client pourrait recevoir le même type de document mais avec les prix de stockage, location des véhicules, coût du trajet, du carburant, ...
\end{itemize}

\paragraph{Gestion des réquisitions \& waybills/delivery~notes}
De nombreux types de documents sont traités en interne et doivent pouvoir être enregistrés, numérisés, affichés, produits, modifiés, supprimés et archivés par la suite logicielle. Tous les documents en question disposent d'un numéro d'identification unique, généré par le secrétariat central. Les documents concernés sont les suivants~:
\begin{itemize}
	\item permis de conduire des chauffeurs~;
	\item cartes d'identité des chauffeurs~;
	\item assurances des véhicules~;
	\item patentes des prestataires~;
	\item contrats avec les prestataires~;
	\item réquisition logistique~;
	\item waybills/delivery~notes.
\end{itemize}
Dans le cas général, une \emph{réquisition} correspond à une mission d'expédition répondant à un besoin métier formulé. Ce besoin peut nécessiter 1 ou plusieurs moyens de transports, pour le(s)quel(s) un \emph{waybill/delivery~note} est produit.
\\
Les deux cas suivants sont donc possibles~:
\\
1 \emph{réquisition} -> 1 \emph{waybill/delivery~note}~;
\\
1 \emph{réquisition} -> n \emph{waybills/delivery~notes}.
\\
Un 3ième cas existe également, lors duquel un partenaire émet une \emph{réquisition} pour l'une de ses missions. Dans ce cas de figure, les \emph{waybills/delivery~notes} étant produit par le dit partenaire, la \emph{réquisition} ne conduit à aucune production de documents supplémentaires~:
\\
1 \emph{réquisition} -> 0 \emph{waybills/delivery~notes}.
\\
La suite logicielle doit être en mesure de gérer chacun des cas de figure énoncé, ainsi que d'ajouter, modifier, supprimer, importer, exporter, imprimer et archiver ces deux documents ainsi que l'ensemble des informations qu'ils véhiculent (cf. annexe).
\\
\begin{notation}
\emph{Heuristique organisationnelle.}
\\
Rappelons que l'objectif étant d'optimiser les ressources engagées, il est souhaité que la suite logicielle fournisse une aide à la décision dans l'attribution des moyens de transport, via une heuristique analytique.
\\
De même, il serait souhaité de faciliter l'organisation actuelle via l'analyse des logs d'activité. Cette heuristique pourrait notamment repérer des enchaînements logiques d'actions et ainsi proposer un accès rapide aux successeurs les plus fréquents.
\end{notation}

\paragraph{Gestion des chauffeurs, des transports et des prestataires}
La suite logicielle doit permettre d'ajouter, modifier, supprimer, importer, exporter, imprimer et archiver chauffeurs, transports et prestataires ainsi que l'ensemble des informations qui leur sont associés, dont à minima, les suivantes~:
\begin{itemize}
	\item carte d'identité (numérisé) des chauffeurs~;
	\item assurances (numérisé) des transports~;
	\item contrats avec les prestataires~;
	\item patentes (numérisé) des prestataires~;
\end{itemize}
\begin{constraint}[Contrainte de fonctionnalité]
La suite logicielle doit fournir un outil intégré permettant de numériser les documents qui doivent l'être, via une interface graphique clair permettant de fixer facilement les paramètres de numérisation (poids de l'image, résolution, couleur, ...)
\end{constraint}

\subsection{Fonctions de service complémentaires}
Cette section définit les fonctions de service complémentaires. Ces dernières améliorent, facilitent ou complètent le service rendu. Elles sont souhaitées mais non prioritaires.

\subsubsection{Les statistiques}
La solution devrait permettre de produire un ensemble de statistiques à partir des informations enregistrées. Le choix de ces dernières n'étant pas encore arrêtées, une conception modulaire serait préférée de sorte à pouvoir en ajouter ou en supprimer facilement, leur prix devant être précisé dans un bordereau de prix unitaire ou un forfait.

\subsection{Contraintes}

\subsubsection{Contraintes de délai}
Le prestataire s'engage à fournir des garanties contractuelles de délais, et à s'y tenir. Les éventuelles pénalités de retard seront fixées d'un accord commun avec \mo.

\subsubsection{Contraintes techniques}

\paragraph{Architecture serveur}
La solution serveur doit être compatible avec les environnements Microsoft~Windows et GNU/Linux.
\\
En outre, l'utilisation d'un SGBD semblant inévitable, la solution doit également être pleinement compatible avec les SGBDR SQL~Oracle et MySQL~Server et permettre de s'interfacer facilement avec d'autres SGBD.

\paragraph{Architecture client}
La solution cliente doit être pleinement fonctionnelle sous les distributions Microsoft~Windows~7, qui ont 16Go de RAM et 250Go de mémoire, ainsi que sous les dernières versions en date des navigateurs Internet~Explorer et Mozilla~Firefox.
\\
Enfin, La solution doit être accessible depuis des appareils de téléphonie mobile sous Android.

\subsubsection{Contraintes légales}
Les contraintes légales de chaque pays s'appliquant à \mo lors de ses interventions, la solution devra être aux normes (ou pouvoir s'y adapter) de ces pays, notamment en terme de~:
\begin{itemize}
\item droit d'accès à l'information~;
\item conservation des données personnelles~;
\item cryptographie.
\end{itemize}

\subsubsection{Contraintes réglementaires}
\mo a une image et une éthique mondialement connue découlant de ses activités. Il conviendra de la prendre en compte lors de la réalisation de la solution.

\section{Critères d'appréciation}
La solution sera évaluée en fonction des critères suivants~:
\begin{itemize}
\item disponibilité~;
\item capacité~;
\item sécurité~;
\item continuité~;
\item accessibilité~;
\item ressources matérielles consommées~;
\item trafic réseau~;
\item modularité~;
\item compatibilité~;
\item heuristique~;
\item planification~;
\item statistiques~;
\item gestion des documents.
\end{itemize}

\section{Niveaux des critères d'appréciation et ce qui les caractérise}
Plus un niveau est faible, plus ce critère est apprécié.
\\
\begin{tabularx}{\linewidth}{>{\centering}m{2.0cm} c X}%{|p{3.0cm}|c|p{10.0cm}|}%
\toprule
Critère & Niveau & Détails
\\\midrule
Disponibilité & 1 & La solution devra être utilisable dans les conditions et les termes convenus.
\\\hline
Capacité & 1 & L'ensemble du personnel logistique doit pouvoir avoir un accès simultané, et le produit devra prendre en compte une évolution du nombre d'utilisateurs en fonction de la croissance de \mo.
\\\hline
Sécurité & 1 & L'association manipulant des informations personnelles, le produit devra gérer la sécurité des communications et du stockage des informations (en prenant en compte les contraintes légales des états du lieu d'intervention).
\\\hline
Continuité & 1 & Le prestataire fournira des solutions pour assurer la continuité, comme un plan de reprise d'activité pour assurer un rétablissement de la disponibilité dans des délais optimaux.
\\\hline
Accessibilité & 2 & La suite logicielle pourra être utilisée par du personnel sans compétences fortes dans le domaine des technologies de l'information, et par des équipes internationales. À cette fin, l'accessibilité, l'interface, la prise en main, et le nombre de langues seront des facteurs d'évaluation de la solution.
\\\hline
Ressources matérielles consommées & 4 & La consommation de ressources entraîne des coûts (e.g. utilisation des serveurs), et des contraintes techniques (e.g. consommation de batterie sur les clients).
\\\hline
Trafic réseau & 2 & Les coûts de communications variant en fonction du support réseau (internet, GSM, satellitaire), la consommation de bande passante sera également un point déterminant.
\\\hline
Modularité & 4 & La modularité correspond à la minimisation du travail à réaliser dans le cas d'une éventuelle évolution de la suite logicielle.
\\\hline
Compatibilité & 3 & La solution devra avoir un niveau suffisant de compatibilité avec les normes et standards les plus répandus.
\\\hline
Heuristique & 5 & \og{}L'intelligence\fg{} de l'heuristique organisationnelle pour la réalisation de tâches récurrentes et la planification.
\\\hline
Planification & 2 & La planification des transports, livraisons, temps de chargement, et d'attente.
\\\hline
Statistiques & 4 & La génération des statistiques fonctionnelles et organisationnelles.
\\\hline
Gestion des documents & 3 & La production, l'édition, la modification, la numérisation des documents correspondra aux documents actuellement utilisés et aux standards de l'organisation.
\\\bottomrule
\end{tabularx}

\subsection{Niveaux dont l'obtention est imposée}
Les niveaux 1 à 3 sont imposés.

\subsection{Niveaux souhaités mais révisables}
Les niveaux strictement supérieurs à 3 sont souhaités.
