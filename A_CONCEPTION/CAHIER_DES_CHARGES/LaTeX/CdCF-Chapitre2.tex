\chapter{Expression fonctionnelle du besoin}
Cette section tient un compte détaillé des spécifications fonctionnelles du projet. Sont donc abordés ci-dessous :
l'ensemble des fonctionnalités de service attendues et de contraintes à respecter ;
les critères d'acceptation ;
les niveaux de critères d'appréciation et ce qui les caractérise. 

\section{Fonctions de service et de contrainte}
Cette section précise les services attendus et contraintes à respecter.

\subsection{Fonctions de service principales}
Cette section définit l'ensemble des services principaux attendus. Ces derniers sont la raison d'être du produit. Ils sont incontournables.

\subsubsection{La localisation}
La solution proposera des versions traduites en plusieurs langues, y compris avec des alphabets et des sens de lecture différents. Au minimum, les traductions dans les langages suivants devront être fournis :
Anglais ;
Arabe ;
Espagnol ;
Français ;
En outre, la solution devra permettre de passer aisément d'une langue à l'autre sans nécessiter de redémarrage.

\subsubsection{Les utilisateurs}
Différents profils d'utilisateurs peuvent être amenés à profiter des fonctionnalités de la suite logicielle comme décrits dans les paragraphes subséquents. 
Si dans un premier temps la solution est destinée à une utilisation temps réel, il conviendra de rendre exploitable les informations jugées importantes par les serveurs locaux du site.
L'architecture de la suite logicielle devra de ce fait, s'adapter à l'environnement dans laquelle elle est utilisée en vue de permettre l'accès aux données même si l'utilisateur n'est pas à proximité du serveur central.

\subsubsection{Les groupes d'utilisateurs}
Chaque profil utilisateur doit être lié à un et un seul groupe d'utilisateur.
Chaque groupe d'utilisateurs doit être associé à un ensemble de droits d'accès.
Au minimum, la solution doit fournir un groupe d'utilisateurs particulier, appelé \og{}administrateurs\fg{} et détenteurs de tous les droits d'accès disponibles. L'opportunité souhaitant être laissée de pouvoir gérer (ajouter, modifier et/ou supprimer) les groupes utilisateurs et leurs droits associés, ce dernier doit permettre d'opérer facilement ces manipulations.

\subsubsection{Les actions}
Gérer les sociétés de transport -> Contrats, bon pour accord de paiement
Gérer les véhicules
Gérer les chauffeurs
Gérer les missions de transport -> Tableaux de bord

\subsubsection{L'interface utilisateur}
La solution doit fournir au minimum et pour chacun de ces groupes utilisateurs une interface propre permettant de réaliser les actions suivantes :
Gestionnaire de transport :
Moyen de contrôle des opérations (à venir, en cours...)
Outil d'édition de tableaux de bord
...
Gestionnaire de la planification :
Fournir aux métiers des données relatives à l'état des transports en cours
...
Gestionnaire des prestataires :
édition de bons pour accord de paiement
...
Métiers :
Consultation des transports en cours en relation avec le métier
Secrétariat central :
Accès aux documents sous forme numérique
Statistiques
...

\subsubsection{Les documents internes}
De nombreux types de documents sont traités en interne et doivent pouvoir être enregistrés, numérisés, lus, produits, modifiés, supprimés et archivés par la suite logicielle. Tous les documents en question disposent d'un numéro d'identification unique, généré par le secrétariat central. Les documents concernés sont les suivants :
Relatif au(x) chauffeur(s) :
Permis de conduire
Carte d'identité
Relatif au(x) véhicule(s) :
Assurances
Relatif au(x) société(s) de transport :
Patentes
Contrats
Relatif au(x) mission(s) de transport :
Réquisition :
Une réquisition avec un numéro unique sur un même lieu d'intervention : Dans le cas général, une réquisition correspond à un Bordereau d'expédition (waybill out). Cependant, il y a des cas où une même réquisition impliquerait plusieurs transports. Par exemple un seul numéro de réquisition peut être liée à dix Bordereaux d'expédition. 1 réquisition (définitive ou temporaire) -> 1 ou plusieurs bordereau(x) d'expédition.
Dans le cas où un ou plusieurs transports sont envoyés à un organisme externe, et si ce dernier a la charge de définir la mission du transport et par conséquent son ou ses bordereau(x) d'expédition, une réquisition ne serait liée à aucun bordereaux d'expédition. 1 réquisition (définitive ou temporaire) -> 0 bordereau d'expédition.
Bordereau d'expédition (Waybill / Delivery note) :
Tient compte des informations suivantes :
L'origine du produit transporté (le superviseur, le lieu, la date de départ planifiée, la date et l'horaire de départ effectif)
La destination du produit (la date d'arrivée planifiée, la date et l'horaire d'arrivée effectif)
Les dates de chargement et déchargement
L'identifiant du transport utilisé (les données associées telles que le genre ou la capacité volumique) 
Les données du transporteur (état civil, le type de permis)
La date et la raison d'une annulation éventuelle du transport 
La date et la description de tout type de problème rencontré

\subsubsection{La synchronisation}
La solution devra gérer explicitement la synchronisation et les paramètres associés comme (liste non exhaustive) :
éléments à synchroniser
Permissions
Gestion des conflits

\subsubsection{Les cas d'utilisations}

\subsection{Fonctions de service complémentaires}
Cette section définit les fonctions de service complémentaires. Ces dernières améliorent, facilitent ou complètent le service rendu. Elles sont souhaitées mais non prioritaires.

\subsubsection{Les statistiques}
La solution devrait permettre de produire un ensemble de statistiques à partir des informations enregistrées. Le choix de ces dernières n'étant pas encore arrêtées, une conception modulaire serait préférée de sorte à pouvoir en ajouter ou en supprimer facilement, leur prix devant être précisé dans un bordereau de prix unitaire ou un forfait.

\subsubsection{Les préférences d'utilisateurs}
La solution devrait pouvoir permettre aux utilisateurs de garder en mémoire leurs préférences, comme la langue utilisée. L'objectif étant d'épargner aux utilisateurs la charge de devoir constamment régler ces paramètres à chaque utilisation.

\subsubsection{Heuristique organisationnelle}
Une fonctionnalité complémentaire est la création d'une heuristique permettant de faciliter l'organisation via l'analyse des logs d'activité. Elle pourrait notamment repérer des enchaînements logiques d'actions et ainsi proposer un accès rapide aux successeurs les plus fréquents.

\subsubsection{Génération de documents "sur mesure"}
Le produit pourra permettre de générer des \og{}tableaux de bord\fg{} qui sont des documents de synthèse servant de supports qualité et contenant les informations préalablement choisies en fonction du destinataire du document.
\\
Par exemple, pour une livraison donnée, on peut distinguer deux types de documents synthèse (liste non exhaustive) en fonction du destinataire :
Le métier recevrait un document mettant en valeur les lieux et temps de chargement des marchandises avec les dates de chaque opération en rapport avec le contenu de la livraison
Un client pourrait recevoir le même type de document mais avec les prix de stockage, location des véhicules, coût du trajet et du carburant, ...

\subsection{Contraintes}

\subsubsection{Contraintes de délai}
Le prestataire s'engage à fournir des garanties contractuelles de délais, et à s'y tenir. Les éventuelles pénalités de retard seront fixées d'un accord commun avec Garmir Khatch.

\subsubsection{Contraintes techniques}

\paragraph{Architecture serveur}
La solution serveur doit être compatible avec les environnements Microsoft Windows et GNU/Linux.
\\
En outre, l'utilisation d'un SGBD semblant inévitable, la solution doit également être pleinement compatible avec les SGBDR SQL Oracle et MySQL Server et permettre de s'interfacer facilement avec d'autres SGBD.

\paragraph{Architecture client}
La solution cliente doit être pleinement fonctionnelle sous les distributions Microsoft Windows 7, qui ont 16Go de RAM et 250Go de mémoire, ainsi que sous les dernières versions en date des navigateurs Internet Explorer et Mozilla Firefox.
\\
Enfin, La solution doit être accessible depuis des appareils de téléphonie mobile sous Android.

\paragraph{Contraintes sur les moyens de communication}
En situation d'urgence, les moyens de communication disponibles notamment dans les régions affectées peuvent être \og{}réduits\fg{} (réseau GSM indisponible, par exemple). Les procédures de l'Organisation mettent, certes, à disposition des équipes terrain, pour leur propre sécurité, des moyens de communication de type satellitaires, mais les coûts de communication restent élevés qu'ils soient basés sur la quantité de données échangés ou sur les temps de communication. De fait, préférence est donnée aux moyens \og{}standards\fg{} de communication (ADSL, GSM...). Il reste néanmoins qu'il doit être possible d'adapter l'utilisation de la suite logicielle aux contraintes de coûts, de débits... des moyens de communication.

\paragraph{Synchronisation}
Les communications réseau n'étant pas toujours en place dans tous les endroits de la zone d'intervention, il doit être possible de faire fonctionner la solution en mode \og{}hors ligne\fg{} et gérer la synchronisation une fois les moyens de communication retrouvés.

\paragraph{Sauvegarde}
La solution doit gérer sa propre sauvegarde externalisée et être compatible avec le top 10 des solutions de sauvegarde les plus communes (dans le cas où l'une d'entre elles soit utilisée à posteriori).
\\
La solution doit fournir un moyen simple de gérer la sauvegarde, afin de la lancer aussi bien de façon manuelle qu'automatique. Les éléments suivant doivent être paramétrables (liste non-exhaustive) :
Planification périodique,
Contenu de la sauvegarde,
Période à sauvegarder.

\subsubsection{Contraintes légales}
Les contraintes légales de chaque pays s'appliquant à Garmir Khatch lors de ses interventions, la solution devra être aux normes (ou pouvoir s'y adapter) de ces pays, notamment :
Droit d'accès à l'information,
Conservation des données personnelles,
Cryptographie.

\subsubsection{Contraintes réglementaires}
Garmir Khatch a une image et une éthique mondialement connue découlant de ses activités. Il conviendra de la prendre en compte lors de la réalisation de la solution.

\subsubsection{Contraintes de sécurité}
En fonction de la situation, il doit pouvoir être possible de définir le niveau de cryptage des données échangées sur le réseau, voir désactiver ce dernier.

\paragraph{Contraintes sur la sécurité des données échangées}
Les situations d'urgence ne dispensent pas les acteurs humanitaires de se conformer à la législation des pays dans lesquels ils interviennent, notamment pour ce qui est des moyens de sécurisation des échanges de données. La transparence est si possible de mise, afin de ne pas instiller de doutes et mettre en particulier en danger la crédibilité et l'action de l'Organisation. En outre, des interventions dans des territoires potentiellement conflictuels ou sous contrôle militaire strict - Cachemire par exemple - quand bien même celles-ci restent très exceptionnelles eu égard au mandat de l'Organisation sont possibles. Dans ce cadre, l'utilisation de moyens de sécurité des données transmises (cryptage) peut être proscrite ou tout du moins adaptée justifiable et adaptée à une cryptanalyse \og{}rapide\fg{}.

\section{Critères d'appréciation}
La solution sera évaluée en fonction des critères suivants :
Disponibilité
Capacité
Sécurité
Continuité
Accessibilité
Ressources matérielles consommées
Trafic réseau
Modularité
Compatibilité
Heuristique
Planification
Statistiques
Gestion des documents

\section{Niveaux des critères d'appréciation et ce qui les caractérise}
Plus un niveau est faible, plus ce critère est apprécié.
Critère
Niveau
Détails
Disponibilité
1
La solution devra être utilisable dans les conditions et les termes convenus.
Capacité
1
L'ensemble du personnel logistique doit pouvoir avoir un accès simultané, et le produit devra prendre en compte une évolution du nombre d'utilisateurs en fonction de la croissance de Garmir Khatch.
Sécurité
1
L'association manipulant des informations personnelles, le produit devra gérer la sécurité des communications et du stockage des informations (en prenant en compte les contraintes légales des états du lieu d'intervention).
Continuité
1
Le prestataire fournira des solutions pour assurer la continuité, comme un plan de reprise d'activité pour assurer un rétablissement de la disponibilité dans des délais optimaux.
Accessibilité
2
La suite logicielle pourra être utilisée par du personnel sans compétences fortes dans le domaine des technologies de l'information, et par des équipes internationales. À cette fin, l'accessibilité, l'interface, la prise en main, et le nombre de langues seront des facteurs d'évaluation de la solution.
Ressources matérielles consommées
4
La consommation de ressources entraîne des coûts (e.g. utilisation des serveurs), et des contraintes techniques (e.g. consommation de batterie sur les clients).
Trafic réseau
2
Les coûts de communications variant en fonction du support réseau (internet, GSM, satellitaire), la consommation de bande passante sera également un point déterminant.
Modularité
4
La modularité correspond à la minimisation du travail à réaliser dans le cas d'une éventuelle évolution de la suite logicielle
Compatibilité
3
La solution devra avoir un niveau suffisant de compatibilité avec les normes et standards les plus répandus .
Heuristique
5
\og{}L'intelligence\fg{} de l'heuristique organisationnelle pour la réalisation de tâches récurrentes et la planification
Planification
2
La planification des transports, livraisons, temps de chargement, et d'attente
Statistiques
4
La génération des statistiques fonctionnelles et organisationnelles
Gestion des documents
3
La production, l'édition, la modification, la numérisation des documents correspondra aux documents actuellement utilisés et aux standards de l'organisation

\subsection{Niveaux dont l'obtention est imposée}
Les niveaux 1 à 3 sont imposés.

\subsection{Niveaux souhaités mais révisables}
Les niveaux strictement supérieurs à 3 sont souhaités.
