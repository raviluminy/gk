%-------------------------------------------------------------------------------
%
%     CHARGEMENT DES EXTENSIONS
%
%-------------------------------------------------------------------------------

\documentclass[11pt,fleqn]{report}

\usepackage{garmirkhatch}



%-------------------------------------------------------------------------------
%
%     GLOBAL VALUES
%
%-------------------------------------------------------------------------------


%-------------------------------------------------------------------------------
%     Informations spécifiques au document
%-------------------------------------------------------------------------------

\ZTitle{Système de gestion des transports}
\ZSubject{Cahier des charges}
\ZVersion{2.0}
\ZDate{2014-03-03}
\ZAuthor{\Balde,\\\Cadon,\\\Gairoard,\\\Julien,\\\Lericolais,\\\Mezelle,\\\Pachy,\\\SuangaWeto,\\\Toure}


%-------------------------------------------------------------------------------
%     Contenu
%-------------------------------------------------------------------------------

\begin{document}

\ZMakeCover

\ZMakeInformations{
	% Historique des modifications
	% Version & Date & Auteur(s) & Modification(s)
	0.0 & 2014-03-03 & \Cadon & Création \\
}{
	% Historique des approbations
	% Version & Date & Approbateur(s)
	0.0 & 2014-03-03 & \Cadon \\
}{
	% Historique des validations
	% Version & Date & Responsable(s)
	0.0 & 2014-03-03 & \Cadon \\
}

\ZMakeTableOfContents

\chapter{Introduction}
Ce document présente le Plan d'Assurance Qualité (PAQ) relatif au projet de Transport Management System (TMS) par l'assistance à maîtrise d'ouvrage \amo au bénéfice de \mo.
\\
Il définit donc les méthodes, l'organisation et les éléments permettant d'assurer et de contrôler la qualité du projet.

\chapter{Structure du projet}
L'ensemble des documents produits durant le projet sera composé~:
\begin{enumerate}
	\item du PAQ~;
	\item du cahier des charges~;
	\item des comptes rendus de réunions~;
	\item des listes des tâches avec les assignations.
\end{enumerate}

\chapter{Organisation et suivi du projet}

\section{Acteurs}
\begin{table}[htbp]
	\begin{tabularx}{\linewidth}{X X}
		\toprule
		\textbf{Prénom(s) \& Nom} & \textbf{Adresse e-mail} \\
		\midrule
		\Agopian & \AgopianEmail \\
		\hline
		\Balde & \BaldeEmail \\
		\hline
		\Cadon & \CadonEmail \\
		\hline
		\Gairoard & \GairoardEmail \\
		\hline
		\Julien & \JulienEmail \\
		\hline
		\Lericolais & \LericolaisEmail \\
		\hline
		\Mezelle & \MezelleEmail \\
		\hline
		\Pachy & \PachyEmail \\
		\hline
		\SuangaWeto & \SuangaWetoEmail \\
		\hline
		\Toure & \ToureEmail \\
		\bottomrule
	\end{tabularx}
	\caption{Liste des acteurs}
	\label{Acteurs}
\end{table}

\section{Méthodologie de gestion du projet}
La méthodologie de gestion du projet de réalisation du TMS suit un cycle de développement en V, guidé par les tests et dont les principes sont~:
\begin{enumerate}
	\item tous les acteurs sont joignables par e-mail~;
	\item des réunions hebdomadaires externes sont planifiées jusqu'à la livraison finale tous les vendredis matins sur le campus universitaire de Saint-Charles à Marseille~;
	\item les présentations de l'avancement de l'application sont fréquentes et se déroulent pendant les réunions avec le représentant de la société.
\end{enumerate}
Afin de favoriser le partage des connaissances entre chaque membre du groupe, les différentes tâches seront effectuées principalement en binôme. La composition de chaque binôme variera en fonction des tâches à réaliser et des savoir-faire de chacun.
\\
Une fois qu'une tâche est terminée, si celle-ci a une suite logique, il incombe aux prédécesseurs de prévenir les successeurs.

\section{Réunions}

\subsection{Réunions internes}
Des réunions internes bihebdomadaires sont organisées afin de garantir une bonne coordination de l'équipe. Durant ces réunions tous les membres sont présents.

\subsection{Réunions externes}
Des réunions externes hebdomadaires sont organisées avec le représentant de la société afin de garantir une réactivité optimale concernant ses attentes et ses besoins.
\\
Les réunions externes donnent lieu à la rédaction d'un compte rendu contenant la liste exhaustive des tâches à réaliser, et remis au client au plus tard 48h après la fin de la réunion (dans les jours ouvrables).

\section{Outils de travail}
Afin d'éviter des problèmes d'incompatibilité, les outils suivants seront utilisés~:
\begin{table}[htbp]
	\centering
	\begin{tabularx}{\linewidth}{m{30mm} m{30mm} X}
		\toprule
		\textbf{Nom} & \textbf{Version} & \textbf{URL} \\
		\midrule
		GNU/Linux & Ubuntu 12.04 LTS & www.ubuntu.fr \\
		GNU/Linux & Debian & www.debian.org \\
		Microsoft~Windows & 7 & www.microsoft.com \\
		\bottomrule
	\end{tabularx}
	\caption{Outils de travail~: systèmes d'exploitation}
	\label{OutilsOS}
\end{table}
\begin{table}[htbp]
	\centering
	\begin{tabularx}{\linewidth}{m{30mm} m{30mm} X}
		\toprule
		\textbf{Nom} & \textbf{Version} & \textbf{URL} \\
		\midrule
		LaTeX & - & - \\
		\bottomrule
	\end{tabularx}
	\caption{Outils de travail~: documentation}
	\label{OutilsDocumentation}
\end{table}
\begin{table}[htbp]
	\centering
	\begin{tabularx}{\linewidth}{m{30mm} m{30mm} X}
		\toprule
		\textbf{Nom} & \textbf{Version} & \textbf{URL} \\
		\midrule
		Github & 1.7.9.5 & https://github.com/raviluminy/gk \\
		\bottomrule
	\end{tabularx}
	\caption{Outils de travail~: partage des fichiers}
	\label{OutilsDepot}
\end{table}

\chapter{Abréviations}
\begin{table}[htbp]
	\centering
	\begin{tabularx}{\linewidth}{m{30mm} X}
		\toprule
		\textbf{Abréviation} & \textbf{Signification} \\
		\midrule
		AMO & Assistance à Maîtrise d'Ouvrage \\
		CdC & Cahier des Charges \\
		CdCF & Cahier des Charges Fonctionnel \\
		CdCT & Cahier des Charges Technique \\
		CR & Compte-rendu \\
		Go & Giga-octet \\
		LdT & Liste des Tâches \\
		OdJ & Ordre du Jour \\
		PAQ & Plan d'Assurance Qualité \\
		SGBD & Système de Gestion de Bases de Données \\
		SGBDR & Système de Gestion de Bases de Données Relationnelles \\
		DAT & Dossier d'Architecture Technique \\
		PTI & Protocole Technique d'Installation \\
		PdR & Procédure de Recette \\
		DE & Dossier d'Exploitation \\
		PCS & Plan de Continuité des Services \\
		\bottomrule
	\end{tabularx}
	\caption{Abréviations communes}
	\label{AbreviationsCommunes}
\end{table}

\chapter{Gestion de la documentation}
Cette section décrit la manière dont sont gérés les documents relatifs au projet.

\section{Structure des documents}
Tout document produit doit se plier au modèle de document de la société.

\section{Nomenclature}
La liste exhaustive des documents concernés est décrite ci-dessous (hors sous-documents, images et librairies)~:
\begin{enumerate}
	\item \textbf{/Annexe/Docs/}
	\\
	Contient les documents relatifs à \mo, dont~:
	\begin{enumerate}
		\item GARMIR KHATCH - EXEMPLE DE CONTRAT DE TRANSPORT.doc,
		\item GARMIR KHATCH - SUIVI DES TRANSPORTS (EXEMPLE).xlsx,
		\item LOGISTICS REQUISITION.pdf,
		\item WAYBILL DELIVERY NOTE.pdf~;
	\end{enumerate}
	\item \textbf{/Annexe/RE/}
	\\
	Contient les documents relatifs aux réunions externes~:
	\begin{enumerate}
		\item RE-XXX-(YYYY-MM-DD)-CR-\textit{version}.tex,
		\item RE-XXX-(YYYY-MM-DD)-LdT-\textit{version}.tex,
		\item RE-XXX-(YYYY-MM-DD)-OdJ-\textit{version}.tex~;
	\end{enumerate}
	\item \textbf{/Annexe/RI/}
	\\
	Contient les documents relatifs aux réunions internes~:
	\begin{enumerate}
		\item RI-XXX-(YYYY-MM-DD)-CR-\textit{version}.tex,
		\item RI-XXX-(YYYY-MM-DD)-LdT-\textit{version}.tex,
		\item RI-XXX-(YYYY-MM-DD)-OdJ-\textit{version}.tex~;
	\end{enumerate}
	\item \textbf{/Conception/}
	\\
	Contient les documents relatifs à la conception~:
	\begin{enumerate}
		\item CdC-\textit{version}.tex,
		\item CdCF-\textit{version}.tex,
		\item CdCT-\textit{version}.tex,
		\item PAQ-\textit{version}.tex~;
	\end{enumerate}
	\item \textbf{/Realisation/}
	\\
	Contient les documents relatifs à la réalisation~:
	\begin{enumerate}
		\item DAT-\textit{version}.tex,
		\item PTI-\textit{version}.tex,
		\item PdR-\textit{version}.tex,
		\item DE-\textit{version}.tex,
		\item PCS-\textit{version}.tex.
	\end{enumerate}
\end{enumerate}
Dans chaque document énoncé, l'élément \textit{version} représente la version du fichier dans un format pointé. Ci-dessous quelques exemples~:
\begin{enumerate}
	\item /Annexe/RE/RE-001-(2014-02-07)-CR-2.0.tex~;
	\item /Annexe/RI/RI-003-(2014-02-21)-OdJ-1.3.tex~;
	\item /Conception/CdCF-2.1.tex~;
	\item /Realisation/DAT-0.1.tex.
\end{enumerate}

\chapter{Planification}

\begin{figure}
	\centering
	\begin{tikzpicture}
		% Déclaration des styles
		\tikzstyle{tache}=[rectangle,text=black,minimum height=9mm,minimum width=19mm]
		\tikzstyle{temps}=[rectangle,text=black,minimum height=9mm,minimum width=9mm]
		\tikzstyle{jour}=[rectangle,text=black,minimum height=9mm,minimum width=13mm,draw=black,fill=black!5]
		\tikzstyle{color1}=[draw=ZMainColor,fill=black!5]
		\tikzstyle{color2}=[draw=red,       fill=red!5]
		\tikzstyle{color3}=[draw=green,     fill=green!5]
		\tikzstyle{color4}=[draw=orange,    fill=orange!5]
		\tikzstyle{color5}=[draw=blue,      fill=blue!5]
		\tikzstyle{color6}=[draw=magenta,   fill=magenta!5]
		% Déclaration des noeuds
		% Déclaration des noeuds de temps
		\node[jour] (j1) at ( 30mm, 15mm) {\centering\tiny Vendredi};
		\node[jour] (j2) at ( 44mm, 15mm) {\centering\tiny Samedi};
		\node[jour] (j3) at ( 58mm, 15mm) {\centering\tiny Dimanche};
		\node[jour] (j4) at ( 72mm, 15mm) {\centering\tiny Lundi};
		\node[jour] (j5) at ( 86mm, 15mm) {\centering\tiny Mardi};
		\node[jour] (j6) at (100mm, 15mm) {\centering\tiny Mercredi};
		\node[jour] (j7) at (114mm, 15mm) {\centering\tiny Jeudi};
		% Déclaration des noeuds de tâches
		\node[tache,color1] (t1) at ( 0mm,- 0mm) {Tâche1};
		\node[temps,color1] (h1) at (15mm,- 0mm) {Xh};
		\node[tache,color2] (t2) at ( 0mm,-10mm) {Tâche2};
		\node[temps,color2] (h2) at (15mm,-10mm) {Xh};
		\node[tache,color3] (t3) at ( 0mm,-20mm) {Tâche3};
		\node[temps,color3] (h3) at (15mm,-20mm) {Xh};
		\node[tache,color4] (t4) at ( 0mm,-30mm) {Tâche4};
		\node[temps,color4] (h4) at (15mm,-30mm) {Xh};
		\node[tache,color5] (t5) at ( 0mm,-40mm) {Tâche5};
		\node[temps,color5] (h5) at (15mm,-40mm) {Xh};
		\node[tache,color6] (t6) at ( 0mm,-50mm) {Tâche6};
		\node[temps,color6] (h6) at (15mm,-50mm) {Xh};
		% Déclaration des lignes de tâches
		\draw[color1] (h1) -- (120mm,- 0mm);
		\draw[color2] (h2) -- (120mm,-10mm);
		\draw[color3] (h3) -- (120mm,-20mm);
		\draw[color4] (h4) -- (120mm,-30mm);
		\draw[color5] (h5) -- (120mm,-40mm);
		\draw[color6] (h6) -- (120mm,-50mm);
		% Déclaration des noeuds d'affectation
		\node[tache,color1] (a1) at (60mm,- 0mm) {Ressources1};
		\node[tache,color2] (a2) at (60mm,-10mm) {Ressources2};
		\node[tache,color3] (a3) at (60mm,-20mm) {Ressources3};
		\node[tache,color4] (a4) at (60mm,-30mm) {Ressources4};
		\node[tache,color5] (a5) at (60mm,-40mm) {Ressources5};
		\node[tache,color6] (a6) at (60mm,-50mm) {Ressources6};
	\end{tikzpicture}
	\caption{Planification}
	\label{Planification}
\end{figure}

\end{document}
