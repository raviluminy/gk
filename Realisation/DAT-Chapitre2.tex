\chapter{Architecture fonctionnelle}

% À charge à Mj de remplir cette partie
% Indications :
%  Utilisateurs, traitements, données...
% 
% Bon courage ;-)

\section{Fonctions de service et de contrainte}
Cette section précise les services fonctionnels prévus et contraintes respectées.

\subsection{La localisation}
La suite logicielle proposera des versions traduites en plusieurs langues, y compris avec des alphabets différents. Au minimum, les traductions dans les langages suivants seront fournis~:
\begin{itemize}
	\item anglais~;
	\item arabe~;
	\item espagnol~;
	\item français.
\end{itemize}
En outre, la suite logicielle fournira la possibilité de passer \emph{aisément} d'une langue à l'autre sans nécessiter de redémarrage.

\subsection{Les cas d'utilisations}
Cette section présente le diagramme des cas d'utilisation et précise pour chacun d'eux l'ensemble des actions prévues.
\begin{figure}[htbp]
	\begin{tikzpicture}
		\begin{umlsystem}[x=8.00,y=11.00,fill=red!10]{Administration}
			\umlusecase[x=0.00,y=0.00,name=GSauvegarde]{Gestion des sauvegardes}
			\umlusecase[x=0.00,y=1.00,name=GDroits]{Gestion des droits d'accès}
			\umlusecase[x=0.00,y=2.00,name=GSécurité]{Gestion des niveaux de sécurité (cryptographie)}
		\end{umlsystem}
		\begin{umlsystem}[x=8.00,y=0.00,fill=green!10]{Gestion des ressources}
			\umlusecase[x=0.00,y=0.00,name=GPrestataires]{Gestion des prestataires}
			\umlusecase[x=0.00,y=1.00,name=GTransports]{Gestion des moyens de transport}
			\umlusecase[x=0.00,y=2.00,name=GChauffeurs]{Gestion des chauffeurs}
			\umlusecase[x=0.00,y=3.00,name=GRéquisitions]{Gestion des réquisitions}
			\umlusecase[x=0.00,y=4.00,name=GWaybills]{Gestion des waybills/delivery~notes}
			\umlusecase[x=0.00,y=5.00,name=GTableauxBord]{Gestion des tableaux de bord}
			\umlusecase[x=0.00,y=6.00,name=GStatistiques]{Gestion des statistiques}
			\umlusecase[x=0.00,y=7.00,name=GSynchronisation]{Gestion de la synchronisation}
			\umlusecase[x=0.00,y=8.00,name=GPréférences]{Gestion des préférences}
		\end{umlsystem}
		\umlactor[x=0.00,y=10.00]{Administrateur}
		\umlactor[x=0.00,y=3.00]{Utilisateur}
		\umlinherit{Administrateur}{Utilisateur}
		\umlassoc{Utilisateur}{GPrestataires}
		\umlassoc{Utilisateur}{GTransports}
		\umlassoc{Utilisateur}{GChauffeurs}
		\umlassoc{Utilisateur}{GRéquisitions}
		\umlassoc{Utilisateur}{GWaybills}
		\umlassoc{Utilisateur}{GTableauxBord}
		\umlassoc{Utilisateur}{GStatistiques}
		\umlassoc{Utilisateur}{GSynchronisation}
		\umlassoc{Utilisateur}{GPréférences}
		\umlassoc{Administrateur}{GSauvegarde}
		\umlassoc{Administrateur}{GDroits}
		\umlassoc{Administrateur}{GSécurité}
	\end{tikzpicture}
	\caption{Diagramme des cas d'utilisation}
	\label{ucd}
\end{figure}

\subsubsection{Gestion des niveaux de sécurité (cryptographie)}
La suite logicielle fournira une interface permettant d'ajouter, supprimer et définir les règles de sécurités appliquées aux communications sur un lieu d'intervention, c'est à dire, les algorithmes de chiffrement utilisés lors des échanges de données sur le réseau.
\\
Initialement, les algorithmes suivants seront fournis~:
\begin{itemize}
	\item primitives d'échange de clé~:
	\begin{itemize}
		\item NULL~: i.e. aucun,
		\item RSA~: conformément à la \href{http://tools.ietf.org/html/rfc3447}{RFC 3447}~;
	\end{itemize}
	\item primitives de chiffrement~:
	\begin{itemize}
		\item NULL~: i.e. aucun,
		\item AES~: conformément à la \href{http://tools.ietf.org/html/rfc3394}{RFC 3394}~;
	\end{itemize}
	\item primitives de hachage~:
	\begin{itemize}
		\item NULL~: i.e. aucun,
		\item MD5~: conformément à la \href{http://tools.ietf.org/html/rfc1321}{RFC 1321},
		\item SHA~: conformément à la \href{http://tools.ietf.org/html/rfc3174}{RFC 3174}~;
	\end{itemize}
\end{itemize}

\subsubsection{Gestion des utilisateurs et des droits d'accès}
La suite logicielle permettra de gérer~:
\begin{enumerate}
	\item utilisateurs~;
	\item groupes d'utilisateurs.
\end{enumerate}
Un utilisateur pourra appartenir à un ou plusieurs groupes d'utilisateurs.
\\
La suite logicielle fournira à minima un groupe d'utilisateur appelé \emph{administrateur} capable de~:
\begin{enumerate}
	\item ajouter, modifier et/ou supprimer utilisateurs et groupes d'utilisateurs~;
	\item affecter des droits d'accès aux utilisateurs et groupes d'utilisateurs~;
	\item ranger les utilisateurs dans des groupes d'utilisateurs.
\end{enumerate}
Les droits d'accès possibles seront les suivants~:
\begin{enumerate}
	\item autorisé~;
	\item indéfini~;
	\item interdit~;
\end{enumerate}
Les conflits de droits résultant seront gérés par les 2 priorités suivantes~:
\begin{enumerate}
	\item les droits d'accès individuels seront prioritaires sur ceux inhérents aux groupes~;
	\item les interdictions seront prioritaires sur les autorisations~;
\end{enumerate}

\subsubsection{Gestion des sauvegardes}
La suite logicielle permettra d'effectuer des sauvegardes externalisées. Les moyens techniques mis en place à cette fin sont décris dans la section \ref{SauvegardeTechnique}.
\\
Ce système permettra les manipulations suivantes~:
\begin{enumerate}
	\item lancer une sauvegarde (aussi bien distante que locale)~;
	\item charger une sauvegarde existante (qu'elle soit distante ou locale)~;
	\item ajouter, modifier, supprimer, importer et exporter une \emph{configuration de sauvegarde}.
\end{enumerate}
Une \emph{configuration de sauvegarde} rassemblera les paramètres suivants~:
\begin{enumerate}
	\item la liste exhaustive des données sauvegardées~;
	\item la planification (date, heure, minute, ...) et ses répétitions (une seule fois, tous les jours - ou seulement certains, toutes les semaines, tous les mois, ...)~;
	\item le support de réception qu'il soit distant ou local (serveur, client, baie de disque, disque dur, disque optique, clé USB, ...).
\end{enumerate}

\subsubsection{Gestion des préférences}
La solution sera conçue de sorte à permettre l'intégration d'un module permettant aux utilisateurs de garder en mémoire leurs préférences, comme la langue utilisée. Néanmoins, ce module ne sera pas fournit, mais pourra faire l'objet d'une mise à jour ultérieure.

\subsubsection{Gestion de la synchronisation}
La suite logicielle est prévue pour fonctionner avec ou sans réseau. Afin de mettre à jour les données lorsque les moyens de communications sont rétablis, un système de synchronisation sera mis en place. Les détails techniques de cette synchronisation sont présentés en section \ref{SynchronisationTechnique}.

\subsubsection{Gestion des tableaux de bord}
La suite logicielle permettra de générer des \emph{tableaux de bord}, dont le contenu pourra être définit, chargé et enregistré dans une \emph{configuration de tableau de bord}.

\subsubsection{Gestion des statistiques}
La suite logicielle sera conçue de sorte à permettre l'intégration d'un module capable de produire un ensemble de statistiques à partir des informations enregistrées. Néanmoins, ce module ne sera pas fournit par la solution proposée, mais pourra faire l'objet d'une mise à jour ultérieure.

\subsubsection{Gestion des données}
La suite logicielle permettra de numériser, des documents papier. Ces derniers pourront être ensuite enregistrés, affichés, supprimés, importés, exportés et archivés.
\\
En outre, la suite logicielle permettra d'ajouter, modifier, et supprimer les informations qu'elle aura à traiter, et les liens qui les relies. Initialement, les informations suivantes seront déjà définies et pourront être traitées~:
\begin{enumerate}
	\item planifications~;
	\item réquisitions~;
	\item waybills / delivery notes~;
	\item prestataires~;
	\item chauffeurs~;
	\item véhicules.
\end{enumerate}

\paragraph{Planifications}
Une planification rassemble les informations suivantes~:
\begin{enumerate}
	\item planifications~;
	\item réquisitions~;
	\item waybills / delivery notes~;
	\item prestataires~;
	\item chauffeurs~;
	\item véhicules.
\end{enumerate}

\paragraph{Réquisitions}
Une Réquisition rassemble les informations suivantes~:
\begin{enumerate}
	\item code du pays, et numéro unique de réquisition~;
	\item origine, destination, date de la réquisition, date de livraison souhaitée~;
	\item moyen de transport (air, mer, route, ...)~;
	\item pour chaque article~:
	\begin{enumerate}
		\item code de l'article,
		\item numéro de compte,
		\item description de l'article,
		\item quantité,
		\item unité~;
		\item limite budgétaire (prix unitaire, prix total)~;
	\end{enumerate}
	\item les accords~:
	\begin{enumerate}
		\item nom, et date de signature du demandeur,
		\item nom, et date de signature du  chef de projet (en charge du budget),
		\item nom, et date de signature de l'agent financier,
		\item nom, et date de signature de la logistique~;
	\end{enumerate}
	\item les détails de l'envoi~:
	\begin{enumerate}
		\item adresse du consignataire (adresse complète, téléphone, mobile, fax, nom du contact & adresse e-mail),
		\item adresse de livraison (si différente - adresse complète, téléphone, mobile, fax, nom du contact & adresse e-mail),
		\item agent de dédouanement (le cas échéant - adresse complète, tel, mobile, fax, contact & email),
		\item marquage de l'envoi (ex: no. CTN, logo Croix-Rouge, "en transit", "donation de..."),
		\item demandes spéciales ou remarques (ex: instructions d'emballage, incoterms, documents joints, etc.)~;
	\end{enumerate}
\end{enumerate}

\paragraph{Waybills / delivery notes}

\paragraph{Prestataires}

\paragraph{Chauffeurs}

\paragraph{Véhicules}

\subsection{Contraintes}
Cette section tient compte des contraintes respectées jusque là non évoquées.

\begin{constraint}[Contraintes de délai]
\amo s'engage à fournir des garanties contractuelles de délais, et à s'y tenir. Les éventuelles pénalités de retard seront fixées d'un accord commun avec \mo.
\end{constraint}
