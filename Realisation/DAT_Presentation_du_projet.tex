\chapter{Présentation du projet}\label{Presentation_du_projet}

\section{Introduction}
La présentation du projet s'articule autour de plusieurs points~: 
\begin{itemize}
   \item le calendrier du projet~;
   \item son objectif~;
   \item ses enjeux~;
   \item ses exigences.
\end{itemize}

\section{Calendrier du projet}
Le calendrier du projet permet de suivre l'évolution du projet.
C'est dans celui-ci qu'il faut indiquer les différentes phases du projet et les deadlines associées à ces phases.

\subsection{Phase de réalisation du projet}
Cette phase consiste à répondre au projet décrit par le cahier des charges.
Les documents suivants en sont issus~:
\begin{itemize}
	\item code source~;
	\item documentation d'architecture (DAT, PTI, ...) et d'exploitation~;
	\item droits de propriété.
\end{itemize}

\section{Objectif du projet}
L'objectif est de fournir la solution logicielle de type Transport Management System (TMS) attendue, et sa documentation associée.
 
\section{Enjeux du projet}
\mo espère améliorer la qualité de ses services grâce à l'assistance du TMS et de l'automatisation de certaines de ses tâches.
Le TMS lui permettra non seulement de réduire les coûts relatifs à la logistique mais également de maintenir son professionnalisme, améliorant ainsi son image de marque par preuve de son efficacité et de son efficience sur le terrain.

\section{Exigences du projet}
Cette section précise les services attendus et les exigences du projet.

\subsection{La localisation}
La suite logicielle doit proposer des versions traduites en plusieurs langues avec des alphabets et des sens de lecture différents.
À minima, les traductions dans les langues suivantes sont fourni~:
\begin{enumerate}
	\item anglais~;
	\item français~;
	\item espagnol~;
	\item arabe.
\end{enumerate}

\subsection{L'interface utilisateur}
La solution fournit au minimum, et pour chacun des groupes utilisateurs ci dessous, une interface propre permettant de réaliser les actions qui leur sont liées.

\subsection{Les cas d'utilisations}
Le TMS devra permettre~:
\begin{enumerate}
	\item la gestion des niveaux de sécurité (cryptographie)~;
	\item la gestion des droits d'accès~;
	\item la gestion des sauvegardes et des préférences~; 
	\item la gestion de la synchronisation~;
	\item la gestion des tableaux de bord~;
	\item la gestion des statistiques~;
	\item la gestion des réquisitions \& waybills/delivery~notes~;
	\item la gestion des chauffeurs~;
	\item la gestion des véhicules~;
	\item la gestion des prestataires.
\end{enumerate}

\subsection{Les contraintes}
Le présent document tiens compte des contraintes suivantes~:
\begin{constraint}[Contrainte de compatibilité]
	La suite logicielle doit impérativement être compatible avec les matériels utilisés, les systèmes d'exploitation et, au minimum, les versions de logiciels.
\end{constraint}
\begin{constraint}[Contrainte de coûts et de moyens de communications]
	Il faut limiter les coûts et se plier au préférences sur les moyens de communications définies dans le cahier des charges.
\end{constraint}
\begin{constraint}[Contrainte de fonctionnalité]
	La suite logicielle fournit un outil intégré permettant de numériser les documents, via une interface graphique claire permettant de fixer facilement les paramètres de numérisation.
\end{constraint}
\begin{constraint}[Contrainte de délai]
	\amo s'engage à fournir des garanties contractuelles de délais, et à s'y tenir.
	Les éventuelles pénalités de retard sont fixées d'un accord commun avec \mo.
\end{constraint}
\begin{constraint}[Contrainte légale]
	Les contraintes légales de chaque pays s'appliquant à \mo lors de ses interventions, la solution peut s'adapter aux normes en vigueur.
\end{constraint}
\begin{constraint}[Contrainte réglementaire]
	\mo a une image et une éthique mondialement connue découlant de ses activités.
	Il conviendra de la prendre en compte lors de la réalisation de la solution.
\end{constraint}
