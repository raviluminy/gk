\chapter{Présentation du projet}

% À charge à Ahoua Khady de remplir cette partie
% Indications :
%  Calendrier,objectif, enjeux et exigences 
% 
% Bon courage ;-)

\section{Introduction}
La présentation du projet s'articule autour de plusieurs points qui sont ~: 
\begin{itemize}
   \item le calendrier du projet ~;
   \item son objectif ~;
   \item ses enjeux ~;
   \item ses exigences ~;
\end{itemize}

\section{le calendrier du projet}
Le calendrier du projet permet de suivre l'évolution du projet. c'est dans celui ci qu'il faut indiquer les différentes phases du projet et les deadlines associées à ces phases.

\subsection{la phase de conception du projet}
Dans cette phase, il s'agissait de la rédaction du cahier des charges. Elle s'étendait sur la période du 07 février au 2 mars 2014. La version finale du cahier des charges a été livrée le dimanche 2 mars 2014 , cependant elle peut être soumise à d'éventuelles modifications. 

\subsection{La phase de réalisation du projet}
Cette phase consiste en la rédaction du cadre de réponse au cahier des charges. A l'issu de cette phase, il faudrait fournir entre autres documents:
  \begin{itemize}
    \item le code source~;
    \item la documentation d'exploitation et d'architecture(DAT, PTI ...)~;
    \item les droits de propriété~;
  \end{itemize}

L'étape de réalisation a commencé le lundi 3 mars 2014 et la deadline est encore à définir.

\section{L'objectif du projet}
 L'objectif de la réalisation du projet est  de fournir d'une part, à court terme une version finale du dossier d'architecture technique(DAT), les documents d'exploitation et d'autre part, à long terme la solution logicielle de type Transport Management System(TMS) attendue.
 
 \section{Les enjeux du projet}
\mo espère améliorer la qualité de ses services Grâce au TMS, et le fonctionnement global de l'organisation grâce à l'assistance de l'outil et l'automatisation de certaines de ces tâches. Le TMS lui permettra non seulement de réduire les coûts relatifs à la logistique mais également de maintenir son professionnalisme, améliorant ainsi son image de marque par preuve de son efficacité et de son efficience sur le terrain.

\section{Les exigences du projet}
cette section précise les services attendus et les exigences du projet.

\subsection{La localisation}
La suite logicielle doit proposer des versions traduites en plusieurs langues(Anglais, Français, Espagnol, Arabe) avec des alphabets et des sens de lecture différents.

\subsection{L'interface utilisateur}
La solution doit fournir au minimum et pour chacun de ces groupes utilisateurs une interface propre permettant de réaliser les actions liées à chacun d'eux.

\subsection{Les cas d'utilisations}
Le TMS devra permettre :
\begin{itemize}
\item la gestion des niveaux de sécurité(cryptographie)~;
\item la gestion des droits d'accès~;
\item la gestion des sauvegardes et des préférences~; 
\item la gestion de la synchronisation~;
\item la gestion des tableaux de bord~;
\item la gestion des statistiques, des réquisitions \& Waybills/ Delivery notes~;
\item la gestion des chauffeurs et des prestataires.
\end{itemize}

\subsection{Les contraintes}
Il y a plusieurs contraintes définis par \mo qui se doivent d'être respecter:
\begin{itemize}
\item \textbf{la contrainte de compatibilité~:} La suite logicielle doit impérativement être compatible avec les matériels utilisés, les systèmes d'exploitation et, au minimum, les versions de logiciels.

\item \textbf{la contrainte sur les coûts et les moyens de communications~:} il faut limiter les coûts et se plier au préférences sur les moyens de communications définies dans le cahier des charges.

\item \textbf{la contrainte de fonctionnalité~:} La suite logicielle doit fournir un outil intégré permettant de numériser les documents qui doivent l'être, via une interface graphique clair permettant de fixer facilement les paramètres de numérisation.

\item \textbf{la contrainte de délai~:} Le prestataire s'engage à fournir des garanties contractuelles de délais, et à s'y tenir. Les éventuelles pénalités de retard seront fixées d'un accord commun avec \mo.

\item \textbf{la contrainte légale~:} Les contraintes légales de chaque pays s'appliquant à \mo lors de ses interventions, la solution devra être aux normes (ou pouvoir s'y adapter) de ces pays.

\item \textbf{la contrainte réglementaire~:} \mo a une image et une éthique mondialement connue découlant de ses activités. Il conviendra de la prendre en compte lors de la réalisation de la solution.

\end{itemize}

