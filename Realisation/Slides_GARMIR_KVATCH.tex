% **************************** %
%          Slides GK           %
% **************************** %

\documentclass[10pt,fleqn]{beamer}
 
% **************************** %
%            Package           %
% **************************** %
% \usepackage{GarmirKhatch}
\usepackage[utf8]{inputenc}
\usepackage[T1]{fontenc}
\usepackage[frenchb]{babel}
\usepackage[french]{layout}
\usepackage{lmodern}
\usepackage{ragged2e}
\usepackage{fancyhdr}
\usepackage{verbatim}
\usepackage{graphicx}
\usepackage{wrapfig}
\usepackage{url}
\usepackage{hyperref}
\usepackage{amsmath}
\usepackage{multirow}
\usepackage{multicol}
\usepackage{array}
\usepackage{colortbl}
\usepackage{comment}
\usepackage{tikz}
\usetikzlibrary{positioning, shadows}

% **************************** %
%          Préambule           %
% **************************** %
% Option pdf
\hypersetup{
      %pdfpagemode = FullScreen,
      pdfauthor   = {AMUFSI2014},
      pdftitle    = {Slides_GARMIR_KVATCH},
      pdfsubject  = {Système de gestion des transports},
      pdfkeywords = {AMO, GK, TMS, CdC, DAT, PTI}
}
% Thème du pdf 
\usetheme{Warsaw}
% Logo de l'université d'Aix-Marseille
%\logo{\includegraphics[height=6mm]{logo}}
% Affiche les notes
%\setbeameroption{show notes}
% Blocks arrondies et ombrés
\setbeamertemplate{blocks}[rounded][shadow=true] 
% Balle pour la liste d'items
\setbeamertemplate{itemize item}[ball]
% Triangle pour la liste de sous items
\setbeamertemplate{itemize subitem}[triangle]
% Affiche l'ensemble du frame en gris clair
\beamertemplatetransparentcovered
% Faire apparaître un sommaire avant chaque section
\AtBeginSection[]{
\begin{frame}
\frametitle{Sommaire}
\tableofcontents[currentsection, hideallsubsections]
\end{frame}
}

% **************************** %
%        Page de garde         %
% **************************** %
\title[Analyse statique]{{\Large \textsc{Garmir Kvatch \\Système de gestion des transports}}}
\author[\textsc{Garmir Kvatch - Système de gestion des transports}]{M2 FSIL - FSI}
\institute{Encadrant : M. Roland \textsc{Agopian}\\
Faculté des Sciences d'Aix-Marseille Université\\
Campus de Luminy}
\date{\scriptsize{ 27 mars 2014}}

% **************************** %
%       Corps du document      %
% **************************** %
\begin{document}
 
% **************************** %
%            Entête            %
% **************************** %
\begin{frame}
\begin{figure}
\centering
\includegraphics[scale=0.52]{Images/EnTeteSciences}
\end{figure}
\titlepage
\end{frame}

\begin{frame}
\frametitle{Sommaire}
\tableofcontents[hideallsubsections]
\end{frame}

\section[Introduction]{Introduction}
\subsection[Définition de l'analyse statique]{Définition de l'analyse statique}
\begin{frame}
\transdissolve[duration=0.2]<1->
\frametitle{Définition}

\end{frame}

\subsection[Objet de la consultation]{Objet de la demande de prestation d'AMO}
\begin{frame}
\frametitle{Objet de la consultation}
L'objectif de ce projet de fin d'étude est de sélectionner un prestataire dans le cadre d’une mission d’assistance à maîtrise d’ouvrage informatique pour le développement d’une architecture logicielle pour la gestion des transports de marchandise en situation d’urgence.
\end{frame}

\section[Présentation de Garmir Kvatch]{Présentation de Garmir Kvatch}
\begin{frame}
\frametitle{}
\begin{block}{\textbf{Qu'est-ce que Garmir Kvatch ?}}
\begin{itemize}
\item Organisation humanitaires (187 Sociétés Nationales)
\item Vient en aide aux personnes plus vulnérables
\item Fort réseau de volontaires
\item Rend les communautés plus résistantes
\end{itemize}
\end{block}
\end{frame}

\section[]{}
\subsection[]{ }
\begin{frame}
\frametitle{}
\end{frame}

\subsection[]{ }
\begin{frame}
\frametitle{}
\end{frame}

\section[Conclusion]{Conclusion}
\begin{frame}
\transdissolve[duration=0.2]<1->
\frametitle{Conclusion}
\end{frame}

\begin{frame}
\transdissolve[duration=0.2]<1->
\frametitle{Merci de votre attention}
\end{frame}

\end{document}