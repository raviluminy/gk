\section{Données}\label{DonneesTechnique}

% Indications :
%  Stockage (coté client, coté serveur, SGBD, format, ...)
%  Transfert (format, norme, ...)
%  Import/Export...
% 
% Bon courage ;-)

\subsection{Problématique}
Ce chapitre traite du stockage et du formatage des données sur les différents postes, et dans les différentes situations d'utilisation précédemment décrites.
Cette problématique est divisée selon trois points clés~:
\begin{itemize}
	\item le stockage~;
	\item le transfert~;
	\item l'importation et l'exportation.
\end{itemize}

\subsection{Le stockage}
Cette section développe en détail les différents type de stockage au sein de l'architecture clients/serveurs.

\subsubsection{Côté serveur}
%% Mj %% Je ne suis pas vraiment en accord avec le choix de cette solution, cf. CdCF. Il est précisé que \mo dispose déjà de SGBDR opérationnels et en service que sont MySQL et SQL~Oracle dont les licenses sont déjà possédées. En outre, cela implique que \mo dispose des connaisances et des compétences internes en relation avec ces 2 SGBDR...
Le stockage des données côté serveur se fait grâce au SGDBR PostgreSQL. Cette solution a été retenue pour plusieurs raisons~:
\begin{itemize}
	\item solution libre et gratuite (sans coût de licence)~;
	\item solution compatible avec la plupart des systèmes d'exploitation (Windows, GNU/Linux, MacOS, ...)~;
	\item solution conforme aux normes qui régissent le langage SQL, ce qui garantit une bonne portabilité en cas de migration sur une autre solution~;
	\item solution pouvant gérer de très gros volumes de données à l'aide de son optimiseur très performant (\~ 13To)~;
	\item solution de prédilection de notre entreprise, ce qui garantit une interopérabilité maximale.
\end{itemize}

\subsubsection{Côté client}
%% Mj %% Pas très clair. Qu'est-ce qui est effectivement mis en place côté client~? SGBDR et/ou XML~?
Le stockage des données côté client se fait soit via une base de données locale (également sous PostgreSQL), s'il est possible de la synchroniser avec la base de donnée centrale et que le périphérique le permet, soit via des fichiers au format XML.
Le choix de l'utilisation des fichiers XML est dû aux raisons suivantes~:
\begin{itemize}
	\item format universel, compatible à l'import/export avec la plupart des SGDBR (dont PostgreSQL)~;
	\item taux de compression assez important sur ce format, ce qui peut favoriser le transfert des données dans des situations où la connexion est limitée~;
	\item exploitable sur les périphériques Android.
\end{itemize}

\subsubsection{Structuration des données}
\begin{figure}[htbp]
	\centering
	\begin{tikzpicture}
		\begin{umlpackage}[x=0,y=0]{Requisition}
			\umlclass[x=0,y=0,width=60mm]{Requisition}{
				% Identifiants
				\ZPrimaryKey{CountryCode}			: VARCHAR	\\ % Code du pays (du lieu d'intervention)
				\ZPrimaryKey{Number(Id)}			: VARCHAR	\\ % Numéro unique de réquisition
				% Estimations~?
				\ZNoneKey	{ForCostEstimate}		: INTEGER	\\ % Demande de devis
				\ZNoneKey	{ForPurchase}			: INTEGER	\\ % Demande d'achat
				\ZNoneKey	{WhDispatchRelease}		: INTEGER	\\ % Mouvement de stock
				% Informations générales
				\ZNoneKey	{To}					: VARCHAR	\\ % Destination
				\ZNoneKey	{From}					: VARCHAR	\\ % Origine
				\ZNoneKey	{Date}					: DATE		\\ % Date de demande
				\ZNoneKey	{DesiredDeliveryDate}	: DATE		\\ % Date de livraison souhaitée
				% Codes budgétaires et devises
				\ZNoneKey	{Project}				: INTEGER	\\ % Demande d'achat
				\ZNoneKey	{Activity}				: INTEGER	\\ % Demande d'achat
				\ZNoneKey	{MCode}					: INTEGER	\\ % Demande d'achat
				\ZNoneKey	{Currency}				: INTEGER	\\ % Demande d'achat
				% Moyen(s) de transport
				\ZNoneKey	{TransportMeans}		: [AIR,SEA,ROAD,RAIL]	\\ % Type de transport (aérien, maritime, terrestre, ...)
				% Cargaison/marchandises transportées
				\ZForeignKey{Cargo}					: REFERENCE	\\ % Cargaison (possiblement complexe)
				\ZNoneKey	{CargoTotalValue}		: INTEGER	\\ % Valeur totale de la cargaison
				% Accords
				\ZForeignKey{Requester}				: REFERENCE	\\ % Demandeur
				\ZNoneKey	{RequesterDate}			: DATE		\\ % Date d'accord du demandeur
				\ZForeignKey{ProjectManager}		: REFERENCE	\\ % Chef de projet (en charge du budget)
				\ZNoneKey	{ProjectManagerDate}	: DATE		\\ % Date d'accord du chef de projet
				\ZForeignKey{FinanceOfficer}		: REFERENCE	\\ % Agent financier (seulement pour les demandes d'achat)
				\ZNoneKey	{FinanceOfficerDate}	: DATE		\\ % Date d'accord de l'agent financier (seulement pour les demandes d'achat)
				\ZForeignKey{Logistics}				: REFERENCE	\\ % Responsable de la logistique (seulement pour les mouvements d'entrepôt)
				\ZNoneKey	{LogisticsDate}			: DATE		\\ % Date d'accord du responsable de la logistique (seulement pour les mouvements d'entrepôt)
				\ZForeignKey{GlobalFleetBase}		: REFERENCE	\\ % Base générale de la flotte de véhicules (seulement pour les demandes de véhicules)
				\ZNoneKey	{GlobalFleetBaseDate}	: DATE		\\ % Date d'accord de la base générale de la flotte de véhicules (seulement pour les demandes de véhicules)
				% Détails de l'envoi
				\ZForeignKey{Consignee}				: REFERENCE	\\ % Consignataire
				\ZForeignKey{Delivery}				: REFERENCE	\\ % Livraison
				\ZForeignKey{ClearingAgent}			: REFERENCE	\\ % Agent de dédouanement
				\ZForeignKey{ShippingMarks}			: REFERENCE	\\ % Marquage de l'envoi
				\ZNoneKey	{SpecialRequirementsOrRemarks}	: VARCHAR	% Demande spéciales ou remarques
			}{}
			\umlclass[x=70mm,y=0,width=60mm]{Cargo}{
				\ZPrimaryKey{Number(Id)}			: INTEGER	\\ % Identifiant
				\ZForeignKey{ItemCode}				: REFERENCE	\\ % Code de l'article
				\ZNoneKey	{Quantity}				: INTEGER	\\ % Quantité
				\ZNoneKey	{UoM}					: INTEGER	\\ % Unité
				% Limite budgétaire
				\ZNoneKey	{UnitPrice}				: INTEGER	\\ % Prix unitaire
				\ZNoneKey	{TotalPrice}			: INTEGER	% Prix total
			}{}
			\umlclass[x=70mm,y=50mm,width=60mm]{Item}{
				\ZPrimaryKey{ItemCode(Id)}			: INTEGER	\\ % Code de l'article
				\ZNoneKey	{Account}				: VARCHAR	\\ % Numéro du compte
				\ZNoneKey	{CommodityTrackingOrDonor}	: INTEGER	\\ % Unité
				\ZNoneKey	{ItemDescription}		: VARCHAR	% Description de l'article
				\ZNoneKey	{Weight}					: INTEGER	\\ % Unité
				\ZNoneKey	{Volume}				: INTEGER	\\ % Prix unitaire
				
			}{}
			%\umlclass[x=70mm,y=-50mm,width=60mm]{Person}{
			%}{}
			\umlclass[[x=70mm,y=-50mm,width=60mm]{Waybill}{
				\ZPrimaryKey{CountryCode}			: INTEGER	\\ % Code de l'article
				\ZPrimaryKey{Number(Id)}			: VARCHAR	\\ % Numéro du compte
				\ZNoneKey	{Date}				: VARCHAR	\\ % Numéro du compte
				\ZNoneKey	{Warehouse}				: VARCHAR	\\ % Numéro du compte
				\ZNoneKey	{Destination}				: VARCHAR	\\ % Numéro du compte
				\ZNoneKey	{Beneficiary}				: VARCHAR	\\ % Numéro du compte
				% Transport Data
				\ZNoneKey	{ContractNumber}				: VARCHAR	\\ % Numéro du compte
				\ZNoneKey	{Vehicle}				: VARCHAR	\\ % Numéro du véhicule (avec type de transport)
				\ZNoneKey	{RegistrationNumber}				: VARCHAR	\\ % Numéro du compte
				\ZNoneKey	{ETD}				: VARCHAR	\\ % Numéro du compte
				\ZNoneKey	{Vehicle}				: VARCHAR	\\ % Numéro du compte
				\ZNoneKey	{Vehicle}				: VARCHAR	\\ % Numéro du compte
				\ZNoneKey	{Vehicle}				: VARCHAR	\\ % Numéro du compte
				\ZForeignKey{Cargo}				: REFERENCE	\\ % Code de l'article
				\ZNoneKey	{Comments}				: VARCHAR	\\ % Numéro du compte
				\ZNoneKey	{CommentsFromReceiver}				: VARCHAR	\\ % Numéro du compte
%				\ZNoneKey	{UnitTotal}				: VARCHAR	\\ % Numéro du compte
%				\ZNoneKey	{WeightTotal}			: VARCHAR	\\ % Numéro du compte
				\ZNoneKey	{}					: INTEGER	\\ % Unité
				% Informations de chargement
				\ZForeignKey	{LoadedBy}				: REFERENCE	\\ % Prix unitaire
				\ZForeignKey	{TransportedBy}			: REFERENCE	\\ % Prix unitaire
				\ZNoneKey	{LoadingDate}				: INTEGER	\\ % Prix unitaire
%				\ZNoneKey	{TransportingLocation}				: INTEGER	\\ % Prix unitaire
				\ZNoneKey	{TransportingDate}				: INTEGER	\\ % Prix unitaire
				\ZNoneKey	{LoadingLocation}				: INTEGER	\\ % Prix unitaire
				\ZNoneKey	{LoadedCondition}				: VARCHAR	\\ % Prix unitaire
				\ZNoneKey	{TransportesCondition}				: VARCHAR	\\ % Prix unitaire
				% Informations de reception
				\ZForeignKey	{Receiver}				: REFERENCE	\\ % Prix unitaire
				\ZNoneKey	{ReceptionDate}				: INTEGER	\\ % Prix unitaire
				\ZNoneKey	{ReceptionLocation}				: INTEGER	\\ % Prix unitaire
				\ZNoneKey	{ReceptionCondition}				: VARCHAR	\\ % Prix unitaire
			}{}
			\umlclass[x=70mm,y=0,width=60mm]{Cargo}{
				\ZPrimaryKey{Number(Id)}			: INTEGER	\\ % Identifiant
				\ZForeignKey{ItemCode}				: REFERENCE	\\ % Code de l'article
				\ZNoneKey	{Quantity}				: INTEGER	\\ % Quantité
%				\ZNoneKey	{UnitTypeWeight}				: INTEGER	\\ % Quantité
				\ZForeignKey{Requisition}		: REFERENCE	\\ % Code de l'article
				\ZNoneKey	{Remarks}			: VARCHAR	% Prix total
%				\ZNoneKey	{Weight}					: INTEGER	\\ % Poids total
%				\ZNoneKey	{Volume}				: INTEGER	\\ % Volume total
			}{}
			\umlclass[x=70mm,y=50mm,width=60mm]{Transporter}{
				\ZPrimaryKey{ItemCode(Id)}			: INTEGER	\\ % Code de l'article
				\ZNoneKey	{Lastname}				: VARCHAR	\\ % Numéro du compte
				\ZNoneKey	{Firstname}	: INTEGER	\\ % Unité
				\ZNoneKey	{Birthday}		: VARCHAR	% Description de l'article
				\ZNoneKey	{NumberID}					: INTEGER	\\ % Unité
				\ZForeignKey	{DrivingLicence}		: REFERENCE	\\ % Unité
			}{}
			\umlclass[x=70mm,y=50mm,width=60mm]{Vehicle}{
				\ZPrimaryKey{ItemCode(Id)}			: INTEGER	\\ % Code de l'article
				\ZNoneKey	{Account}				: VARCHAR	\\ % Numéro du compte
				\ZNoneKey	{CommodityTrackingOrDonor}	: INTEGER	\\ % Unité
				\ZNoneKey	{ItemDescription}		: VARCHAR	% Description de l'article
				\ZNoneKey	{Weight}					: INTEGER	\\ % Unité
				\ZNoneKey	{Volume}				: INTEGER	\\ % Prix unitaire
				
			}{}
			\umlclass[x=70mm,y=50mm,width=60mm]{Provider}{
				\ZPrimaryKey{ItemCode(Id)}			: INTEGER	\\ % Code de l'article
				\ZNoneKey	{Account}				: VARCHAR	\\ % Numéro du compte
				\ZNoneKey	{CommodityTrackingOrDonor}	: INTEGER	\\ % Unité
				\ZNoneKey	{ItemDescription}		: VARCHAR	% Description de l'article
				\ZNoneKey	{Weight}					: INTEGER	\\ % Unité
				\ZNoneKey	{Volume}				: INTEGER	\\ % Prix unitaire
				
			}{}
			\umlclass[x=70mm,y=50mm,width=60mm]{DrivingLicence}{
				\ZPrimaryKey{ItemCode(Id)}			: INTEGER	\\ % Code de l'article
				\ZNoneKey	{Category}				: VARCHAR	\\ % Numéro du compte
				\ZNoneKey	{ValidityArea}	: VARCHAR	\\ % National ou international
			}{}
		\end{umlpackage}
	\end{tikzpicture}
	\caption{Structuration des données dans le SGBDR}
	\label{DonneesStructurationSgbdr}
\end{figure}


\subsection{Le transfert}
Le transfert des données quant à lui, est réalisé selon un format et une norme qui est préalablement défini.

\subsubsection{Format}
Eu égard des solutions retenues pour le transfert des données, le transfert des données se fera grâce à des fichiers XML.
La possibilité sera donnée de compresser ces fichiers, dans un premier temps au format \emph{ZIP}, mais ce choix peut être remis en question selon les observations réalisées sur les données de test, afin de retenir une éventuelle meilleure solution.

\subsubsection{norme}
% Définir la norme du format XML correspondant à une table SQL
% avec une DTD associée.
% 
% Exemple de DTD pour une personne :
% 
% <!ELEMENT personne (nom,prenom,telephone),email? >
% <!ELEMENT nom (#PCDATA) >
% <!ELEMENT prenom (#PCDATA) >
% <!ELEMENT telephone (#PCDATA) >
% <!ELEMENT email (#PCDATA) >

\subsection{L'import/export de données}
L'import et l'export de données se fait directement via une fonctionnalité de l'outil côté client, qui permet de sélectionner les données à importer dans le contexte spécifique, ainsi que les données à exporter vers le serveur central, toujours selon ce contexte.
Comme décrit dans la partie traitant de ce sujet, l'import/export de données est réalisé selon différents supports (réseau, support physique) à la discrétion de l'exploitant.
