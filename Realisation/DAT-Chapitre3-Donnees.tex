\section{Données}\label{DonneesTechnique}

% Indications :
%  Stockage (coté client, coté serveur, SGBD, format, ...)
%  Transfert (format, norme, ...)
%  Import/Export...
% 
% Bon courage ;-)

\subsection{Problématique}
Ce chapitre traite du stockage et du formatage des données sur les différents postes, et dans les différentes situations d'utilisation précédemment décrites. \\
Cette problématique est divisée selon trois points clés~:
\begin{itemize}
	\item le stockage~;
	\item le transfert~;
	\item l'importation et l'exportation.
\end{itemize}

\subsection{Le stockage}
Cette section développe en détail les différents type de stockage au sein de l'architecture clients/serveurs.

\subsubsection{Côté serveur}
Le stockage des données côté serveur se fait grâce au SGDBR MySQL. Cette solution a été retenue pour plusieurs raisons~:
\begin{itemize}
	\item solution traitant rapidement les données~;
	\item solution compatible avec la plupart des systèmes d'exploitation (Windows, GNU/Linux, MacOS, ...)~;
	\item solution facile à utiliser~;
	\item solution pouvant facilement s'interfacer à l'aide d'API diverses~;
	\item solution de prédilection de notre entreprise, ce qui garantit une interopérabilité maximale.
\end{itemize}

\subsubsection{Côté client}
Le stockage des données côté client se fait soit via deux représentations d'une base de données locale : une tournant sur MySQL et un autre utilisant des fichiers au format XML. s'il est possible d'effectuer la synchronisation avec la base de donnée centrale et que le périphérique le permet, la première représentation est choisi par défaut. Si la synchronisation au serveur central est indisponible, le stockage des données est fait par la deuxième représentation de la base au format XML. \\
Le choix de l'utilisation des fichiers XML est dû aux raisons suivantes~:
\begin{itemize}
	\item format universel, compatible à l'import/export avec la plupart des SGDBR (dont MySQL)~;
	\item taux de compression assez important sur ce format, ce qui peut favoriser le transfert des données dans des situations où la connexion est limitée~;
	\item exploitable sur les périphériques Android.
\end{itemize}

\subsubsection{Structuration des données}
\begin{figure}[htbp]
	\centering
	\begin{tikzpicture}
		\begin{umlpackage}[x=0,y=0]{Requisition}
			\umlclass[x=0mm,y=0mm,width=60mm]{Requisition}{
				% Identifiants
				\ZPrimaryKey{CountryCode}			: STRING	\\ % Code du pays (du lieu d'intervention)
				\ZPrimaryKey{Number(Id)}			: STRING	\\ % Numéro unique de réquisition
				% Estimations~?
				\ZNoneKey	{ForCostEstimate}		: NUMBER	\\ % Demande de devis
				\ZNoneKey	{ForPurchase}			: NUMBER	\\ % Demande d'achat
				\ZNoneKey	{WhDispatchRelease}		: NUMBER	\\ % Mouvement de stock
				% Informations générales
				\ZNoneKey	{To}					: STRING	\\ % Destination
				\ZNoneKey	{From}					: STRING	\\ % Origine
				\ZNoneKey	{Date}					: DATE		\\ % Date de demande
				\ZNoneKey	{DesiredDeliveryDate}	: DATE		\\ % Date de livraison souhaitée
				% Codes budgétaires et devises
				\ZNoneKey	{Project}				: NUMBER	\\ % ?
				\ZNoneKey	{Activity}				: NUMBER	\\ % ?
				\ZNoneKey	{MCode}					: NUMBER	\\ % ?
				\ZNoneKey	{Currency}				: NUMBER	\\ % ?
				% Moyen(s) de transport
				\ZForeignKey{TransportMeans}		: ?			\\ % Type de transport (aérien, maritime, terrestre, ...)
				% Cargaison/marchandises transportées
				\ZForeignKey{Cargo}					: ?			\\ % Cargaison (possiblement complexe)
				\ZNoneKey	{CargoTotalValue}		: NUMBER	\\ % Valeur totale de la cargaison
				% Accords
				\ZForeignKey{Requester}				: ?			\\ % Demandeur
				\ZNoneKey	{RequesterDate}			: DATE		\\ % Date d'accord du demandeur
				\ZForeignKey{ProjectManager}		: ?			\\ % Chef de projet (en charge du budget)
				\ZNoneKey	{ProjectManagerDate}	: DATE		\\ % Date d'accord du chef de projet
				\ZForeignKey{FinanceOfficer}		: ?			\\ % Agent financier (seulement pour les demandes d'achat)
				\ZNoneKey	{FinanceOfficerDate}	: DATE		\\ % Date d'accord de l'agent financier (seulement pour les demandes d'achat)
				\ZForeignKey{Logistics}				: ?			\\ % Responsable de la logistique (seulement pour les mouvements d'entrepôt)
				\ZNoneKey	{LogisticsDate}			: DATE		\\ % Date d'accord du responsable de la logistique (seulement pour les mouvements d'entrepôt)
				\ZForeignKey{GlobalFleetBase}		: ?			\\ % Base générale de la flotte de véhicules (seulement pour les demandes de véhicules)
				\ZNoneKey	{GlobalFleetBaseDate}	: DATE		\\ % Date d'accord de la base générale de la flotte de véhicules (seulement pour les demandes de véhicules)
				% Détails de l'envoi
				\ZForeignKey{Consignee}						: ?			\\	% Consignataire
				\ZForeignKey{Delivery}						: ?			\\	% Livraison
				\ZForeignKey{ClearingAgent}					: ?			\\	% Agent de dédouanement
				\ZForeignKey{ShippingMarks}					: ?			\\	% Marquage de l'envoi
				\ZNoneKey	{SpecialRequirementsOrRemarks}	: STRING		% Demande spéciales ou remarques
			}{}
			\umlclass[x=70mm,y=0mm,width=60mm]{Cargo}{
				\ZPrimaryKey{Number(Id)}				: NUMBER	\\	% Identifiant
				\ZForeignKey{ItemCode}					: ?			\\	% Code de l'article
				\ZNoneKey	{Quantity}					: NUMBER	\\	% Quantité
				\ZNoneKey	{UoM}						: NUMBER	\\	% Unité
				% Limite budgétaire
				\ZNoneKey	{UnitPrice}					: NUMBER	\\	% Prix unitaire
				\ZNoneKey	{TotalPrice}				: NUMBER		% Prix total
			}{}
			\umlclass[x=70mm,y=50mm,width=60mm]{Item}{
				\ZPrimaryKey{ItemCode(Id)}				: NUMBER	\\	% Code de l'article
				\ZNoneKey	{Description}				: STRING	\\	% Description de l'article
				\ZNoneKey	{Account}					: STRING	\\	% Numéro du compte
				\ZNoneKey	{CommodityTrackingOrDonor}	: STRING	\\	% Unité
				\ZNoneKey	{Weight}					: NUMBER	\\	% Poids (en kg)
				\ZNoneKey	{Volume}					: NUMBER		% Volume (en m3)
			}{}
		\end{umlpackage}
		\begin{umlpackage}[x=100,y=0]{TransportMean}
			\umlclass[x=0,y=0,width=60mm]{TransportMean}{
				\ZPrimaryKey{TransportMeanId(Id)}		: STRING		% Type de transport (aérien, ferroviaire, routié, maritime, ...) respectivement noté [AIR, RAIL, ROAD, SEA, ...]
			}{}
		\end{umlpackage}
	\end{tikzpicture}
	\caption{Structuration des données dans le SGBDR~: le package réquisition}
	\label{DonneesStructurationSgbdrRequisition}
\end{figure}

\begin{figure}[htbp]
	\centering
	\begin{tikzpicture}
		\begin{umlpackage}[x=0,y=0]{WaybillDeliveryNote}
			%\umlclass[x=70mm,y=-50mm,width=60mm]{Person}{
			%}{}
			\umlclass[x=0mm,y=0mm,width=60mm]{WaybillDeliveryNote}{
				\ZPrimaryKey{CountryCode}				: NUMBER	\\ % Code de l'article
				\ZPrimaryKey{Number(Id)}				: STRING	\\ % Numéro du compte
				\ZNoneKey	{Date}						: STRING	\\ % Numéro du compte
				\ZNoneKey	{Warehouse}					: STRING	\\ % Numéro du compte
				\ZNoneKey	{Destination}				: STRING	\\ % Numéro du compte
				\ZNoneKey	{Beneficiary}				: STRING	\\ % Numéro du compte
				% Transport Data
				\ZNoneKey	{ContractNumber}			: STRING	\\ % Numéro du compte
				\ZNoneKey	{Vehicle}					: STRING	\\ % Numéro du véhicule (avec type de transport)
				\ZNoneKey	{RegistrationNumber}		: STRING	\\ % Numéro du compte
				\ZNoneKey	{ETD}						: STRING	\\ % Numéro du compte
				\ZNoneKey	{Vehicle}					: STRING	\\ % Numéro du compte
				\ZNoneKey	{Vehicle}					: STRING	\\ % Numéro du compte
				\ZNoneKey	{Vehicle}					: STRING	\\ % Numéro du compte
				\ZForeignKey{Cargo}						: ?			\\ % Code de l'article
				\ZNoneKey	{Comments}					: STRING	\\ % Numéro du compte
				\ZNoneKey	{CommentsFromReceiver}		: STRING	\\ % Numéro du compte
%				\ZNoneKey	{UnitTotal}					: STRING	\\ % Numéro du compte
%				\ZNoneKey	{WeightTotal}				: STRING	\\ % Numéro du compte
				% Informations de chargement
				\ZForeignKey{Loader}					: ?			\\	% Responsable du chargement
				\ZNoneKey	{LoadingDate}				: DATE		\\	% Date de chargement
				\ZNoneKey	{LoadingLocation}			: STRING	\\	% Lieu (site) du chargement
				\ZNoneKey	{LoadingCondition}			: STRING	\\	% Condition du chargement
				\ZForeignKey{Transporter}				: ?			\\	% Responsable du transport
				\ZNoneKey	{TransportingDate}			: DATE		\\	% Date du transport
%				\ZNoneKey	{TransportingLocation}		: STRING	\\	% Lieu (site) du transport
				\ZNoneKey	{TransportingCondition}		: STRING	\\	% Condition du transport
				% Informations de reception
				\ZForeignKey{Receiver}					: ?			\\	% Responsable de la réception
				\ZNoneKey	{ReceptionDate}				: NUMBER	\\	% Date de réception
				\ZNoneKey	{ReceptionLocation}			: NUMBER	\\	% Lieu (site) de réception
				\ZNoneKey	{ReceptionCondition}		: STRING		% Condition de réception
			}{}
			\umlclass[x=70mm,y=0,width=60mm]{Cargo}{
				\ZPrimaryKey{Number(Id)}				: NUMBER	\\	% Identifiant
				\ZForeignKey{ItemCode}					: REFERENCE	\\	% Code de l'article
				\ZNoneKey	{Quantity}					: NUMBER	\\	% Quantité
%				\ZNoneKey	{UnitTypeWeight}			: NUMBER	\\	% Quantité
				\ZForeignKey{Requisition}				: REFERENCE	\\	% Code de l'article
%				\ZNoneKey	{Weight}					: NUMBER	\\	% Poids total
%				\ZNoneKey	{Volume}					: NUMBER	\\	% Volume total
				\ZNoneKey	{Remarks}					: STRING		% Prix total
			}{}
		\end{umlpackage}
	\end{tikzpicture}
	\caption{Structuration des données dans le SGBDR~: le package waybill/delivery~note}
	\label{DonneesStructurationSgbdrWaybill}
\end{figure}

\begin{figure}[htbp]
	\centering
	\begin{tikzpicture}
		\begin{umlpackage}[x=0,y=0]{Driver}
			\umlclass[x=70mm,y=0mm,width=60mm]{Driver}{
				\ZPrimaryKey{DriverId}					: NUMBER	\\	% Identifiant du conducteur
				\ZNoneKey	{Lastname}					: STRING	\\	% Nom de famille du conducteur
				\ZNoneKey	{Firstname}					: STRING	\\	% Prénom du conducteur
				\ZNoneKey	{Birthday}					: DATE		\\	% Date de naissance du conducteur
				\ZForeignKey{DrivingLicence}			: ?				% Permis possédés par le conducteur
			}{}
			\umlclass[x=70mm,y=0mm,width=60mm]{OwndDrivingLicence}{
				\ZForeignKey{DriverId}					: ?			\\	% Identifiant du conducteur
				\ZForeignKey{DrivingLicenceId}			: ?				% Identifiant du permis de conduire
			}{}
			\umlclass[x=70mm,y=50mm,width=60mm]{DrivingLicence}{
				\ZPrimaryKey{DrivingLicenceId}			: STRING	\\	% Identifiant du permis de conduire
				\ZNoneKey	{Category}					: STRING	\\	% Catégorie des véhicules concernés (A, B, C, D, ...)
				\ZNoneKey	{ValidityArea}				: STRING		% Zone de validité du permis de conduire (pays si national, europe, international, ...)
			}{}
			\umlclass[x=70mm,y=0mm,width=60mm]{EmployerEmployee}{
				\ZForeignKey{ProviderId}				: ?			\\	% Identifiant du prestataire
				\ZForeignKey{DriverId}					: ?				% Identifiant du conducteur
			}{}
		\end{umlpackage}
	\end{tikzpicture}
	\caption{Structuration des données dans le SGBDR~: le package driver}
	\label{DonneesStructurationSgbdrDriver}
\end{figure}

\begin{figure}[htbp]
	\centering
	\begin{tikzpicture}
		\begin{umlpackage}[x=0,y=0]{Vehicle}
			\umlclass[x=70mm,y=-50mm,width=60mm]{Vehicle}{
				\ZPrimaryKey{VehicleId}					: NUMBER	\\	% Identifiant du vehicle
				\ZForeignKey{VehicleTypeId}				: ?			\\	% Identifiant du type du vehicle
				\ZNoneKey	{MaxWeight}					: NUMBER	\\	% Poids maximal embarquable
				\ZNoneKey	{MaxVolume}					: NUMBER	\\	% Volume maximal embarquable
				\ZNoneKey	{MaxDistance}				: NUMBER		% Distance maximale parcourable sans refaire le plein
			}{}
			\umlclass[x=70mm,y=0mm,width=60mm]{VehicleType}{
				\ZPrimaryKey{VehicleTypeId}				: STRING		% Identifiant du type de vehicle (avion, bateau, train, camion, mule, ...)
			}{}
			\umlclass[x=70mm,y=0mm,width=60mm]{EmployerVehicle}{
				\ZForeignKey{ProviderId}				: ?			\\	% Identifiant du prestataire
				\ZForeignKey{VehicleId}					: ?				% Identifiant du véhicule
			}{}
		\end{umlpackage}
	\end{tikzpicture}
	\caption{Structuration des données dans le SGBDR~: le package vehicle}
	\label{DonneesStructurationSgbdrVehicle}
\end{figure}

\begin{figure}[htbp]
	\centering
	\begin{tikzpicture}
		\begin{umlpackage}[x=0,y=0]{Provider}
			\umlclass[x=70mm,y=-50mm,width=60mm]{Provider}{
				\ZPrimaryKey{ProviderId}				: NUMBER	\\	% Identifiant du prestataire
				\ZNoneKey	{ProvideName}				: STRING	\\	% Nom du prestataire
				\ZNoneKey	{ProvideLicence}			: NUMBER	\\	% Patente du prestataire
			}{}
			\umlclass[x=70mm,y=0mm,width=60mm]{ProviderContracts}{
				\ZPrimaryKey{ProviderId}				: STRING		% Identifiant du prestataire
				\ZForeignKey{ProviderContractTypeId}	: ?				% Identifiant du type de contract
				\ZNoneKey	{ProviderContractNumeric}	: IMAGE			% Version numérique du contract
			}{}
			\umlclass[x=70mm,y=0mm,width=60mm]{ProviderContractType}{
				\ZPrimaryKey{ProviderContractTypeId}	: STRING		% Identifiant du type de vehicle (avion, bateau, train, camion, mule, ...)
			}{}
		\end{umlpackage}
	\end{tikzpicture}
	\caption{Structuration des données dans le SGBDR~: le package provider}
	\label{DonneesStructurationSgbdrProvider}
\end{figure}

\clearpage
MCD~:
\begin{figure}[htbp]
	\centering
	\begin{tikzpicture}
		\begin{umlpackage}[x=10,y=0]{VehiclePackage}
			\umlsimpleclass[x=0mm,y=-10mm,width=40mm]{Vehicle}{}{}
			\umlsimpleclass[x=0mm,y=0mm,width=40mm]{VehicleType}{}{}
		\end{umlpackage}
		\begin{umlpackage}[x=10,y=-5]{ProviderPackage}
			\umlsimpleclass[x=0mm,y=0mm,width=40mm]{Provider}{}{}
			\umlsimpleclass[x=0mm,y=-10mm,width=40mm]{ProviderContracts}{}{}
			\umlsimpleclass[x=0mm,y=-20mm,width=40mm]{ProviderContractType}{}{}
		\end{umlpackage}
		\umlsimpleclass[x=10,y=-3,width=40mm]{EmployerVehicle}{}{}
		\umluniassoc{Vehicle}{EmployerVehicle}
		\umluniassoc{Provider}{EmployerVehicle}
		\begin{umlpackage}[x=10,y=-10]{Driver}
			\umlsimpleclass[x=0mm,y=0mm,width=40mm]{Driver}{}{}
			\umlsimpleclass[x=0mm,y=-10mm,width=40mm]{OwndDrivingLicence}{}{}
			\umlsimpleclass[x=0mm,y=-20mm,width=40mm]{DrivingLicence}{}{}
		\end{umlpackage}
		\umlsimpleclass[x=10,y=-8,width=40mm]{EmployerEmployee}{}{}
		\umluniassoc{Driver}{EmployerEmployee}
		\umluniassoc{Provider}{EmployerEmployee}
		\begin{umlpackage}[x=0,y=0]{WaybillDeliveryNote}
			\umlsimpleclass[x=0mm,y=0mm,width=40mm]{WaybillDeliveryNote}{}{}
			\umlsimpleclass[x=0mm,y=-10mm,width=40mm]{Cargo}{}{}
		\end{umlpackage}
		\begin{umlpackage}[x=0,y=-5]{Requisition}
			\umlsimpleclass[x=0mm,y=0mm,width=40mm]{Requisition}{}{}
			\umlsimpleclass[x=0mm,y=-10mm,width=40mm]{Cargo}{}{}
			\umlsimpleclass[x=0mm,y=-20mm,width=40mm]{Item}{}{}
		\end{umlpackage}
%		\begin{umlpackage}[x=100,y=0]{TransportMean}
%			\umlclass[x=0,y=0,width=60mm]{TransportMean}{}{}
%		\end{umlpackage}
	\end{tikzpicture}
	\caption{Structuration des données dans le SGBDR~: MCD}
	\label{DonneesStructurationSgbdrMcd}
\end{figure}

%\umluniassoc[geometry=-|,arg1=x,mult1=1,pos1=1.9,arg2=y,mult2=*,pos2=0.2]{A1}{A2} 

\subsection{Le transfert}
Le transfert des données quant à lui, est réalisé selon un format et une norme qui est préalablement défini.

\subsubsection{Format}
Eu égard des solutions retenues pour le transfert des données, le transfert des données se fera grâce à des fichiers XML. \\
La possibilité est donnée de compresser ces fichiers, dans un premier temps au format \emph{ZIP}, mais ce choix peut être remis en question selon les observations réalisées sur les données de test, afin de retenir une éventuelle meilleure solution.

\subsubsection{norme}
% Définir la norme du format XML correspondant à une table SQL
% avec une DTD associée.
% 
% Exemple de DTD pour une personne :
% 
% <!ELEMENT personne (nom,prenom,telephone),email? >
% <!ELEMENT nom (#PCDATA) >
% <!ELEMENT prenom (#PCDATA) >
% <!ELEMENT telephone (#PCDATA) >
% <!ELEMENT email (#PCDATA) >

\subsection{L'import/export de données}
L'import et l'export de données se fait directement via une fonctionnalité de l'outil côté client, qui permet de sélectionner les données à importer dans le contexte spécifique, ainsi que les données à exporter vers le serveur central, toujours selon ce contexte.
Comme décrit dans la partie traitant de ce sujet, l'import/export de données est réalisé selon différents supports (réseau, support physique) à la discrétion de l'exploitant.
