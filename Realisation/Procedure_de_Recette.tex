%-------------------------------------------------------------------------------
%
%     CHARGEMENT DES EXTENSIONS
%
%-------------------------------------------------------------------------------

\documentclass[11pt,fleqn]{report}
\usepackage{GarmirKhatch}

%-------------------------------------------------------------------------------
%     Informations spécifiques au document
%-------------------------------------------------------------------------------

\ZTitle{Système de gestion des transports}
\ZSubject{Procédure de recette}
\ZVersion{1.0}
\ZDate{2014-03-26}

%-------------------------------------------------------------------------------
%     Contenu
%-------------------------------------------------------------------------------


\begin{document}
    
\ZMakeCover

\ZMakeInformations{
	% Historique des modifications
	% Version & Date & Auteur(s) & Modification(s)
	1.0 & 2014-03-26 & \Julien & Réalisation des tests et du document \\
}{
	% Historique des approbations
	% Version & Date & Approbateur(s)
	1.0 & 2014-03-26 & \Cadon \\
}{
	% Historique des validations
	% Version & Date & Responsable(s)
	1.0 & - & - \\
}

\ZMakeTableOfContents

\chapter{Introduction}
\section{Objectif}
Ce document propose une série de tests et de scénarios décrivant avec précision les démarches à suivre dans le cadre de l'utilisation du logiciel de gestion des transports. Il sert de support à la validation du logiciel lors de la recette auprès du client.
Différents types de tests ont été réalisé : 
\begin{itemize}
\item les tests sur l'interface graphique,
\item les tests unitaires,
\item les tests sur le système de gestion de données.
\end{itemize}
Les tests unitaires ont été fait sur l'environnement de développement. 
Au cours de l'évolution du développement de l'outil, le code et les données de bases ont changés c'est pourquoi les jeux d'essais peuvent être rechargés  et ré initialisés afin de permettre de refaire les mêmes tests dans les mêmes conditions. 
Il est à noter que des tests de régression ont été fait suite à des corrections du programme.

La phase de réalisation s'étalant sur une courte durée les tests ont commencé en parallèle de celle-ci. De ce fait, dans un premier temps les tests étaient effectués sur des classes \emph{mocks}, qui permettent de simuler les résultats d'une classe. Puis une fois les classes réalisées, les tests ont porté sur ces-dernières.

\chapter{Tests sur interface graphique}
L'outil a été développé en c++ à l'aide du framework Qt. Ce framework intègre un outil de test grâce à la librairie QTestlib.
C'est donc à l'aide de QTest que les scénarios ci-dessous ont été réalisé.
Pour effectuer et visualiser tous les tests à partir de l'exécutable, il faut rajouter l'argument \emph{-testall}.

\section{Tests graphiques de la partie waybill}
\begin{center}
\newcolumntype{R}[1]{>{\raggedleft\arraybackslash }b{#1}}
\newcolumntype{L}[1]{>{\raggedright\arraybackslash }b{#1}}
\newcolumntype{C}[1]{>{\centering\arraybackslash }b{#1}}
\begin{tabular}{|L{3cm}|L{10cm}|}
\hline \emph{Cas de test :} & Test-Graphique-01  \\
\hline \emph{Titre :} & TestWaybillCountryCode   \\
\hline \emph{Objectif :} & Vérifier le bon fonctionnement du champ du code pays    \\
\hline \emph{Procédure :} & Cliquer dans le champ du code pays, le remplir avec une chaîne et vérifier que la chaîne correspond   \\
\hline \emph{Données de test :} & On rempli avec la chaîne ``Fr''   \\
\hline \emph{Config :} & Using QtTest library 5.2.1, Qt 5.2.1   \\
\hline \emph{Résultat :} & PASS   \\
\hline 
\end{tabular} 
\end{center}

\begin{center}
\newcolumntype{R}[1]{>{\raggedleft\arraybackslash }b{#1}}
\newcolumntype{L}[1]{>{\raggedright\arraybackslash }b{#1}}
\newcolumntype{C}[1]{>{\centering\arraybackslash }b{#1}}
\begin{tabular}{|L{3cm}|L{10cm}|}
\hline \emph{Cas de test :} & Test-Graphique-02  \\
\hline \emph{Titre :} & TestWaybillCountryId   \\
\hline \emph{Objectif :} & Vérifier le bon fonctionnement du champ de l'id du pays    \\
\hline \emph{Procédure :} & Cliquer dans le champ de l'id du pays, le remplir avec une id et vérifier que l'id correspond   \\
\hline \emph{Données de test :} & On rempli avec l'id ``7''   \\
\hline \emph{Config :} & Using QtTest library 5.2.1, Qt 5.2.1   \\
\hline \emph{Résultat :} & PASS   \\
\hline 
\end{tabular} 
\end{center}

\begin{center}
\newcolumntype{R}[1]{>{\raggedleft\arraybackslash }b{#1}}
\newcolumntype{L}[1]{>{\raggedright\arraybackslash }b{#1}}
\newcolumntype{C}[1]{>{\centering\arraybackslash }b{#1}}
\begin{tabular}{|L{3cm}|L{10cm}|}
\hline \emph{Cas de test :} & Test-Graphique-03  \\
\hline \emph{Titre :} & TestWaybillWarehouse   \\
\hline \emph{Objectif :} & Vérifier le bon fonctionnement du champ de localisation de l'entrepôt   \\
\hline \emph{Procédure :} & Cliquer dans le champ de localisation de l'entrepôt, le remplir avec une chaîne et vérifier que la chaîne correspond   \\
\hline \emph{Données de test :} & On rempli avec la chaîne ``Marseille''   \\
\hline \emph{Config :} & Using QtTest library 5.2.1, Qt 5.2.1   \\
\hline \emph{Résultat :} & PASS   \\
\hline 
\end{tabular} 
\end{center}

\begin{center}
\newcolumntype{R}[1]{>{\raggedleft\arraybackslash }b{#1}}
\newcolumntype{L}[1]{>{\raggedright\arraybackslash }b{#1}}
\newcolumntype{C}[1]{>{\centering\arraybackslash }b{#1}}
\begin{tabular}{|L{3cm}|L{10cm}|}
\hline \emph{Cas de test :} & Test-Graphique-04  \\
\hline \emph{Titre :} & TestWaybillVehicle   \\
\hline \emph{Objectif :} & Vérifier le bon fonctionnement du champ du numéro d'identification et du champ du transport.   \\
\hline \emph{Procédure :} & Cliquer dans les champ du numéro d'identification et  transport, saisir deux chaînes et les vérifier \\
\hline \emph{Données de test :} & On rempli avec les chaînes ``200'' et ``100''\\
\hline \emph{Config :} & Using QtTest library 5.2.1, Qt 5.2.1   \\
\hline \emph{Résultat :} & PASS   \\
\hline 
\end{tabular} 
\end{center}

\begin{center}
\newcolumntype{R}[1]{>{\raggedleft\arraybackslash }b{#1}}
\newcolumntype{L}[1]{>{\raggedright\arraybackslash }b{#1}}
\newcolumntype{C}[1]{>{\centering\arraybackslash }b{#1}}
\begin{tabular}{|L{3cm}|L{10cm}|}
\hline \emph{Cas de test :} & Test-Graphique-05  \\
\hline \emph{Titre :} & TestWaybillLoading   \\
\hline \emph{Objectif :} & Vérifier le bon fonctionnement des champs concernant le chargement.   \\
\hline \emph{Procédure :} & Cliquer dans les différents champs, et saisir des chaînes. \\
\hline \emph{Données de test :} & On rempli avec les chaînes ``Actor'', ``Location'', ``Condition''.\\
\hline \emph{Config :} & Using QtTest library 5.2.1, Qt 5.2.1   \\
\hline \emph{Résultat :} & PASS   \\
\hline 
\end{tabular} 
\end{center}

\begin{center}
\newcolumntype{R}[1]{>{\raggedleft\arraybackslash }b{#1}}
\newcolumntype{L}[1]{>{\raggedright\arraybackslash }b{#1}}
\newcolumntype{C}[1]{>{\centering\arraybackslash }b{#1}}
\begin{tabular}{|L{3cm}|L{10cm}|}
\hline \emph{Cas de test :} & Test-Graphique-06  \\
\hline \emph{Titre :} & TestWaybillTransport   \\
\hline \emph{Objectif :} & Vérifier le bon fonctionnement des champs concernant le transport.   \\
\hline \emph{Procédure :} & Cliquer dans les différents champs, et saisir des chaînes. \\
\hline \emph{Données de test :} & On rempli avec les chaînes ``Actor'', ``Location'', ``Condition''.\\
\hline \emph{Config :} & Using QtTest library 5.2.1, Qt 5.2.1   \\
\hline \emph{Résultat :} & PASS   \\
\hline 
\end{tabular} 
\end{center}

\begin{center}
\newcolumntype{R}[1]{>{\raggedleft\arraybackslash }b{#1}}
\newcolumntype{L}[1]{>{\raggedright\arraybackslash }b{#1}}
\newcolumntype{C}[1]{>{\centering\arraybackslash }b{#1}}
\begin{tabular}{|L{3cm}|L{10cm}|}
\hline \emph{Cas de test :} & Test-Graphique-07  \\
\hline \emph{Titre :} & TestWaybillReception   \\
\hline \emph{Objectif :} & Vérifier le bon fonctionnement des champs concernant la recepetion du chargement.   \\
\hline \emph{Procédure :} & Cliquer dans les différents champs, et saisir des chaînes. \\
\hline \emph{Données de test :} & On rempli avec les chaînes ``Actor'', ``Location'', ``Condition''.\\
\hline \emph{Config :} & Using QtTest library 5.2.1, Qt 5.2.1   \\
\hline \emph{Résultat :} & PASS   \\
\hline 
\end{tabular} 
\end{center}

Les résutats de tous les tests sont satisfaisant.\\
Pour effectuer et visualiser les tests de la partie \emph{waybill} à partir de l'exécutable, il faut rajouter l'argument \emph{-test1}.

********* Start testing of WaybillTabTest *********\\
Config: Using QtTest library 5.2.1, Qt 5.2.1\\
PASS   : WaybillTabTest::initTestCase()\\
PASS   : WaybillTabTest::qshowIfNotHidden()\\
PASS   : WaybillTabTest::onplannedLoadingDateEditdateChanged()\\
PASS   : WaybillTabTest::onrealLoadingDateEditdateChanged()\\
PASS   : WaybillTabTest::onplannedTransportDateEditdateChanged()\\
PASS   : WaybillTabTest::onrealTransportDateEditdateChanged()\\
PASS   : WaybillTabTest::onplannedReceptionDateEditdateChanged()\\
PASS   : WaybillTabTest::onrealReceptionDateEditdateChanged()\\
PASS   : WaybillTabTest::loadCss()\\
PASS   : WaybillTabTest::testCountryCode()\\
PASS   : WaybillTabTest::testId()\\
PASS   : WaybillTabTest::testPlannedLoading()\\
PASS   : WaybillTabTest::testPlannedReception()\\
PASS   : WaybillTabTest::testPlannedTransport()\\
PASS   : WaybillTabTest::testRealLoading()\\
PASS   : WaybillTabTest::testRealReception()\\
PASS   : WaybillTabTest::testRealTransport()\\
PASS   : WaybillTabTest::testTransportRegistrationNo()\\
PASS   : WaybillTabTest::testTransportVehicle()\\
PASS   : WaybillTabTest::testWarehouse()\\
PASS   : WaybillTabTest::cleanupTestCase()\\
Totals: 21 passed, 0 failed, 0 skipped\\
********* Finished testing of WaybillTabTest *********\\


\section{Tests graphiques de la partie réquisition}
\begin{center}
\newcolumntype{R}[1]{>{\raggedleft\arraybackslash }b{#1}}
\newcolumntype{L}[1]{>{\raggedright\arraybackslash }b{#1}}
\newcolumntype{C}[1]{>{\centering\arraybackslash }b{#1}}
\begin{tabular}{|L{3cm}|L{10cm}|}
\hline \emph{Cas de test :} & Test-Graphique-08  \\
\hline \emph{Titre :} & testCountryCode   \\
\hline \emph{Objectif :} & Vérifier le bon fonctionnement du champ du code pays dans l'onglet réquisition   \\
\hline \emph{Procédure :} & Cliquer dans le champ du code pays, le remplir avec une chaîne et vérifier que la chaîne correspond   \\
\hline \emph{Données de test :} & On rempli avec la chaîne ``Fr''   \\
\hline \emph{Config :} & Using QtTest library 5.2.1, Qt 5.2.1   \\
\hline \emph{Résultat :} & PASS   \\
\hline 
\end{tabular} 
\end{center}

\begin{center}
\newcolumntype{R}[1]{>{\raggedleft\arraybackslash }b{#1}}
\newcolumntype{L}[1]{>{\raggedright\arraybackslash }b{#1}}
\newcolumntype{C}[1]{>{\centering\arraybackslash }b{#1}}
\begin{tabular}{|L{3cm}|L{10cm}|}
\hline \emph{Cas de test :} & Test-Graphique-09  \\
\hline \emph{Titre :} & testId   \\
\hline \emph{Objectif :} & Vérifier le bon fonctionnement du champ de l'identifiant de la réquisition   \\
\hline \emph{Procédure :} & Cliquer dans le champ de l'identifiant, le remplir avec une chaîne et vérifier que la chaîne correspond   \\
\hline \emph{Données de test :} & On rempli avec la chaîne ``8''   \\
\hline \emph{Config :} & Using QtTest library 5.2.1, Qt 5.2.1   \\
\hline \emph{Résultat :} & PASS   \\
\hline 
\end{tabular} 
\end{center}

\begin{center}
\newcolumntype{R}[1]{>{\raggedleft\arraybackslash }b{#1}}
\newcolumntype{L}[1]{>{\raggedright\arraybackslash }b{#1}}
\newcolumntype{C}[1]{>{\centering\arraybackslash }b{#1}}
\begin{tabular}{|L{3cm}|L{10cm}|}
\hline \emph{Cas de test :} & Test-Graphique-10  \\
\hline \emph{Titre :} & testOrigin   \\
\hline \emph{Objectif :} & Vérifier le bon fonctionnement du champ de l'émetteur de la réquisition   \\
\hline \emph{Procédure :} & Cliquer dans le champ, le remplir avec une chaîne et vérifier que la chaîne correspond   \\
\hline \emph{Données de test :} & On rempli avec la chaîne ``Name1''   \\
\hline \emph{Config :} & Using QtTest library 5.2.1, Qt 5.2.1   \\
\hline \emph{Résultat :} & PASS   \\
\hline 
\end{tabular} 
\end{center}

\begin{center}
\newcolumntype{R}[1]{>{\raggedleft\arraybackslash }b{#1}}
\newcolumntype{L}[1]{>{\raggedright\arraybackslash }b{#1}}
\newcolumntype{C}[1]{>{\centering\arraybackslash }b{#1}}
\begin{tabular}{|L{3cm}|L{10cm}|}
\hline \emph{Cas de test :} & Test-Graphique-11  \\
\hline \emph{Titre :} & testDestination   \\
\hline \emph{Objectif :} & Vérifier le bon fonctionnement du champ du destinataire de la réquisition   \\
\hline \emph{Procédure :} & Cliquer dans le champ, le remplir avec une chaîne et vérifier que la chaîne correspond   \\
\hline \emph{Données de test :} & On rempli avec la chaîne ``Name2''   \\
\hline \emph{Config :} & Using QtTest library 5.2.1, Qt 5.2.1   \\
\hline \emph{Résultat :} & PASS   \\
\hline 
\end{tabular} 
\end{center}

\begin{center}
\newcolumntype{R}[1]{>{\raggedleft\arraybackslash }b{#1}}
\newcolumntype{L}[1]{>{\raggedright\arraybackslash }b{#1}}
\newcolumntype{C}[1]{>{\centering\arraybackslash }b{#1}}
\begin{tabular}{|L{3cm}|L{10cm}|}
\hline \emph{Cas de test :} & Test-Graphique-12  \\
\hline \emph{Titre :} & testRequesterName   \\
\hline \emph{Objectif :} & Vérifier le bon fonctionnement du champ de la personne responsable de la requête   \\
\hline \emph{Procédure :} & Cliquer dans le champ, le remplir avec une chaîne et vérifier que la chaîne correspond   \\
\hline \emph{Données de test :} & On rempli avec la chaîne ``Requester Name''   \\ 
\hline \emph{Config :} & Using QtTest library 5.2.1, Qt 5.2.1   \\
\hline \emph{Résultat :} & PASS   \\
\hline 
\end{tabular} 
\end{center}

\begin{center}
\newcolumntype{R}[1]{>{\raggedleft\arraybackslash }b{#1}}
\newcolumntype{L}[1]{>{\raggedright\arraybackslash }b{#1}}
\newcolumntype{C}[1]{>{\centering\arraybackslash }b{#1}}
\begin{tabular}{|L{3cm}|L{10cm}|}
\hline \emph{Cas de test :} & Test-Graphique-13  \\
\hline \emph{Titre :} & testProjectManager   \\
\hline \emph{Objectif :} & Vérifier le bon fonctionnement du champ de la personne responsable du projet   \\
\hline \emph{Procédure :} & Cliquer dans le champ, le remplir avec une chaîne et vérifier que la chaîne correspond   \\
\hline \emph{Données de test :} & On rempli avec la chaîne ``Project Manager''   \\
\hline \emph{Config :} & Using QtTest library 5.2.1, Qt 5.2.1   \\
\hline \emph{Résultat :} & PASS   \\
\hline 
\end{tabular} 
\end{center}

\begin{center}
\newcolumntype{R}[1]{>{\raggedleft\arraybackslash }b{#1}}
\newcolumntype{L}[1]{>{\raggedright\arraybackslash }b{#1}}
\newcolumntype{C}[1]{>{\centering\arraybackslash }b{#1}}
\begin{tabular}{|L{3cm}|L{10cm}|}
\hline \emph{Cas de test :} & Test-Graphique-14  \\
\hline \emph{Titre :} & testFinanceOfficer   \\
\hline \emph{Objectif :} & Vérifier le bon fonctionnement du champ de la personne responsable des finances   \\
\hline \emph{Procédure :} & Cliquer dans le champ, le remplir avec une chaîne et vérifier que la chaîne correspond   \\
\hline \emph{Données de test :} & On rempli avec la chaîne ``Finance Officer''   \\
\hline \emph{Config :} & Using QtTest library 5.2.1, Qt 5.2.1   \\
\hline \emph{Résultat :} & PASS   \\
\hline 
\end{tabular} 
\end{center}

\begin{center}
\newcolumntype{R}[1]{>{\raggedleft\arraybackslash }b{#1}}
\newcolumntype{L}[1]{>{\raggedright\arraybackslash }b{#1}}
\newcolumntype{C}[1]{>{\centering\arraybackslash }b{#1}}
\begin{tabular}{|L{3cm}|L{10cm}|}
\hline \emph{Cas de test :} & Test-Graphique-15  \\
\hline \emph{Titre :} & testLogistics   \\
\hline \emph{Objectif :} & Vérifier le bon fonctionnement du champ logisticien   \\
\hline \emph{Procédure :} & Cliquer dans le champ, le remplir avec une chaîne et vérifier que la chaîne correspond   \\
\hline \emph{Données de test :} & On rempli avec la chaîne ``Logistics''   \\
\hline \emph{Config :} & Using QtTest library 5.2.1, Qt 5.2.1   \\
\hline \emph{Résultat :} & PASS   \\
\hline 
\end{tabular} 
\end{center}

\begin{center}
\newcolumntype{R}[1]{>{\raggedleft\arraybackslash }b{#1}}
\newcolumntype{L}[1]{>{\raggedright\arraybackslash }b{#1}}
\newcolumntype{C}[1]{>{\centering\arraybackslash }b{#1}}
\begin{tabular}{|L{3cm}|L{10cm}|}
\hline \emph{Cas de test :} & Test-Graphique-19  \\
\hline \emph{Titre :} & testLogistics   \\
\hline \emph{Objectif :} & Vérifier le bon fonctionnement du champ GlobalFleetBase   \\
\hline \emph{Procédure :} & Cliquer dans le champ, le remplir avec une chaîne et vérifier que la chaîne correspond   \\
\hline \emph{Données de test :} & On rempli avec la chaîne ``GlobalFleetBase''   \\
\hline \emph{Config :} & Using QtTest library 5.2.1, Qt 5.2.1   \\
\hline \emph{Résultat :} & PASS   \\
\hline 
\end{tabular} 
\end{center}


Les résutats de tous les tests sont satisfaisant.\\
Pour effectuer et visualiser les tests de la partie \emph{réquisition} à partir de l'exécutable, il faut rajouter l'argument \emph{-test2}.

********* Start testing of RequisitionTabTest *********\\
Config: Using QtTest library 5.2.1, Qt 5.2.1\\
PASS   : RequisitionTabTest::initTestCase()\\
PASS   : RequisitionTabTest::qshowIfNotHidden()\\
PASS   : RequisitionTabTest::loadCss()\\
PASS   : RequisitionTabTest::testCountryCode()\\
PASS   : RequisitionTabTest::testId()\\
PASS   : RequisitionTabTest::testOrigin()\\
PASS   : RequisitionTabTest::testDestination()\\
PASS   : RequisitionTabTest::testRequesterName()\\
PASS   : RequisitionTabTest::testProjectManager()\\
PASS   : RequisitionTabTest::testFinanceOfficer()\\
PASS   : RequisitionTabTest::testLogistics()\\
PASS   : RequisitionTabTest::testGlobalFleetBase()\\
PASS   : RequisitionTabTest::cleanupTestCase()\\
Totals: 13 passed, 0 failed, 0 skipped\\
********* Finished testing of RequisitionTabTest *********\\

\section{Tests graphiques de la partie personnels}

\begin{center}
\newcolumntype{R}[1]{>{\raggedleft\arraybackslash }b{#1}}
\newcolumntype{L}[1]{>{\raggedright\arraybackslash }b{#1}}
\newcolumntype{C}[1]{>{\centering\arraybackslash }b{#1}}
\begin{tabular}{|L{3cm}|L{10cm}|}
\hline \emph{Cas de test :} & Test-Graphique-20  \\
\hline \emph{Titre :} & testId   \\
\hline \emph{Objectif :} & Vérifier le bon fonctionnement du champ d'identification d'une personne   \\
\hline \emph{Procédure :} & Cliquer dans le champ, le remplir avec une chaîne et vérifier que la chaîne correspond   \\
\hline \emph{Données de test :} & On rempli avec la chaîne ``55''   \\
\hline \emph{Config :} & Using QtTest library 5.2.1, Qt 5.2.1   \\
\hline \emph{Résultat :} & PASS   \\
\hline 
\end{tabular} 
\end{center}

\begin{center}
\newcolumntype{R}[1]{>{\raggedleft\arraybackslash }b{#1}}
\newcolumntype{L}[1]{>{\raggedright\arraybackslash }b{#1}}
\newcolumntype{C}[1]{>{\centering\arraybackslash }b{#1}}
\begin{tabular}{|L{3cm}|L{10cm}|}
\hline \emph{Cas de test :} & Test-Graphique-21  \\
\hline \emph{Titre :} & testLastName   \\
\hline \emph{Objectif :} & Vérifier le bon fonctionnement du champ nom de famille d'une personne   \\
\hline \emph{Procédure :} & Cliquer dans le champ, le remplir avec une chaîne et vérifier que la chaîne correspond   \\
\hline \emph{Données de test :} & On rempli avec la chaîne ``Agopian''   \\
\hline \emph{Config :} & Using QtTest library 5.2.1, Qt 5.2.1   \\
\hline \emph{Résultat :} & PASS   \\
\hline 
\end{tabular} 
\end{center}

\begin{center}
\newcolumntype{R}[1]{>{\raggedleft\arraybackslash }b{#1}}
\newcolumntype{L}[1]{>{\raggedright\arraybackslash }b{#1}}
\newcolumntype{C}[1]{>{\centering\arraybackslash }b{#1}}
\begin{tabular}{|L{3cm}|L{10cm}|}
\hline \emph{Cas de test :} & Test-Graphique-22  \\
\hline \emph{Titre :} & testFirstName   \\
\hline \emph{Objectif :} & Vérifier le bon fonctionnement du champ prénom d'une personne   \\
\hline \emph{Procédure :} & Cliquer dans le champ, le remplir avec une chaîne et vérifier que la chaîne correspond   \\
\hline \emph{Données de test :} & On rempli avec la chaîne ``Roland''   \\
\hline \emph{Config :} & Using QtTest library 5.2.1, Qt 5.2.1   \\
\hline \emph{Résultat :} & PASS   \\
\hline 
\end{tabular} 
\end{center}

\begin{center}
\newcolumntype{R}[1]{>{\raggedleft\arraybackslash }b{#1}}
\newcolumntype{L}[1]{>{\raggedright\arraybackslash }b{#1}}
\newcolumntype{C}[1]{>{\centering\arraybackslash }b{#1}}
\begin{tabular}{|L{3cm}|L{10cm}|}
\hline \emph{Cas de test :} & Test-Graphique-23  \\
\hline \emph{Titre :} & testFix   \\
\hline \emph{Objectif :} & Vérifier le bon fonctionnement du champ numéro de téléphone fixe   \\
\hline \emph{Procédure :} & Cliquer dans le champ, le remplir avec une chaîne et vérifier que la chaîne correspond   \\
\hline \emph{Données de test :} & On rempli avec la chaîne ``0491502148''   \\
\hline \emph{Config :} & Using QtTest library 5.2.1, Qt 5.2.1   \\
\hline \emph{Résultat :} & PASS   \\
\hline 
\end{tabular} 
\end{center}

\begin{center}
\newcolumntype{R}[1]{>{\raggedleft\arraybackslash }b{#1}}
\newcolumntype{L}[1]{>{\raggedright\arraybackslash }b{#1}}
\newcolumntype{C}[1]{>{\centering\arraybackslash }b{#1}}
\begin{tabular}{|L{3cm}|L{10cm}|}
\hline \emph{Cas de test :} & Test-Graphique-24  \\
\hline \emph{Titre :} & testMobil   \\
\hline \emph{Objectif :} & Vérifier le bon fonctionnement du champ numéro de téléphone mobile   \\
\hline \emph{Procédure :} & Cliquer dans le champ, le remplir avec une chaîne et vérifier que la chaîne correspond   \\
\hline \emph{Données de test :} & On rempli avec la chaîne ``0691502148''   \\
\hline \emph{Config :} & Using QtTest library 5.2.1, Qt 5.2.1   \\
\hline \emph{Résultat :} & PASS   \\
\hline 
\end{tabular} 
\end{center}

\begin{center}
\newcolumntype{R}[1]{>{\raggedleft\arraybackslash }b{#1}}
\newcolumntype{L}[1]{>{\raggedright\arraybackslash }b{#1}}
\newcolumntype{C}[1]{>{\centering\arraybackslash }b{#1}}
\begin{tabular}{|L{3cm}|L{10cm}|}
\hline \emph{Cas de test :} & Test-Graphique-25  \\
\hline \emph{Titre :} & testEmail   \\
\hline \emph{Objectif :} & Vérifier le bon fonctionnement du champ e-mail   \\
\hline \emph{Procédure :} & Cliquer dans le champ, le remplir avec une chaîne et vérifier que la chaîne correspond   \\
\hline \emph{Données de test :} & On rempli avec la chaîne ``roland@agopian.fr''   \\
\hline \emph{Config :} & Using QtTest library 5.2.1, Qt 5.2.1   \\
\hline \emph{Résultat :} & PASS   \\
\hline 
\end{tabular} 
\end{center}

\begin{center}
\newcolumntype{R}[1]{>{\raggedleft\arraybackslash }b{#1}}
\newcolumntype{L}[1]{>{\raggedright\arraybackslash }b{#1}}
\newcolumntype{C}[1]{>{\centering\arraybackslash }b{#1}}
\begin{tabular}{|L{3cm}|L{10cm}|}
\hline \emph{Cas de test :} & Test-Graphique-26  \\
\hline \emph{Titre :} & testAddress   \\
\hline \emph{Objectif :} & Vérifier le bon fonctionnement du champ adresse.   \\
\hline \emph{Procédure :} & Cliquer dans le champ, le remplir avec une chaîne et vérifier que la chaîne correspond   \\
\hline \emph{Données de test :} & On rempli avec la chaîne ``171 Avenue de Luminy''   \\
\hline \emph{Config :} & Using QtTest library 5.2.1, Qt 5.2.1   \\
\hline \emph{Résultat :} & PASS   \\
\hline 
\end{tabular} 
\end{center}

\begin{center}
\newcolumntype{R}[1]{>{\raggedleft\arraybackslash }b{#1}}
\newcolumntype{L}[1]{>{\raggedright\arraybackslash }b{#1}}
\newcolumntype{C}[1]{>{\centering\arraybackslash }b{#1}}
\begin{tabular}{|L{3cm}|L{10cm}|}
\hline \emph{Cas de test :} & Test-Graphique-27  \\
\hline \emph{Titre :} & testCategoty   \\
\hline \emph{Objectif :} & Vérifier le bon fonctionnement du champ catégorie du permis.   \\
\hline \emph{Procédure :} & Cliquer dans le champ, le remplir avec une chaîne et vérifier que la chaîne correspond   \\
\hline \emph{Données de test :} & On rempli avec la chaîne ``Class A''   \\
\hline \emph{Config :} & Using QtTest library 5.2.1, Qt 5.2.1   \\
\hline \emph{Résultat :} & PASS   \\
\hline 
\end{tabular} 
\end{center}

\begin{center}
\newcolumntype{R}[1]{>{\raggedleft\arraybackslash }b{#1}}
\newcolumntype{L}[1]{>{\raggedright\arraybackslash }b{#1}}
\newcolumntype{C}[1]{>{\centering\arraybackslash }b{#1}}
\begin{tabular}{|L{3cm}|L{10cm}|}
\hline \emph{Cas de test :} & Test-Graphique-28  \\
\hline \emph{Titre :} & testNumber   \\
\hline \emph{Objectif :} & Vérifier le bon fonctionnement du champ numéro de permis.   \\
\hline \emph{Procédure :} & Cliquer dans le champ, le remplir avec une chaîne et vérifier que la chaîne correspond   \\
\hline \emph{Données de test :} & On rempli avec la chaîne ``15648547''   \\
\hline \emph{Config :} & Using QtTest library 5.2.1, Qt 5.2.1   \\
\hline \emph{Résultat :} & PASS   \\
\hline 
\end{tabular} 
\end{center}

Les résutats de tous les tests sont satisfaisant.\\
Pour effectuer et visualiser les tests de la partie \emph{personnel} à partir de l'exécutable, il faut rajouter l'argument \emph{-test5}.

********* Start testing of StaffTabTest *********\\
Config: Using QtTest library 5.2.1, Qt 5.2.1\\
PASS   : StaffTabTest::initTestCase()\\
PASS   : StaffTabTest::qshowIfNotHidden()\\
PASS   : StaffTabTest::loadCss()\\
PASS   : StaffTabTest::testId()\\
PASS   : StaffTabTest::testLastName()\\
PASS   : StaffTabTest::testFirstName()\\
PASS   : StaffTabTest::testFix()\\
PASS   : StaffTabTest::testMobil()\\
PASS   : StaffTabTest::testEmail()\\
PASS   : StaffTabTest::testAddress()\\
PASS   : StaffTabTest::testCategoty()\\
PASS   : StaffTabTest::testNumber()\\
PASS   : StaffTabTest::cleanupTestCase()\\
Totals: 13 passed, 0 failed, 0 skipped\\
********* Finished testing of StaffTabTest *********\\

\section{Tests graphiques de la partie véhicules}

\begin{center}
\newcolumntype{R}[1]{>{\raggedleft\arraybackslash }b{#1}}
\newcolumntype{L}[1]{>{\raggedright\arraybackslash }b{#1}}
\newcolumntype{C}[1]{>{\centering\arraybackslash }b{#1}}
\begin{tabular}{|L{3cm}|L{10cm}|}
\hline \emph{Cas de test :} & Test-Graphique-29  \\
\hline \emph{Titre :} & testVehicule   \\
\hline \emph{Objectif :} & Vérifier le bon fonctionnement du champ numéro du véhicule.   \\
\hline \emph{Procédure :} & Cliquer dans le champ, le remplir avec une chaîne et vérifier que la chaîne correspond   \\
\hline \emph{Données de test :} & On rempli avec la chaîne ``vh100''   \\
\hline \emph{Config :} & Using QtTest library 5.2.1, Qt 5.2.1   \\
\hline \emph{Résultat :} & PASS   \\
\hline 
\end{tabular} 
\end{center}

\begin{center}
\newcolumntype{R}[1]{>{\raggedleft\arraybackslash }b{#1}}
\newcolumntype{L}[1]{>{\raggedright\arraybackslash }b{#1}}
\newcolumntype{C}[1]{>{\centering\arraybackslash }b{#1}}
\begin{tabular}{|L{3cm}|L{10cm}|}
\hline \emph{Cas de test :} & Test-Graphique-30  \\
\hline \emph{Titre :} & testVehiculeRegistration   \\
\hline \emph{Objectif :} & Vérifier le bon fonctionnement du champ de la plaque d'immatriculation du véhicule.   \\
\hline \emph{Procédure :} & Cliquer dans le champ, le remplir avec une chaîne et vérifier que la chaîne correspond   \\
\hline \emph{Données de test :} & On rempli avec la chaîne ``100''   \\
\hline \emph{Config :} & Using QtTest library 5.2.1, Qt 5.2.1   \\
\hline \emph{Résultat :} & PASS   \\
\hline 
\end{tabular} 
\end{center}

Les résutats de tous les tests sont satisfaisant.\\
Pour effectuer et visualiser les tests de la partie \emph{véhicules} à partir de l'exécutable, il faut rajouter l'argument \emph{-test4}.

********* Start testing of VehicleTabTest *********\\
Config: Using QtTest library 5.2.1, Qt 5.2.1\\
PASS   : VehicleTabTest::initTestCase()\\
PASS   : VehicleTabTest::qshowIfNotHidden()\\
PASS   : VehicleTabTest::loadCss()\\
PASS   : VehicleTabTest::testVehiculeRegistration()\\
PASS   : VehicleTabTest::testVehicule()\\
PASS   : VehicleTabTest::cleanupTestCase()\\
Totals: 6 passed, 0 failed, 0 skipped\\
********* Finished testing of VehicleTabTest *********\\

\section{Tests graphiques de la partie prestataires}

\begin{center}
\newcolumntype{R}[1]{>{\raggedleft\arraybackslash }b{#1}}
\newcolumntype{L}[1]{>{\raggedright\arraybackslash }b{#1}}
\newcolumntype{C}[1]{>{\centering\arraybackslash }b{#1}}
\begin{tabular}{|L{3cm}|L{10cm}|}
\hline \emph{Cas de test :} & Test-Graphique-31  \\
\hline \emph{Titre :} & testWaybillCountryCode   \\
\hline \emph{Objectif :} & Vérifier le bon fonctionnement du champ du code pays du prestataire.   \\
\hline \emph{Procédure :} & Cliquer dans le champ, le remplir avec une chaîne et vérifier que la chaîne correspond   \\
\hline \emph{Données de test :} & On rempli avec la chaîne ``FRA''   \\
\hline \emph{Config :} & Using QtTest library 5.2.1, Qt 5.2.1   \\
\hline \emph{Résultat :} & PASS   \\
\hline 
\end{tabular} 
\end{center}

\begin{center}
\newcolumntype{R}[1]{>{\raggedleft\arraybackslash }b{#1}}
\newcolumntype{L}[1]{>{\raggedright\arraybackslash }b{#1}}
\newcolumntype{C}[1]{>{\centering\arraybackslash }b{#1}}
\begin{tabular}{|L{3cm}|L{10cm}|}
\hline \emph{Cas de test :} & Test-Graphique-32  \\
\hline \emph{Titre :} & testWaybillId   \\
\hline \emph{Objectif :} & Vérifier le bon fonctionnement du champ de l'identifiant du prestataire.   \\
\hline \emph{Procédure :} & Cliquer dans le champ, le remplir avec une chaîne et vérifier que la chaîne correspond   \\
\hline \emph{Données de test :} & On rempli avec la chaîne ``110''   \\
\hline \emph{Config :} & Using QtTest library 5.2.1, Qt 5.2.1   \\
\hline \emph{Résultat :} & PASS   \\
\hline 
\end{tabular} 
\end{center}

\begin{center}
\newcolumntype{R}[1]{>{\raggedleft\arraybackslash }b{#1}}
\newcolumntype{L}[1]{>{\raggedright\arraybackslash }b{#1}}
\newcolumntype{C}[1]{>{\centering\arraybackslash }b{#1}}
\begin{tabular}{|L{3cm}|L{10cm}|}
\hline \emph{Cas de test :} & Test-Graphique-33  \\
\hline \emph{Titre :} & testWaybillTransportVehicle   \\
\hline \emph{Objectif :} & Vérifier le bon fonctionnement du champ de l'identifiant du véhicule du prestataire.   \\
\hline \emph{Procédure :} & Cliquer dans le champ, le remplir avec une chaîne et vérifier que la chaîne correspond   \\
\hline \emph{Données de test :} & On rempli avec la chaîne ``vh120''   \\
\hline \emph{Config :} & Using QtTest library 5.2.1, Qt 5.2.1   \\
\hline \emph{Résultat :} & PASS   \\
\hline 
\end{tabular} 
\end{center}

\begin{center}
\newcolumntype{R}[1]{>{\raggedleft\arraybackslash }b{#1}}
\newcolumntype{L}[1]{>{\raggedright\arraybackslash }b{#1}}
\newcolumntype{C}[1]{>{\centering\arraybackslash }b{#1}}
\begin{tabular}{|L{3cm}|L{10cm}|}
\hline \emph{Cas de test :} & Test-Graphique-34  \\
\hline \emph{Titre :} & testWaybillTransportRegistration   \\
\hline \emph{Objectif :} & Vérifier le bon fonctionnement du champ de la plaque d'immatriculation du véhicule du prestataire.   \\
\hline \emph{Procédure :} & Cliquer dans le champ, le remplir avec une chaîne et vérifier que la chaîne correspond   \\
\hline \emph{Données de test :} & On rempli avec la chaîne ``56''   \\
\hline \emph{Config :} & Using QtTest library 5.2.1, Qt 5.2.1   \\
\hline \emph{Résultat :} & PASS   \\
\hline 
\end{tabular} 
\end{center}

Les résutats de tous les tests sont satisfaisant.\\
Pour effectuer et visualiser les tests de la partie \emph{prestataires} à partir de l'exécutable, il faut rajouter l'argument \emph{-test6}.

********* Start testing of ProviderTabTest *********\\
Config: Using QtTest library 5.2.1, Qt 5.2.1\\
PASS   : ProviderTabTest::initTestCase()\\
PASS   : ProviderTabTest::qshowIfNotHidden()\\
PASS   : ProviderTabTest::loadCss()\\
PASS   : ProviderTabTest::testWaybillCountryCode()\\
PASS   : ProviderTabTest::testWaybillId()\\
PASS   : ProviderTabTest::testWaybillTransportVehicle()\\
PASS   : ProviderTabTest::testWaybillTransportRegistration()\\
PASS   : ProviderTabTest::cleanupTestCase()\\
Totals: 8 passed, 0 failed, 0 skipped\\
********* Finished testing of ProviderTabTest *********\\

\section{Tests graphiques du status}

\begin{center}
\newcolumntype{R}[1]{>{\raggedleft\arraybackslash }b{#1}}
\newcolumntype{L}[1]{>{\raggedright\arraybackslash }b{#1}}
\newcolumntype{C}[1]{>{\centering\arraybackslash }b{#1}}
\begin{tabular}{|L{3cm}|L{10cm}|}
\hline \emph{Cas de test :} & Test-Graphique-35  \\
\hline \emph{Titre :} & testComment   \\
\hline \emph{Objectif :} & Vérifier le bon fonctionnement du champ du commentaire dans le champ status.   \\
\hline \emph{Procédure :} & Cliquer dans le champ, le remplir avec une chaîne et vérifier que la chaîne correspond   \\
\hline \emph{Données de test :} & On rempli avec la chaîne ``Blablabla''   \\
\hline \emph{Config :} & Using QtTest library 5.2.1, Qt 5.2.1   \\
\hline \emph{Résultat :} & PASS   \\
\hline 
\end{tabular} 
\end{center}

Les résultats de tous les tests sont satisfaisant.\\
Pour effectuer et visualiser les tests du \emph{status} à partir de l'exécutable, il faut rajouter l'argument \emph{-test3}.

********* Start testing of StatusWidgetGroupTest *********\\
Config: Using QtTest library 5.2.1, Qt 5.2.1\\
PASS   : StatusWidgetGroupTest::initTestCase()\\
PASS   : StatusWidgetGroupTest::qshowIfNotHidden()\\
PASS   : StatusWidgetGroupTest::loadCss()\\
PASS   : StatusWidgetGroupTest::testComment()\\
PASS   : StatusWidgetGroupTest::cleanupTestCase()\\
Totals: 5 passed, 0 failed, 0 skipped\\


\chapter{Tests unitaires}
\section{LDAP}
L'annuaire LDAP joue plusieurs rôles dans ce projet. Il permet l'authentification, ainsi que de connaître les droits d'une personne.
Dans un premier temps, les tests unitaires concernant l'annuaire ont été fait sur une classe \emph{mock}, qui permettait de simuler l'existence de l'annuaire en attendant que celui-ci soit opérationnel. La majorité des tests ont ensuite été utiles pour révéler plusieurs erreurs dans l'implémentation du LDAP qui ont pu être corrigé.
\begin{center}
\newcolumntype{R}[1]{>{\raggedleft\arraybackslash }b{#1}}
\newcolumntype{L}[1]{>{\raggedright\arraybackslash }b{#1}}
\newcolumntype{C}[1]{>{\centering\arraybackslash }b{#1}}
\begin{tabular}{|L{3cm}|L{10cm}|}
\hline \emph{Cas de test :} & Test-Unitaire-01  \\
\hline \emph{Titre :} & testConnection   \\
\hline \emph{Objectif :} & Vérifier une connexion à l'annuaire LDAP   \\
\hline \emph{Procédure :} & Initialisation d'une connexion avec un nom d'hôte et un port existant   \\
\hline \emph{Données de test :} & hostname : localhost, port : 389   \\
\hline \emph{Résultat :} & PASS   \\
\hline 
\end{tabular} 
\end{center}

\begin{center}
\newcolumntype{R}[1]{>{\raggedleft\arraybackslash }b{#1}}
\newcolumntype{L}[1]{>{\raggedright\arraybackslash }b{#1}}
\newcolumntype{C}[1]{>{\centering\arraybackslash }b{#1}}
\begin{tabular}{|L{3cm}|L{10cm}|}
\hline \emph{Cas de test :} & Test-Unitaire-02  \\
\hline \emph{Titre :} & testConnectionBadHostname   \\
\hline \emph{Objectif :} & Vérifier une connexion à l'annuaire LDAP   \\
\hline \emph{Procédure :} & Initialisation d'une connexion avec un faux nom d'hôte et un port existant   \\
\hline \emph{Données de test :} & hostname : toto, port : 389   \\
\hline \emph{Résultat :} & FAIL comme prévu   \\
\hline 
\end{tabular} 
\end{center}

\begin{center}
\newcolumntype{R}[1]{>{\raggedleft\arraybackslash }b{#1}}
\newcolumntype{L}[1]{>{\raggedright\arraybackslash }b{#1}}
\newcolumntype{C}[1]{>{\centering\arraybackslash }b{#1}}
\begin{tabular}{|L{3cm}|L{10cm}|}
\hline \emph{Cas de test :} & Test-Unitaire-03  \\
\hline \emph{Titre :} & testConnectionBadPort   \\
\hline \emph{Objectif :} & Vérifier une connexion à l'annuaire LDAP   \\
\hline \emph{Procédure :} & Initialisation d'une connexion avec un nom d'hôte et un faux port \\
\hline \emph{Données de test :} & hostname : localhost, port : 666   \\
\hline \emph{Résultat :} & FAIL comme prévu   \\
\hline 
\end{tabular} 
\end{center}
  
\begin{center}
\newcolumntype{R}[1]{>{\raggedleft\arraybackslash }b{#1}}
\newcolumntype{L}[1]{>{\raggedright\arraybackslash }b{#1}}
\newcolumntype{C}[1]{>{\centering\arraybackslash }b{#1}}
\begin{tabular}{|L{3cm}|L{10cm}|}
\hline \emph{Cas de test :} & Test-Unitaire-04  \\
\hline \emph{Titre :} & testConnectionBadHostnameAndPort   \\
\hline \emph{Objectif :} & Vérifier une connexion à l'annuaire LDAP   \\
\hline \emph{Procédure :} & Initialisation d'une connexion avec un faux nom d'hôte et un faux port \\
\hline \emph{Données de test :} & hostname : toto, port : 666   \\
\hline \emph{Résultat :} & FAIL comme prévu   \\
\hline 
\end{tabular} 
\end{center}  

\begin{center}
\newcolumntype{R}[1]{>{\raggedleft\arraybackslash }b{#1}}
\newcolumntype{L}[1]{>{\raggedright\arraybackslash }b{#1}}
\newcolumntype{C}[1]{>{\centering\arraybackslash }b{#1}}
\begin{tabular}{|L{3cm}|L{10cm}|}
\hline \emph{Cas de test :} & Test-Unitaire-05  \\
\hline \emph{Titre :} & testAuthentification   \\
\hline \emph{Objectif :} & S'authentifier à l'annuaire LDAP   \\
\hline \emph{Procédure :} & Authentification avec un login et son mot de passe correspondant \\
\hline \emph{Données de test :} & login : biensuanga, password : biensuanga   \\
\hline \emph{Résultat :} & PASS   \\
\hline 
\end{tabular} 
\end{center}  

\begin{center}
\newcolumntype{R}[1]{>{\raggedleft\arraybackslash }b{#1}}
\newcolumntype{L}[1]{>{\raggedright\arraybackslash }b{#1}}
\newcolumntype{C}[1]{>{\centering\arraybackslash }b{#1}}
\begin{tabular}{|L{3cm}|L{10cm}|}
\hline \emph{Cas de test :} & Test-Unitaire-06  \\
\hline \emph{Titre :} & testAuthentificationBadUid   \\
\hline \emph{Objectif :} & S'authentifier à l'annuaire LDAP   \\
\hline \emph{Procédure :} & Authentification avec un mauvais login et son mot de passe correspondant \\
\hline \emph{Données de test :} & login : toto, password : biensuanga   \\
\hline \emph{Résultat :} & FAIL comme prévu   \\
\hline 
\end{tabular} 
\end{center}  

\begin{center}
\newcolumntype{R}[1]{>{\raggedleft\arraybackslash }b{#1}}
\newcolumntype{L}[1]{>{\raggedright\arraybackslash }b{#1}}
\newcolumntype{C}[1]{>{\centering\arraybackslash }b{#1}}
\begin{tabular}{|L{3cm}|L{10cm}|}
\hline \emph{Cas de test :} & Test-Unitaire-07  \\
\hline \emph{Titre :} & testAuthentificationBadPwd   \\
\hline \emph{Objectif :} & S'authentifier à l'annuaire LDAP   \\
\hline \emph{Procédure :} & Authentification avec un login et un mot de passe erroné  \\
\hline \emph{Données de test :} & login : biensuanga, password : pwd   \\
\hline \emph{Résultat :} & FAIL comme prévu   \\
\hline 
\end{tabular} 
\end{center} 

\begin{center}
\newcolumntype{R}[1]{>{\raggedleft\arraybackslash }b{#1}}
\newcolumntype{L}[1]{>{\raggedright\arraybackslash }b{#1}}
\newcolumntype{C}[1]{>{\centering\arraybackslash }b{#1}}
\begin{tabular}{|L{3cm}|L{10cm}|}
\hline \emph{Cas de test :} & Test-Unitaire-08  \\
\hline \emph{Titre :} & testAuthentificationBadUidAndPwd   \\
\hline \emph{Objectif :} & S'authentifier à l'annuaire LDAP   \\
\hline \emph{Procédure :} & Authentification avec un mauvais login et un mot de passe erroné  \\
\hline \emph{Données de test :} & login : toto, password : titi   \\
\hline \emph{Résultat :} & FAIL comme prévu   \\
\hline 
\end{tabular} 
\end{center} 

\begin{center}
\newcolumntype{R}[1]{>{\raggedleft\arraybackslash }b{#1}}
\newcolumntype{L}[1]{>{\raggedright\arraybackslash }b{#1}}
\newcolumntype{C}[1]{>{\centering\arraybackslash }b{#1}}
\begin{tabular}{|L{3cm}|L{10cm}|}
\hline \emph{Cas de test :} & Test-Unitaire-09  \\
\hline \emph{Titre :} & testCanRead   \\
\hline \emph{Objectif :} & Demander si une personne a le droit de lecture sur le champ d'une table précise   \\
\hline \emph{Procédure :} & Vérifier les droits de lecture pour une personne ayant effectivement ces droits dans l'annuaire  \\
\hline \emph{Données de test :} & login : biensuanga, table : Waybill, champ : RequisitionId   \\
\hline \emph{Résultat :} & PASS   \\
\hline 
\end{tabular} 
\end{center} 

\begin{center}
\newcolumntype{R}[1]{>{\raggedleft\arraybackslash }b{#1}}
\newcolumntype{L}[1]{>{\raggedright\arraybackslash }b{#1}}
\newcolumntype{C}[1]{>{\centering\arraybackslash }b{#1}}
\begin{tabular}{|L{3cm}|L{10cm}|}
\hline \emph{Cas de test :} & Test-Unitaire-10  \\
\hline \emph{Titre :} & testCanReadBadUid   \\
\hline \emph{Objectif :} & Demander si une personne a le droit de lecture sur le champ d'une table précise   \\
\hline \emph{Procédure :} & Vérifier les droits de lecture pour une personne n'ayant pas ces droits dans l'annuaire  \\
\hline \emph{Données de test :} & login : toto, table : Waybill, champ : RequisitionId   \\
\hline \emph{Résultat :} & FAIL comme prévu   \\
\hline 
\end{tabular} 
\end{center} 

\begin{center}
\newcolumntype{R}[1]{>{\raggedleft\arraybackslash }b{#1}}
\newcolumntype{L}[1]{>{\raggedright\arraybackslash }b{#1}}
\newcolumntype{C}[1]{>{\centering\arraybackslash }b{#1}}
\begin{tabular}{|L{3cm}|L{10cm}|}
\hline \emph{Cas de test :} & Test-Unitaire-11  \\
\hline \emph{Titre :} & testCanReadBadTableName   \\
\hline \emph{Objectif :} & Demander si une personne a le droit de lecture sur le champ d'une table précise   \\
\hline \emph{Procédure :} & Vérifier les droits de lecture pour une personne ayant ces droits dans l'annuaire mais sur une table inexistante \\
\hline \emph{Données de test :} & login : biensuanga, table : avion, champ : RequisitionId   \\
\hline \emph{Résultat :} & FAIL comme prévu   \\
\hline 
\end{tabular} 
\end{center} 

\begin{center}
\newcolumntype{R}[1]{>{\raggedleft\arraybackslash }b{#1}}
\newcolumntype{L}[1]{>{\raggedright\arraybackslash }b{#1}}
\newcolumntype{C}[1]{>{\centering\arraybackslash }b{#1}}
\begin{tabular}{|L{3cm}|L{10cm}|}
\hline \emph{Cas de test :} & Test-Unitaire-12  \\
\hline \emph{Titre :} & testCanReadBadFieldName   \\
\hline \emph{Objectif :} & Demander si une personne a le droit de lecture sur le champ d'une table précise   \\
\hline \emph{Procédure :} & Vérifier les droits de lecture pour une personne ayant ces droits dans l'annuaire mais sur un champ inexistant \\
\hline \emph{Données de test :} & login : biensuanga, table : Waybill, champ : waybillColor   \\
\hline \emph{Résultat :} & FAIL comme prévu   \\
\hline 
\end{tabular} 
\end{center} 

\begin{center}
\newcolumntype{R}[1]{>{\raggedleft\arraybackslash }b{#1}}
\newcolumntype{L}[1]{>{\raggedright\arraybackslash }b{#1}}
\newcolumntype{C}[1]{>{\centering\arraybackslash }b{#1}}
\begin{tabular}{|L{3cm}|L{10cm}|}
\hline \emph{Cas de test :} & Test-Unitaire-13  \\
\hline \emph{Titre :} & testCanWrite   \\
\hline \emph{Objectif :} & Demander si une personne a le droit d'écriture sur le champ d'une table précise   \\
\hline \emph{Procédure :} & Vérifier les droits d'écriture pour une personne ayant ces droits dans l'annuaire \\
\hline \emph{Données de test :} & login : biensuanga, table : Requistion, champ : waybillId   \\
\hline \emph{Résultat :} & PASS   \\
\hline 
\end{tabular} 
\end{center} 

\begin{center}
\newcolumntype{R}[1]{>{\raggedleft\arraybackslash }b{#1}}
\newcolumntype{L}[1]{>{\raggedright\arraybackslash }b{#1}}
\newcolumntype{C}[1]{>{\centering\arraybackslash }b{#1}}
\begin{tabular}{|L{3cm}|L{10cm}|}
\hline \emph{Cas de test :} & Test-Unitaire-14  \\
\hline \emph{Titre :} & testCanWriteBadUid   \\
\hline \emph{Objectif :} &Demander si une personne a le droit d'écriture sur le champ d'une table précise    \\
\hline \emph{Procédure :} & Vérifier les droits d'écriture pour une personne n'ayant pas ces droits dans l'annuaire\\
\hline \emph{Données de test :} & login : toto, table : Requistion, champ : waybillId   \\
\hline \emph{Résultat :} & FAIL comme prévu   \\
\hline 
\end{tabular} 
\end{center} 

\begin{center}
\newcolumntype{R}[1]{>{\raggedleft\arraybackslash }b{#1}}
\newcolumntype{L}[1]{>{\raggedright\arraybackslash }b{#1}}
\newcolumntype{C}[1]{>{\centering\arraybackslash }b{#1}}
\begin{tabular}{|L{3cm}|L{10cm}|}
\hline \emph{Cas de test :} & Test-Unitaire-15  \\
\hline \emph{Titre :} & testCanWriteBadTableName   \\
\hline \emph{Objectif :} &Demander si une personne a le droit d'écriture sur le champ d'une table précise    \\
\hline \emph{Procédure :} & Vérifier les droits d'écriture pour une personne ayant ces droits dans l'annuaire mais sur une table inexistante \\
\hline \emph{Données de test :} & login : biensuanga, table : Avion, champ : waybillId   \\
\hline \emph{Résultat :} & FAIL comme prévu   \\
\hline 
\end{tabular} 
\end{center} 

\begin{center}
\newcolumntype{R}[1]{>{\raggedleft\arraybackslash }b{#1}}
\newcolumntype{L}[1]{>{\raggedright\arraybackslash }b{#1}}
\newcolumntype{C}[1]{>{\centering\arraybackslash }b{#1}}
\begin{tabular}{|L{3cm}|L{10cm}|}
\hline \emph{Cas de test :} & Test-Unitaire-16  \\
\hline \emph{Titre :} & testCanWriteBadFieldName   \\
\hline \emph{Objectif :} &Demander si une personne a le droit d'écriture sur le champ d'une table précise    \\
\hline \emph{Procédure :} & Vérifier les droits d'écriture pour une personne ayant ces droits dans l'annuaire mais sur un champ inexistant \\
\hline \emph{Données de test :} & login : biensuanga, table : Requistion, champ : PersonId4   \\
\hline \emph{Résultat :} & FAIL comme prévu   \\
\hline 
\end{tabular} 
\end{center} 

\begin{center}
\newcolumntype{R}[1]{>{\raggedleft\arraybackslash }b{#1}}
\newcolumntype{L}[1]{>{\raggedright\arraybackslash }b{#1}}
\newcolumntype{C}[1]{>{\centering\arraybackslash }b{#1}}
\begin{tabular}{|L{3cm}|L{10cm}|}
\hline \emph{Cas de test :} & Test-Unitaire-17  \\
\hline \emph{Titre :} & testCanAdd   \\
\hline \emph{Objectif :} &Demander si une personne a le droit d'ajout sur une table précise    \\
\hline \emph{Procédure :} & Vérifier les droits d'ajout pour une personne ayant ces droits dans l'annuaire \\
\hline \emph{Données de test :} & login : biensuanga, table : Waybill \\
\hline \emph{Résultat :} & PASS   \\
\hline 
\end{tabular} 
\end{center} 

\begin{center}
\newcolumntype{R}[1]{>{\raggedleft\arraybackslash }b{#1}}
\newcolumntype{L}[1]{>{\raggedright\arraybackslash }b{#1}}
\newcolumntype{C}[1]{>{\centering\arraybackslash }b{#1}}
\begin{tabular}{|L{3cm}|L{10cm}|}
\hline \emph{Cas de test :} & Test-Unitaire-18  \\
\hline \emph{Titre :} & testCanAddBadUid   \\
\hline \emph{Objectif :} &Demander si une personne a le droit d'ajout sur une table précise    \\
\hline \emph{Procédure :} & Vérifier les droits d'ajout pour une personne n'ayant pas ces droits dans l'annuaire \\
\hline \emph{Données de test :} & login : toto, table : Waybill \\
\hline \emph{Résultat :} & FAIL comme prévu   \\
\hline 
\end{tabular} 
\end{center} 

\begin{center}
\newcolumntype{R}[1]{>{\raggedleft\arraybackslash }b{#1}}
\newcolumntype{L}[1]{>{\raggedright\arraybackslash }b{#1}}
\newcolumntype{C}[1]{>{\centering\arraybackslash }b{#1}}
\begin{tabular}{|L{3cm}|L{10cm}|}
\hline \emph{Cas de test :} & Test-Unitaire-19  \\
\hline \emph{Titre :} & testCanAddBadTableName   \\
\hline \emph{Objectif :} &Demander si une personne a le droit d'ajout sur une table précise    \\
\hline \emph{Procédure :} & Vérifier les droits d'ajout pour une personne ayant ces droits dans l'annuaire mais sur une table interdisant l'ajout de ligne \\
\hline \emph{Données de test :} & login : biensuanga, table : Requistion \\
\hline \emph{Résultat :} & FAIL comme prévu   \\
\hline 
\end{tabular} 
\end{center} 

\begin{center}
\newcolumntype{R}[1]{>{\raggedleft\arraybackslash }b{#1}}
\newcolumntype{L}[1]{>{\raggedright\arraybackslash }b{#1}}
\newcolumntype{C}[1]{>{\centering\arraybackslash }b{#1}}
\begin{tabular}{|L{3cm}|L{10cm}|}
\hline \emph{Cas de test :} & Test-Unitaire-20  \\
\hline \emph{Titre :} & testCanDelete   \\
\hline \emph{Objectif :} &Demander si une personne a le droit de suppression sur une table précise    \\
\hline \emph{Procédure :} & Vérifier les droits de suppression pour une personne ayant ces droits dans l'annuaire \\
\hline \emph{Données de test :} & login : biensuanga, table : Waybill \\
\hline \emph{Résultat :} & PASS   \\
\hline 
\end{tabular} 
\end{center} 

\begin{center}
\newcolumntype{R}[1]{>{\raggedleft\arraybackslash }b{#1}}
\newcolumntype{L}[1]{>{\raggedright\arraybackslash }b{#1}}
\newcolumntype{C}[1]{>{\centering\arraybackslash }b{#1}}
\begin{tabular}{|L{3cm}|L{10cm}|}
\hline \emph{Cas de test :} & Test-Unitaire-21  \\
\hline \emph{Titre :} & testCanDeleteBadUid   \\
\hline \emph{Objectif :} &Demander si une personne a le droit de suppression sur une table précise    \\
\hline \emph{Procédure :} & Vérifier les droits de suppression pour une personne n'ayant pas ces droits dans l'annuaire \\
\hline \emph{Données de test :} & login : toto, table : Waybill \\
\hline \emph{Résultat :} & FAIL comme prévu   \\
\hline 
\end{tabular} 
\end{center} 

\begin{center}
\newcolumntype{R}[1]{>{\raggedleft\arraybackslash }b{#1}}
\newcolumntype{L}[1]{>{\raggedright\arraybackslash }b{#1}}
\newcolumntype{C}[1]{>{\centering\arraybackslash }b{#1}}
\begin{tabular}{|L{3cm}|L{10cm}|}
\hline \emph{Cas de test :} & Test-Unitaire-22  \\
\hline \emph{Titre :} & testCanDeleteBadTableName   \\
\hline \emph{Objectif :} &Demander si une personne a le droit de suppression sur une table précise    \\
\hline \emph{Procédure :} & Vérifier les droits de suppression pour une personne ayant ces droits dans l'annuaire mais sur une table interdisant la suppression\\
\hline \emph{Données de test :} & login : biensuanga, table : Requistion \\
\hline \emph{Résultat :} & FAIL comme prévu   \\
\hline 
\end{tabular} 
\end{center} 

\begin{center}
\newcolumntype{R}[1]{>{\raggedleft\arraybackslash }b{#1}}
\newcolumntype{L}[1]{>{\raggedright\arraybackslash }b{#1}}
\newcolumntype{C}[1]{>{\centering\arraybackslash }b{#1}}
\begin{tabular}{|L{3cm}|L{10cm}|}
\hline \emph{Cas de test :} & Test-Unitaire-23  \\
\hline \emph{Titre :} & testCanReadGroup()   \\
\hline \emph{Objectif :} &Demander si une personne a le droit de lecture sur une table précise alors qu'elle ne le possède pas directement mais via son groupe    \\
\hline \emph{Procédure :} & Vérifier les droits de lecture pour une personne n'ayant ces droits personnellement mais via son groupe \\
\hline \emph{Données de test :} & login : khady, table : Provider, champ : ProviderId \\
\hline \emph{Résultat :} & PASS   \\
\hline 
\end{tabular} 
\end{center} 

\begin{center}
\newcolumntype{R}[1]{>{\raggedleft\arraybackslash }b{#1}}
\newcolumntype{L}[1]{>{\raggedright\arraybackslash }b{#1}}
\newcolumntype{C}[1]{>{\centering\arraybackslash }b{#1}}
\begin{tabular}{|L{3cm}|L{10cm}|}
\hline \emph{Cas de test :} & Test-Unitaire-24  \\
\hline \emph{Titre :} & testCanReadGroup2()   \\
\hline \emph{Objectif :} &Demander si une personne a le droit de lecture sur une table précise alors qu'elle a une restriction personnelle sur celle-ci    \\
\hline \emph{Procédure :} & Vérifier les droits de lecture pour une personne ayant une restriction sur cette table\\
\hline \emph{Données de test :} & login : lucien, table : Vehicle, champ : VehicleId \\
\hline \emph{Résultat :} & FAIL comme prévu   \\
\hline 
\end{tabular} 
\end{center} 

Tous ces tests ont permis de corriger des erreurs dans l'implémentation de l'annuaire.\\
Pour effectuer et visualiser les tests concernant \emph{LDAP} à partir de l'exécutable, il faut rajouter l'argument \emph{-testldap}.

********* Start testing of LdapTest *********\\
Config: Using QtTest library 5.2.1, Qt 5.2.1\\
PASS   : LdapTest::initTestCase()\\
PASS   : LdapTest::qshowIfNotHidden()\\
PASS   : LdapTest::testConnection()\\
XPASS  : LdapTest::testConnectionBadHostname() 'lm.initialize(QString("toto"), 389)' returned TRUE unexpectedly. ()\\
XPASS  : LdapTest::testConnectionBadPort() 'lm.initialize(QString("10.42.0.20"), 666)' returned TRUE unexpectedly. ()\\
XPASS  : LdapTest::testConnectionBadHostnameAndPort() 'lm.initialize(QString("toto"), 666)' returned TRUE unexpectedly. ()\\
PASS   : LdapTest::testAuthentification()\\
XFAIL  : LdapTest::testAuthentificationBadUid() bad login\\
PASS   : LdapTest::testAuthentificationBadUid()\\
XFAIL  : LdapTest::testAuthentificationBadPwd() bad password\\
PASS   : LdapTest::testAuthentificationBadPwd()\\
XFAIL  : LdapTest::testAuthentificationBadUidAndPwd() bad login and password\\
PASS   : LdapTest::testAuthentificationBadUidAndPwd()\\
PASS   : LdapTest::testCanRead()\\
PASS   : LdapTest::testCanReadGroup()\\
XFAIL  : LdapTest::testCanReadGroup2() droit supprime à l'user\\
PASS   : LdapTest::testCanReadGroup2()\\
XFAIL  : LdapTest::testCanReadBadUid() bad login\\
PASS   : LdapTest::testCanReadBadUid()\\
XFAIL  : LdapTest::testCanReadBadTableName() unknow table name\\
PASS   : LdapTest::testCanReadBadTableName()\\
XFAIL  : LdapTest::testCanReadBadFieldName() unknow Field name\\
PASS   : LdapTest::testCanReadBadFieldName()\\
PASS   : LdapTest::testCanWrite()\\
XFAIL  : LdapTest::testCanWriteBadUid() bad login\\
PASS   : LdapTest::testCanWriteBadUid()\\
XFAIL  : LdapTest::testCanWriteBadTableName() unknow table name\\
PASS   : LdapTest::testCanWriteBadTableName()\\
XFAIL  : LdapTest::testCanWriteBadFieldName() unknow field name\\
PASS   : LdapTest::testCanWriteBadFieldName()\\
PASS   : LdapTest::testCanAdd()\\
XFAIL  : LdapTest::testCanAddBadUid() bad login\\
PASS   : LdapTest::testCanAddBadUid()\\
XFAIL  : LdapTest::testCanAddBadTableName() unknow table name\\
PASS   : LdapTest::testCanAddBadTableName()\\
PASS   : LdapTest::testCanDelete()\\
XFAIL  : LdapTest::testCanDeleteBadUid() bad login\\
PASS   : LdapTest::testCanDeleteBadUid()\\
XFAIL  : LdapTest::testCanDeleteBadTableName() unknow table name\\
PASS   : LdapTest::testCanDeleteBadTableName()\\
PASS   : LdapTest::cleanupTestCase()\\
Totals: 24 passed, 3 failed, 0 skipped\\
********* Finished testing of LdapTest *********\\

\section{Gestion de données}
Afin de tester les intéractions avec la base de données, des tests unitaires ont été réalisé sur la classe de Réquisition qui permet d'interagir avec la base de données en la consultant et en la modifiant.

\begin{center}
\newcolumntype{R}[1]{>{\raggedleft\arraybackslash }b{#1}}
\newcolumntype{L}[1]{>{\raggedright\arraybackslash }b{#1}}
\newcolumntype{C}[1]{>{\centering\arraybackslash }b{#1}}
\begin{tabular}{|L{3cm}|L{10cm}|}
\hline \emph{Cas de test :} & Test-Unitaire-25  \\
\hline \emph{Titre :} & testGetRequisition()   \\
\hline \emph{Objectif :} &Afficher tous les champs de la table Réquisition en fonction d'un identifiant.    \\
\hline \emph{Procédure :} & Vérifier que les résultats de la requête correspondent aux données stockées dans la table\\
\hline \emph{Données de test :} & id : 002 \\
\hline \emph{Résultat :} & PASS   \\
\hline 
\end{tabular} 
\end{center} 

\begin{center}
\newcolumntype{R}[1]{>{\raggedleft\arraybackslash }b{#1}}
\newcolumntype{L}[1]{>{\raggedright\arraybackslash }b{#1}}
\newcolumntype{C}[1]{>{\centering\arraybackslash }b{#1}}
\begin{tabular}{|L{3cm}|L{10cm}|}
\hline \emph{Cas de test :} & Test-Unitaire-26  \\
\hline \emph{Titre :} & testGetList()   \\
\hline \emph{Objectif :} &Afficher tous les champs de la table Réquisition pour tous les identifiants.    \\
\hline \emph{Procédure :} & Vérifier que les résultats de la requête correspondent aux données stockées dans la table\\
\hline \emph{Données de test :} &  \\
\hline \emph{Résultat :} & PASS   \\
\hline 
\end{tabular} 
\end{center} 

\begin{center}
\newcolumntype{R}[1]{>{\raggedleft\arraybackslash }b{#1}}
\newcolumntype{L}[1]{>{\raggedright\arraybackslash }b{#1}}
\newcolumntype{C}[1]{>{\centering\arraybackslash }b{#1}}
\begin{tabular}{|L{3cm}|L{10cm}|}
\hline \emph{Cas de test :} & Test-Unitaire-27  \\
\hline \emph{Titre :} & testAdd()   \\
\hline \emph{Objectif :} &Ajouter une ligne dans la table Réquisition.    \\
\hline \emph{Procédure :} & Ajouter une ligne dans la table Réquisition, vérifier que la table correspond aux changements générés par l'outil.\\
\hline \emph{Données de test :} & Tous les champs sont remplis \\
\hline \emph{Résultat :} & PASS   \\
\hline 
\end{tabular} 
\end{center} 

\begin{center}
\newcolumntype{R}[1]{>{\raggedleft\arraybackslash }b{#1}}
\newcolumntype{L}[1]{>{\raggedright\arraybackslash }b{#1}}
\newcolumntype{C}[1]{>{\centering\arraybackslash }b{#1}}
\begin{tabular}{|L{3cm}|L{10cm}|}
\hline \emph{Cas de test :} & Test-Unitaire-28  \\
\hline \emph{Titre :} & testBadAdd()   \\
\hline \emph{Objectif :} &Vérification de l'ajout d'une ligne avec des champs à ``NULL'' dans la table Réquisition.    \\
\hline \emph{Procédure :} & Ajouter une ligne avec des champs à ``NULL'' et vérifier qu'une erreur est générée\\
\hline \emph{Données de test :} & Tous les champs sont remplis sauf l'Id \\
\hline \emph{Résultat :} & FAIL comme prévu   \\
\hline 
\end{tabular} 
\end{center} 

Ce test a été répété avec toutes les combinaisons possibles.


\end{document}