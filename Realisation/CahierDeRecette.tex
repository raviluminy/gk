  %-------------------------------------------------------------------------------
  %
  %     CHARGEMENT DES EXTENSIONS
  %
  %-------------------------------------------------------------------------------

  \documentclass[11pt,fleqn]{report}
  \usepackage{GarmirKhatch}

  %-------------------------------------------------------------------------------
  %     Informations spécifiques au document
  %-------------------------------------------------------------------------------
  \ZTitle{Système de gestion des transports}
  \ZSubject{Cahier de recette}
  \ZVersion{0.1}
  \ZDate{\today}
  \ZAuthor{\Balde,\\\Cadon,\\\Gairoard,\\\Julien,\\\Lericolais,\\\Mezelle,\\\Pachy,\\\SuangaWeto,\\\Toure}

  %-------------------------------------------------------------------------------
  %     Contenu
  %-------------------------------------------------------------------------------


  \begin{document}
  
  \ZMakeCover

  \ZMakeInformations{
	  % Historique des modifications
	  % Version & Date & Auteur(s) & Modification(s)
	  0.1 & 2014-03-15 & \Julien & Création \\
	  \midrule
  }{
	  % Historique des approbations
	  % Version & Date & Approbateur(s)
	  \ZVersion & - & - \\
  }{
	  % Historique des validations
	  % Version & Date & Responsable(s)
	  0.1 & - & - \\
  }


  \chapter{Introduction}

  \section{Objectif}
  Ce document propose une série de scénarios décrivant avec précision les démarches à suivre dans le cadre de l’utilisation du
  logiciel de gestion des transports. Il sert de support à la validation du logiciel lors de la recette auprès du client.

  \chapter{Tests fonctionnels}
  \section{Gestion des personnes}
  \subsection{Scénario 1}
  \begin{center}
  \newcolumntype{R}[1]{>{\raggedleft\arraybackslash }b{#1}}
  \newcolumntype{L}[1]{>{\raggedright\arraybackslash }b{#1}}
  \newcolumntype{C}[1]{>{\centering\arraybackslash }b{#1}}
  \begin{tabular}{|L{3cm}|L{10cm}|}
  \hline \textbf{Cas de test :} & Test-01  \\
  \hline \textbf{Titre :} & Création d'une personne   \\
  \hline \textbf{Objectif :} & Vérifier le cas de la création d'une personne à partir du logiciel   \\
  \hline \textbf{Procédure :} & Création d'une personne à l'aide la méthode blabla   \\
  \hline \textbf{Données de test :} & toto   \\
  \hline \textbf{Résultat :} & copié collé   \\
  \hline 
  \end{tabular} 
  \end{center}

  \chapter{Validation de la recette}
  \end{document}