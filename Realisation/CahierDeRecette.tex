%-------------------------------------------------------------------------------
%
%     CHARGEMENT DES EXTENSIONS
%
%-------------------------------------------------------------------------------
\documentclass[11pt,fleqn]{report}
\usepackage{GarmirKhatch}
%-------------------------------------------------------------------------------
%     Informations spécifiques au document
%-------------------------------------------------------------------------------
\ZTitle{Système de gestion des transports}
\ZSubject{Cahier de recette}
\ZVersion{0.2}
\ZDate{\today}
\ZAuthor{\Balde,\\\Cadon,\\\Gairoard,\\\Julien,\\\Lericolais,\\\Mezelle,\\\Pachy,\\\SuangaWeto,\\\Toure}
%-------------------------------------------------------------------------------
%     Contenu
%-------------------------------------------------------------------------------


\begin{document}
    
\ZMakeCover

\ZMakeInformations{
	    % Historique des modifications
	    % Version & Date & Auteur(s) & Modification(s)
	    0.1 & 2014-03-15 & \Julien & Création \\
	    0.2 & 2014-03-19 & \Julien & Réalisatino des tests de l'interface graphique \\
	    \midrule
    }{
	    % Historique des approbations
	    % Version & Date & Approbateur(s)
	    \ZVersion & - & - \\
    }{
	    % Historique des validations
	    % Version & Date & Responsable(s)
	    0.1 & - & - \\
    }


\chapter{Introduction}
\section{Objectif}
Ce document propose une série de tests et de scénarios décrivant avec précision les démarches à suivre dans le cadre de l’utilisation du logiciel de gestion des transports. Il sert de support à la validation du logiciel lors de la recette auprès du client.
Différents types de tests ont été réalisé : 
\begin{itemize}
\item les tests sur l'interface graphique,
\item les tests unitaires,
\item les tests sur le système de gestion de données.
\end{itemize}
Les tests unitaires ont été fait sur l’environnement de développement. 
Au cours de l’évolution du développement de l'outil, le code et les données de bases ont changés c'est pourquoi les jeux d’essais peuvent être rechargés  et ré initialisés afin de permettre de refaire les mêmes tests dans les mêmes conditions. 
Il est à noter que des tests de régression ont été fait suite à des corrections du programme.

\chapter{Tests sur interface graphique}
L'outil a été développé en c++ à l'aide du framework QT. Ce framework intègre un outil de test grâce à la librairie QTestlib.
C'est donc à l'aide de QTest que les scénarios ci-dessous ont été réalisé.

\section{Test graphique de la partie waybill}
\begin{center}
\newcolumntype{R}[1]{>{\raggedleft\arraybackslash }b{#1}}
\newcolumntype{L}[1]{>{\raggedright\arraybackslash }b{#1}}
\newcolumntype{C}[1]{>{\centering\arraybackslash }b{#1}}
\begin{tabular}{|L{3cm}|L{10cm}|}
\hline \emph{Cas de test :} & Test-Graphique-01  \\
\hline \emph{Titre :} & TestWaybillCountryCode   \\
\hline \emph{Objectif :} & Vérifier le bon fonctionnement du champ du code pays    \\
\hline \emph{Procédure :} & Cliquer dans le champ du code pays, le remplir avec une chaîne et vérifier que la chaîne correspond   \\
\hline \emph{Données de test :} & On rempli avec la chaîne ``Fr''   \\
\hline \emph{Config :} & Using QtTest library 5.2.1, Qt 5.2.1   \\
\hline \emph{Résultat :} & PASS   \\
\hline 
\end{tabular} 
\end{center}

\begin{center}
\newcolumntype{R}[1]{>{\raggedleft\arraybackslash }b{#1}}
\newcolumntype{L}[1]{>{\raggedright\arraybackslash }b{#1}}
\newcolumntype{C}[1]{>{\centering\arraybackslash }b{#1}}
\begin{tabular}{|L{3cm}|L{10cm}|}
\hline \emph{Cas de test :} & Test-Graphique-02  \\
\hline \emph{Titre :} & TestWaybillCountryId   \\
\hline \emph{Objectif :} & Vérifier le bon fonctionnement du champ de l'id du pays    \\
\hline \emph{Procédure :} & Cliquer dans le champ de l'id du pays, le remplir avec une id et vérifier que l'id correspond   \\
\hline \emph{Données de test :} & On rempli avec l'id ``7''   \\
\hline \emph{Config :} & Using QtTest library 5.2.1, Qt 5.2.1   \\
\hline \emph{Résultat :} & PASS   \\
\hline 
\end{tabular} 
\end{center}

\begin{center}
\newcolumntype{R}[1]{>{\raggedleft\arraybackslash }b{#1}}
\newcolumntype{L}[1]{>{\raggedright\arraybackslash }b{#1}}
\newcolumntype{C}[1]{>{\centering\arraybackslash }b{#1}}
\begin{tabular}{|L{3cm}|L{10cm}|}
\hline \emph{Cas de test :} & Test-Graphique-02  \\
\hline \emph{Titre :} & TestWaybillCountryId   \\
\hline \emph{Objectif :} & Vérifier le bon fonctionnement du champ de l'id du pays    \\
\hline \emph{Procédure :} & Cliquer dans le champ de l'id du pays, le remplir avec une id et vérifier que l'id correspond   \\
\hline \emph{Données de test :} & On rempli avec l'id ``7''   \\
\hline \emph{Config :} & Using QtTest library 5.2.1, Qt 5.2.1   \\
\hline \emph{Résultat :} & PASS   \\
\hline 
\end{tabular} 
\end{center}

\begin{center}
\newcolumntype{R}[1]{>{\raggedleft\arraybackslash }b{#1}}
\newcolumntype{L}[1]{>{\raggedright\arraybackslash }b{#1}}
\newcolumntype{C}[1]{>{\centering\arraybackslash }b{#1}}
\begin{tabular}{|L{3cm}|L{10cm}|}
\hline \emph{Cas de test :} & Test-Graphique-03  \\
\hline \emph{Titre :} & TestWaybillWarehouse   \\
\hline \emph{Objectif :} & Vérifier le bon fonctionnement du champ de localisation de l'entrepôt   \\
\hline \emph{Procédure :} & Cliquer dans le champ de localisation de l'entrepôt, le remplir avec une chaîne et vérifier que la chaîne correspond   \\
\hline \emph{Données de test :} & On rempli avec la chaîne ``Marseille''   \\
\hline \emph{Config :} & Using QtTest library 5.2.1, Qt 5.2.1   \\
\hline \emph{Résultat :} & PASS   \\
\hline 
\end{tabular} 
\end{center}


\begin{center}
\newcolumntype{R}[1]{>{\raggedleft\arraybackslash }b{#1}}
\newcolumntype{L}[1]{>{\raggedright\arraybackslash }b{#1}}
\newcolumntype{C}[1]{>{\centering\arraybackslash }b{#1}}
\begin{tabular}{|L{3cm}|L{10cm}|}
\hline \emph{Cas de test :} & Test-Graphique-04  \\
\hline \emph{Titre :} & TestWaybillStatus   \\
\hline \emph{Objectif :} & Vérifier le bon fonctionnement du champ status   \\
\hline \emph{Procédure :} & Vérifier tous les status disponible   \\
\hline \emph{Données de test :} &    \\
\hline \emph{Config :} & Using QtTest library 5.2.1, Qt 5.2.1   \\
\hline \emph{Résultat :} & PASS   \\
\hline 
\end{tabular} 
\end{center}


\begin{center}
\newcolumntype{R}[1]{>{\raggedleft\arraybackslash }b{#1}}
\newcolumntype{L}[1]{>{\raggedright\arraybackslash }b{#1}}
\newcolumntype{C}[1]{>{\centering\arraybackslash }b{#1}}
\begin{tabular}{|L{3cm}|L{10cm}|}
\hline \emph{Cas de test :} & Test-Graphique-05  \\
\hline \emph{Titre :} & TestWaybillComment   \\
\hline \emph{Objectif :} & Vérifier le bon fonctionnement du champ commentaire   \\
\hline \emph{Procédure :} & Cliquer dans le champ du commentaire, saisir une chaîne et vérifier que la chaîne correspond  \\
\hline \emph{Données de test :} & On rempli avec la chaîne ``blabla'' \\
\hline \emph{Config :} & Using QtTest library 5.2.1, Qt 5.2.1   \\
\hline \emph{Résultat :} & PASS   \\
\hline 
\end{tabular} 
\end{center}

\begin{center}
\newcolumntype{R}[1]{>{\raggedleft\arraybackslash }b{#1}}
\newcolumntype{L}[1]{>{\raggedright\arraybackslash }b{#1}}
\newcolumntype{C}[1]{>{\centering\arraybackslash }b{#1}}
\begin{tabular}{|L{3cm}|L{10cm}|}
\hline \emph{Cas de test :} & Test-Graphique-06  \\
\hline \emph{Titre :} & TestWaybillVehicle   \\
\hline \emph{Objectif :} & Vérifier le bon fonctionnement du champ commentaire   \\
\hline \emph{Procédure :} & Cliquer dans le champ du commentaire, saisir une chaîne et vérifier que la chaîne correspond  \\
\hline \emph{Données de test :} & On rempli avec la chaîne ``blabla'' \\
\hline \emph{Config :} & Using QtTest library 5.2.1, Qt 5.2.1   \\
\hline \emph{Résultat :} & PASS   \\
\hline 
\end{tabular} 
\end{center}

\chapter{Tests unitaires}
\section{Gestion des personnes}
\subsection{Scénario 1}
\begin{center}
\newcolumntype{R}[1]{>{\raggedleft\arraybackslash }b{#1}}
\newcolumntype{L}[1]{>{\raggedright\arraybackslash }b{#1}}
\newcolumntype{C}[1]{>{\centering\arraybackslash }b{#1}}
\begin{tabular}{|L{3cm}|L{10cm}|}
\hline \emph{Cas de test :} & Test-01  \\
\hline \emph{Titre :} & Création d'une personne   \\
\hline \emph{Objectif :} & Vérifier le cas de la création d'une personne à partir du logiciel   \\
\hline \emph{Procédure :} & Création d'une personne à l'aide la méthode blabla   \\
\hline \emph{Données de test :} & data   \\
\hline \emph{Résultat :} & PASS   \\
\hline 
\end{tabular} 
\end{center}
  
%\begin{center}
%\newcolumntype{R}[1]{>{\raggedleft\arraybackslash }b{#1}}
%\newcolumntype{L}[1]{>{\raggedright\arraybackslash }b{#1}}
%\newcolumntype{C}[1]{>{\centering\arraybackslash }b{#1}}
%\begin{tabular}{|L{3cm}|L{10cm}|}
%\hline \emph{Cas de test :} & Test-02  \\
%\hline \emph{Titre :} & Création d'une personne   \\
%\hline \emph{Objectif :} & Vérifier le cas de la création d'une personne à partir du logiciel   \\
%\hline \emph{Procédure :} & Création d'une personne à l'aide la méthode blabla   \\
%\hline \emph{Données de test :} & toto   \\
%\hline \emph{Résultat :} & copié collé   \\
%\hline 
%\end{tabular} 
%\end{center}

\chapter{Tests sur le système de gestion de données}

\chapter{Validation de la recette}

\end{document}