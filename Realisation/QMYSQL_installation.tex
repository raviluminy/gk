%-------------------------------------------------------------------------------
%
%     CHARGEMENT DES EXTENSIONS
%
%-------------------------------------------------------------------------------

\documentclass[11pt,fleqn]{report}
\usepackage{GarmirKhatch}
\usepackage{MjMcd}

\definecolor{colone}{RGB}{209,220,204}
\definecolor{coltwo}{RGB}{204,222,210}
\definecolor{colthree}{RGB}{207,233,232}
\definecolor{colfour}{RGB}{248,243,214}
\definecolor{colfive}{RGB}{245,238,197}
\definecolor{colsix}{RGB}{243,235,179}
\definecolor{colseven}{RGB}{241,231,163}


%-------------------------------------------------------------------------------
%     Informations spécifiques au document
%-------------------------------------------------------------------------------

\ZTitle{Système de gestion des transports}
\ZSubject{Manuel d'installation du driver QMYSQL}
\ZVersion{0.1}
\ZDate{\today}


%-------------------------------------------------------------------------------
%     Contenu
%-------------------------------------------------------------------------------

\begin{document}

\ZMakeCover

\ZMakeInformations{
	% Historique des modifications
	% Version & Date & Auteur(s) & Modification(s)
	0.1 & 2014-03-06 & \Gairoard & Rédaction \\
}{
	% Historique des approbations
	% Version & Date & Approbateur(s)
	1.0 & - & - \\
}{
	% Historique des validations
	% Version & Date & Responsable(s)
	1.0 & - & - \\
}

\ZMakeTableOfContents

% ======================================================================= %
%     Ce fichier décrit la procédure d'installation du driver QMYSQL     %
%     dans les environnements Windows, Lunix et Mac OSX.                  %
% ======================================================================= %
%     Auteur : Lionel GAIROARD                                            %
% ----------------------------------------------------------------------- %

\section{Installation manuelle du driver QMYSQL}
\subsection{Comment compiler le plugin QMYSQL sous Unix et Mac OS X}
Vous avez besoin des fichiers d'en-tête de MySQL ainsi que la bibliothèque partagée \textit{libmysqlclient.so}. \\
Selon votre distribution Linux, vous devez peut-être installer un paquet qui est généralement appelé "mysql-dev".
\\
Dites à \textbf{qmake} où trouver les fichiers d'en-tête de MySQL et les bibliothèques partagées (ici, on suppose que MySQL est installé dans /usr/local) et lancez make :
\begin{lstlisting}
cd $QTDIR/src/plugins/sqldrivers/mysql
qmake "INCLUDEPATH+=/usr/local/include" "LIBS+=-L/usr/local/lib -lmysqlclient_r" mysql.pro
make
\end{lstlisting}

Après l'installation de Qt, vous devez également installer le plugin dans l'emplacement standard :
\begin{lstlisting}
cd $QTDIR/src/plugins/sqldrivers/mysql
make install
\end{lstlisting}

\subsection{Comment compiler le plugin QMYSQL sous Windows}
 Vous devez obtenir les fichiers d'installation de MySQL. Exécutez \textit{Setup.exe} et sélectionnez "Installation personnalisée". Installez le module "Libs \& Inclure les fichiers". Construire le plugin comme suit (ici on suppose que MySQL est installé dans C:\textbackslash MySQL) :
\begin{lstlisting}
cd %QTDIR%\src\plugins\sqldrivers\mysql
qmake "INCLUDEPATH+=C:/MySQL/include" "LIBS+=C:/MYSQL/MySQL Server <version>/lib/opt/libmysql.lib" mysql.pro
nmake
\end{lstlisting}

Si vous n'utilisez pas un compilateur Microsoft, remplacer \textbf{nmake} avec \textbf{make} dans la ligne ci-dessus.
\\

\textbf{Note} : Ce plugin de base de données n'est pas pris en charge pour Windows CE. \\
\textbf{Note} : En ajoutant l'option "-o Makefile" comme argument de \textbf{qmake} (pour lui dire où construire le makefile) peut générer la construction du plugin en mode release seulement. Si vous vous attendez à une version de débogage compilée, n'utilisez pas l'option "-o Makefile".

\end{document}
