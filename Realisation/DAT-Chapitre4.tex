\chapter{Exploitation et préparation}

% Indications :
%  Mise en production, déploiement, formation, support et exploitation proprement dite....
% 
% Bon courage ;-)

\section{Installation des librairies et drivers}
Afin d'assurer le bon fonctionnement de l'application, cette dernière requiert que soient présents les packages suivants (dans une version égale ou supérieure à celle précisée)~:
Les éléments nécessaires à la première mise en service sont fournis dans le livrable initial.
\\

\begin{tabularx}{\linewidth}{X X}
	\toprule
	Microsoft Windows	& GNU/Linux		\\
	\midrule
	Windows 7			&				\\
	MinGW 4.8.1 (TMD)	&				\\
	Qt 5.2.1			& Qt 5.2.1		\\
	MySQL 5.6.16		& 				\\
	\bottomrule
\end{tabularx}
\\
\\
Dans le cas de figure où ces derniers soient perdus ou endommagés, le paragraphe suivant précise les procédures nécessaire à leur re-production.

\subsection{Création du pilote MySQL pour Qt5 sur Windows}
Cette section précise comment créer le pilote MySQL pour Qt5 sur Windows en utilisant MinGW.
La rédaction de cette dernière a été motivée par le fait que le pilote en question n'est malheureusement pas fournit par défaut dans le framework Qt.

\paragraph{Obtention du matériel}
Les outils suivants sont nécessaires~:
\begin{enumerate}
	\item Windows 7~;
	\item MinGW 4.8.1 (TDM)~;
	\item Qt 5.2.1 installer \& sources~;
	\item MySQL 5.6.16.
\end{enumerate} 

\paragraph{Installation du matériel}

\subparagraph{Installation du framework Qt}
L'installation des composants par défaut convient très bien, mais il est tout à fait possible de procéder à une installation complète. Sélectionner le répertoire racine d'installation du framework (ici C:). À terme, Qt est alors installé, ici dans le répertoire C:/Qt/Qt5.2.1.
\\
Ensuite, extraire les sources Qt dans un répertoire temporaire de son choix, puis copier le dossier \textit{qt-everywhere-opensource-src-5.2.1} ainsi obtenu dans le répertoire \textit{C:/Qt/Qt5.2.1/} et le renommer \textit{sources}.
\\
En fin de manipulation, le framework Qt est installé dans le répertoire C:/Qt/Qt5.2.1/ et les sources présentes dans C:/Qt/Qt5.2.1/sources/.

\subparagraph{Installation de MySQL}
Sélectionner l'installation personnalisée.
A minima, est requis le {\quote Client C API Library}, le reste étant optionnel.
Laisser faire l'installation dans le répertoire de son choix (par défaut C:/Program Files (x86)/MySQL/My SQL Server 5.6/).

\paragraph{Construction \& déploiement}
Ouvrir le terminal de commande de Qt.
Ce dernier fonctionne comme n'importe quel terminal à ceci prêt qu'il définit par défaut toutes les variables d'environnement Qt nécessaires.
Il s'ouvre sur le répertoire Qt.
\\
Exécuter les commandes suivantes~:

% XXX @mj : essaie de push des fichiers qui compilent ...

\begin{lstlisting}[language=Bash, escapechar=$]
set mysql=C:\\PROGRA~2\\MySQL\\MYSQLS~1.6
cd C:\Qt\Qt5.2.1\sources\qtbase\src\plugins\sqldrivers\mysql\
qmake "INCLUDEPATH+=%mysql%\\include" "LIBS+=%mysql%\\lib\\libmysql.lib" -o Makefile mysql.pro
mingw32-make
\end{lstlisting}
Après construction réussie (et sans erreur), des fichiers seront produits dans C:/Qt/Qt5.2.1/sources/qtbase/plugins/sqldrivers, que sont~:
\begin{enumerate}
	\item libqsqlmysql.a~;
	\item libqsqlmysqld.a~;
	\item qsqlmysql.dll~;
	\item qsqlmysqld.dll.
\end{enumerate}
\begin{lstlisting}[language=Bash, escapechar=$]
For .dll files, move them to C:\Qt\QT5.1.1\5.1.1\mingw48_32\plugins\sqldrivers.

For .a files, move them to C:\Qt\Qt5.1.1\5.1.1\mingw48_32\lib

Also, copy libmysql.dll from %mysql%\lib to C:\Windows
Testing

To use the driver, don't forget to add QT += sql to project file, else it don't work.

Check which drivers are available by this code:

C++
#include <QtCore/QCoreApplication> #include <QtSql> int main() { QCoreApplication a(argc, argv); aDebug() << QSqlDatabase::drivers(); return a.exec(); }
1
2
3
4
5
6
7
8
	
#include <QtCore/QCoreApplication>
#include <QtSql>
 
int main() {
    QCoreApplication a(argc, argv);
    aDebug() << QSqlDatabase::drivers();
    return a.exec();
}

\end{lstlisting}
